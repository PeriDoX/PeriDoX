%%%%%%%%%%%%%%%%%%%%%%%%%%%%%%%%%%%%
% Header                           %
%%%%%%%%%%%%%%%%%%%%%%%%%%%%%%%%%%%%
% 
% Revisions: 2017-04-10 Martin R�del <martin.raedel@dlr.de>
%                       Initial draft
%               
% Contact:   Martin R�del,  martin.raedel@dlr.de
%            DLR Composite Structures and Adaptive Systems
%          
%                                 __/|__
%                                /_/_/_/  
%            www.dlr.de/fa/en      |/ DLR
%
%%%%%%%%%%%%%%%%%%%%%%%%%%%%%%%%%%%%
% Content                          %
%%%%%%%%%%%%%%%%%%%%%%%%%%%%%%%%%%%%

\leveldown{\idxPDKwRestart}
\label{sec:Peridigm:QRG:SolverTools:Restart}
\myindex[\idxPDKeywordName]{\idxPDKwRestart}

\leveldown{Description}

Writes the last state of an analysis for use in a restart analysis.

\levelstay{Literature}

-
% \begin{itemize}[noitemsep]
%   \item \cite{SillingSA2007}
% \end{itemize}

\levelstay{Code}

\leveldown{Release version}

Available from the current \href{\toolrepoaddress}{master}.

\levelstay{Required compiler options}

-

\levelstay{Routines}

\begin{itemize}[noitemsep]
  \item from \texttt{src/core/}:
  \begin{itemize}[noitemsep]
    \item \texttt{Peridigm.cpp/hpp}
    \item \texttt{Peridigm\_State.cpp/hpp}
  \end{itemize}
\end{itemize}

\levelup{Input parameters}

\leveldown{List}

\begin{tabularx}{\linewidth}{lcccX}
\toprule
Name            & Type          & Required      & Default       & Description           \\
\midrule
Restart         & bool          &  \checkmark\textsuperscript{\ref{enm:Peridigm:QRG:SolverTools:Restart:Remark:One}}   & -             &       \\
\bottomrule
\end{tabularx}

\levelstay{Remarks}

\begin{enumerate}[noitemsep]
  \item \label{enm:Peridigm:QRG:SolverTools:Restart:Remark:One} The first and the restart model must be located in the same directory
\end{enumerate}

\levelup{Exemplary input section}

\leveldown{XML-format}

from \texttt{test/regression/Contact\_Perforation\_With\_Restart/}:

Initial run:

\begingroup
\lstset{breaklines=true}
\begin{code}
<ParameterList name="Restart">
  <Parameter name="Restart" type="bool" value="true"/>
</ParameterList>

<ParameterList name="Solver">
  <Parameter name="Verbose" type="bool" value="false"/>
  <Parameter name="Initial Time" type="double" value="0.0"/>
  <Parameter name="Final Time" type="double" value="3.5e-2"/>
  <ParameterList name="Verlet">
    <Parameter name="Fixed dt" type="double" value="3.5e-5"/>
  </ParameterList>
</ParameterList>
\end{code}
\endgroup

Restart run:

\begingroup
\lstset{breaklines=true}
\begin{code}
<ParameterList name="Restart">
  <Parameter name="Restart" type="bool" value="true"/>
</ParameterList>

<ParameterList name="Solver">
  <Parameter name="Verbose" type="bool" value="false"/>
  <Parameter name="Initial Time" type="double" value="3.5e-2"/>
  <Parameter name="Final Time" type="double" value="7.0e-2"/>
  <ParameterList name="Verlet">
    <Parameter name="Fixed dt" type="double" value="3.5e-5"/>
  </ParameterList>
</ParameterList>
\end{code}
\endgroup

\levelstay{Free format}

-

\levelstay{YAML format}

-

\levelup{List of examples}

\begin{itemize}[noitemsep]
%   \item From \texttt{examples/}:
%   \begin{itemize}[noitemsep]
%     \item \texttt{fragmenting\_cylinder/fragmenting\_cylinder.peridigm}
%   \end{itemize}
  \item From \texttt{test/regression/}:
  \begin{itemize}[noitemsep]
    \item \texttt{Contact\_Perforation\_Run1.xml}
    \item \texttt{Contact\_Perforation\_Run2.xml}
  \end{itemize}
%   \item From \texttt{test/verification/}:
%   \begin{itemize}[noitemsep]
%     \item \texttt{NeighborhoodVolume/NeighborhoodVolume.xml}
%     \item \texttt{IsotropicHardeningPlasticFullyPrescribedTension\_NoFlaw/IsotropicHardeningPlasticFullyPrescribedTension\_NoFlaw.xml}
%   \end{itemize}
\end{itemize}