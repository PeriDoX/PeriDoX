%%%%%%%%%%%%%%%%%%%%%%%%%%%%%%%%%%%%
% Header                           %
%%%%%%%%%%%%%%%%%%%%%%%%%%%%%%%%%%%%
% 
% Revisions: 2017-04-10 Martin R�del <martin.raedel@dlr.de>
%                       Initial draft
%               
% Contact:   Martin R�del,  martin.raedel@dlr.de
%            DLR Composite Structures and Adaptive Systems
%          
%                                 __/|__
%                                /_/_/_/  
%            www.dlr.de/fa/en      |/ DLR
% 
%%%%%%%%%%%%%%%%%%%%%%%%%%%%%%%%%%%%
% Content                          %
%%%%%%%%%%%%%%%%%%%%%%%%%%%%%%%%%%%%

\levelstay{Conversion between peridynamic and finite element values}

\paragraph{Material properties}

\begin{itemize}[noitemsep]
 \item 3D \tabto{0.25\linewidth} Without plane stress or strain assumptions
 \item 2D \tabto{0.25\linewidth} With plane stress or strain assumptions
\end{itemize}

Note, that in case of bond-based analysis, the conversion is problem-dependent:

\begin{equation}
\glssymbol{symb:scalar:mat:poissonratio}=\begin{cases}\frac14&\text{3D \& 2D plane strain}\\\frac13&\text{2D plane stress}\end{cases}
\end{equation}

\begin{filecontents}{\tabledir\jobname-materialpropertyconversionltxtable.tex}
% \begin{longtable}{@{}lX@{}}
\begin{longtable}{@{}Xcr@{}lc@{}}%Xrll}
% Caption
\caption{Material property conversion}\\
% \label{tab:Use_Peridigm_Material_property_conversion}\\
\label{tab:Peridigm:QRG:Preliminaries:PDCMConversion}\\
% ---------------------------
% Header & Footer
% ---------------------------
%
% Header
% -----------------
% 1st head
\toprule
Description & Dim. & \multicolumn{2}{l}{Equation} & Source\\
\midrule
\endfirsthead
% Last head
\multicolumn{5}{@{}l}{\ldots continued}\\
\toprule
Description & Dim. & \multicolumn{2}{l}{Equation} & Source\\
\midrule
\endhead
%
% Footer
% -----------------
% n-th foot
\bottomrule
\multicolumn{5}{r@{}}{continued \ldots}\\
\endfoot
% last foot
\bottomrule
\endlastfoot
% ---------------------------
% Content
% ---------------------------\kappa_{3D}=\kappa_{2D}=\kappa_{1D}=\mu=
\idxPDKwBulkModulus{} & 3D & $\glssymbol{symb:scalar:mat:modulus:bulk}_{3D}=$ & $\dfrac{\glssymbol{symb:scalar:mat:modulus:young}}{3\left(1-2\glssymbol{symb:scalar:mat:poissonratio}\right)}$ & \\[2ex]
& 2D & $\glssymbol{symb:scalar:mat:modulus:bulk}_{2D}=$ & $\begin{cases}
\dfrac{\glssymbol{symb:scalar:mat:modulus:young}}{2\left(1-\glssymbol{symb:scalar:mat:poissonratio}\right)}  & \text{, plane stress} \\[2ex]
\dfrac{\glssymbol{symb:scalar:mat:modulus:young}}{2\left(1-\glssymbol{symb:scalar:mat:poissonratio}-2\glssymbol{symb:scalar:mat:poissonratio}^2\right)} & \text{, plane strain}
\end{cases}$ & \cite{BobaruF2017} \\[4ex]
& 1D & $\glssymbol{symb:scalar:mat:modulus:bulk}_{1D}=$ & $\glssymbol{symb:scalar:mat:modulus:young}$  & \\[2ex]
\idxPDKwShearModulus{} & all & $\glssymbol{symb:scalar:mat:modulus:shear}=$ & $\dfrac{\glssymbol{symb:scalar:mat:modulus:young}}{2\left(1+\glssymbol{symb:scalar:mat:poissonratio}\right)}$ & \\[2ex]
% 
\multicolumn{4}{@{}l}{\idxPDKwBondConstant{} - bond based, constant influence function} \\
& 3D & $\glssymbol{symb:scalar:pd:bond:constant}_{\mbox{3D}}=$ & $\dfrac{18\glssymbol{symb:scalar:mat:modulus:bulk}}{\pi\glssymbol{symb:scalar:pd:horizon}^4}=\dfrac{30\glssymbol{symb:scalar:mat:modulus:shear}}{\pi\glssymbol{symb:scalar:pd:horizon}^4}$ & \cite{SillingSA2005,MadenciE2014,BobaruF2017}\footnote{\cite[p.30]{BobaruF2017}, \cite[p.1529]{SillingSA2005}, \cite[p.37]{MadenciE2014}} \\[2ex]%\cite[p.1529]{SillingSA2005}\newline\cite[p.37]{MadenciE2014} \\[2ex]
& 2D & $\glssymbol{symb:scalar:pd:bond:constant}_{\mbox{2D}}=$ & $\dfrac{12\glssymbol{symb:scalar:mat:modulus:bulk}}{\pi \glssymbol{symb:scalar:thickness}\glssymbol{symb:scalar:pd:horizon}^3}=\dfrac{\glssymbol{symb:scalar:mat:modulus:shear}}{\pi \glssymbol{symb:scalar:thickness}\glssymbol{symb:scalar:pd:horizon}^3}\cdot\begin{cases}20&\text{plane strain}\\24&\text{plane stress}\end{cases}$ & \cite{BobaruF2017} \\[2ex]
& 1D & $\glssymbol{symb:scalar:pd:bond:constant}_{\mbox{1D}}=$ & $\dfrac{2\glssymbol{symb:scalar:mat:modulus:bulk}}{\glssymbol{symb:scalar:geo:2D:surface}\glssymbol{symb:scalar:pd:horizon}^2}$  & \cite{BobaruF2017} \\[2ex]
% 
\multicolumn{4}{@{}l}{\idxPDKwCriticalStretch{} - bond based} & \cite{HuangD2015}\footnote{\cite[p.114]{HuangD2015}} \\[3ex]
& 3D & $\glssymbol{symb:scalar:pd:stretch}_{\glssymbol{symb:index:critical}\mbox{3D}}=$ & $\sqrt{\dfrac{5\glssymbol{symb:scalar:mat:energyreleaserate:critical}}{9\glssymbol{symb:scalar:mat:modulus:bulk}_{3D}\glssymbol{symb:scalar:pd:horizon}}}=\sqrt{\dfrac{10\glssymbol{symb:scalar:mat:energyreleaserate:critical}}{\pi \glssymbol{symb:scalar:pd:bond:constant}_{3D}\glssymbol{symb:scalar:pd:horizon}^5}}$ & \cite{SillingSA2005,BobaruF2017} \\[2ex]
& 2D & $\glssymbol{symb:scalar:pd:stretch}_{\glssymbol{symb:index:critical}\mbox{2D}}=$ & $\sqrt{\dfrac{\pi \glssymbol{symb:scalar:mat:energyreleaserate:critical}}{3\glssymbol{symb:scalar:mat:modulus:bulk}_{2D}\glssymbol{symb:scalar:pd:horizon}}}=\sqrt{\dfrac{4\glssymbol{symb:scalar:mat:energyreleaserate:critical}}{\glssymbol{symb:scalar:pd:bond:constant}_{2D}\glssymbol{symb:scalar:thickness}\glssymbol{symb:scalar:pd:horizon}^4}}$  & \cite{BobaruF2017} \\[2ex]
& 1D & $\glssymbol{symb:scalar:pd:stretch}_{\glssymbol{symb:index:critical}\mbox{1D}}=$ & $\sqrt{\dfrac{3\glssymbol{symb:scalar:mat:energyreleaserate:critical}}{\glssymbol{symb:scalar:mat:modulus:bulk}_{1D}\glssymbol{symb:scalar:pd:horizon}}}=\sqrt{\dfrac{6\glssymbol{symb:scalar:mat:energyreleaserate:critical}}{\glssymbol{symb:scalar:pd:bond:constant}_{1D}\glssymbol{symb:scalar:geo:2D:surface}\glssymbol{symb:scalar:pd:horizon}^3}}$ & \cite{BobaruF2017} \\[2ex]
% 
\multicolumn{5}{@{}l}{\idxPDKwCriticalStretch{} - state based} \\
& 3D & $\glssymbol{symb:scalar:pd:stretch}_{\glssymbol{symb:index:critical}\mbox{3D}}=$ & $\sqrt{\dfrac{\glssymbol{symb:scalar:mat:energyreleaserate:critical}}{\left[3\glssymbol{symb:scalar:mat:modulus:shear}+\left(\frac34\right)^4\left(\glssymbol{symb:scalar:mat:modulus:bulk}-\frac{5\glssymbol{symb:scalar:mat:modulus:shear}}{3}\right)\right]\glssymbol{symb:scalar:pd:horizon}}}$ & \cite{MadenciE2014,MadenciE2016}\footnote{\cite[p.120]{MadenciE2014}} \\[2ex]%\label{ftn:MadenciCriticalStretch}
& 2D & $\glssymbol{symb:scalar:pd:stretch}_{\glssymbol{symb:index:critical}\mbox{2D}}=$ & $\sqrt{\dfrac{\glssymbol{symb:scalar:mat:energyreleaserate:critical}}{\left[\frac{6}{\pi}\glssymbol{symb:scalar:mat:modulus:shear}+\frac{16}{9\pi^2}\left(\glssymbol{symb:scalar:mat:modulus:bulk}-2\glssymbol{symb:scalar:mat:modulus:shear}\right)\right]\glssymbol{symb:scalar:pd:horizon}}}$ & \cite{MadenciE2014,MadenciE2016}        \\[2ex]%\footnotemark{\ref{ftn:MadenciCriticalStretch}}
& 1D & $\glssymbol{symb:scalar:pd:stretch}_{\glssymbol{symb:index:critical}\mbox{1D}}=$ & \\
\end{longtable}
\end{filecontents}

\begingroup
\LTXtable{\linewidth}{\tabledir\jobname-materialpropertyconversionltxtable.tex}
\endgroup

Internally, \toolname{} uses $\glssymbol{symb:scalar:mat:modulus:bulk}$, $\glssymbol{symb:scalar:mat:modulus:shear}$ for the stiffness description. The conversion between engineering constants and $\glssymbol{symb:scalar:mat:modulus:bulk}$, $\glssymbol{symb:scalar:mat:modulus:shear}$ is performed in methods

\begin{itemize}[noitemsep]
  \item \verb+PeridigmNS::Material::calculateBulkModulus+
  \item \verb+PeridigmNS::Material::calculateShearModulus+
\end{itemize}

in class \verb+/src/Materials/Peridigm_Material.cpp+.

Currently, \toolname{} only uses the 3D formulation of the peridynamics, e.g. for surface correction factors. Therefore, only the 3D equations shall be used for material modelling.