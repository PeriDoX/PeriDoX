%%%%%%%%%%%%%%%%%%%%%%%%%%%%%%%%%%%%
% Header                           %
%%%%%%%%%%%%%%%%%%%%%%%%%%%%%%%%%%%%
% 
% Revisions: 2017-04-10 Martin Rädel <martin.raedel@dlr.de>
%                       Initial draft
%               
% Contact:   Martin Rädel,  martin.raedel@dlr.de
%            DLR Composite Structures and Adaptive Systems
%          
%                                 __/|__
%                                /_/_/_/  
%            www.dlr.de/fa/en      |/ DLR
% 
%%%%%%%%%%%%%%%%%%%%%%%%%%%%%%%%%%%%
% Content                          %
%%%%%%%%%%%%%%%%%%%%%%%%%%%%%%%%%%%%

\levelstay{\texorpdfstring{\protect\marktool{\cmakename}}{\cmakename{}}}

\marktool{\cmakename} is cross-platform free and open-source software for managing the build process of software using a compiler-independent method. It is maintained by Kitware. The official homepage of is \marktool{\cmakename}

\href{https://cmake.org/}{https://cmake.org/}

\marktool{\trilinosname} release 12.4 and higher use a \marktool{\cmakename} build system, which requires \marktool{\cmakename} version \textbf{2.8.11 or newer}.

\paragraph{Installation from source}

In order to install \marktool{\cmakename} from the official or any other binary source open a terminal and login as root. Change directory to the designated download folder, e.g. \verb+/usr/local/src/+ and perform the following steps:

\begin{code}
cd $DOWNLOAD_DIR
wget http://www.cmake.org/files/v3.4/cmake-3.4.3.tar.gz	%\lstcomment{\# Download}%
tar xvfz cmake-3.4.3.tar.gz 	%\lstcomment{\# unzip}%
cd cmake-3.4.3			%\lstcomment{\# go into directory}%
./configure --prefix=/usr/local/bin/cmake-3.4.3 > configure_cmake.log 2>&1
make > make_cmake.log 2>&1		%\lstcomment{\# build}%
make install > make_install_cmake.log 2>&1 %\lstcomment{\# install}%
\end{code}

If you want to see the available configuration options, run the command below in the terminal.

\begin{code}
./configure --help
\end{code}

In order to configure the installation directory of \marktool{\cmakename} before installation, run the command below

\begin{code}
./configure --prefix=/opt/cmake
\end{code}

After installation without any errors you can verify the installation by running the command below:

\begin{code}
/usr/local/bin/cmake-3.4.3/bin/cmake -version
\end{code}

The output should look something like below (depending upon \marktool{\cmakename} version you are installing).

\begin{code}
cmake version 3.4.3
\end{code}

Afterwards, the \marktool{\cmakename}-directory has to be added to the \verb+PATH+ environment variable

\begingroup
\lstset{breaklines = true}
\lstinputlisting[
  style=scriptstyle,
  linerange={37-37},
  nolol,
  ]{Scripts/modified_bashrc.txt}
\endgroup

\paragraph{Installation with \texorpdfstring{\protect\marktool{\opensusename}}{\opensusename{}}-repository}

For the installation of the build process manager \marktool{\zyppername} visit:

\href{http://software.opensuse.org/download.html?project=server\%3Airc\&package=cmake}{http://software.opensuse.org/download.html?project=server\%3Airc\&package=cmake}

You can either choose the 1-Click-installation or add the repository and install manually. For the latter login to a terminal as \verb+root+ and type

\begingroup
\lstset{breaklines=true}
\begin{code}
zypper addrepo http://download.opensuse.org/repositories/server:irc/openSUSE_Leap_42.1/server:irc.repo
zypper refresh
zypper install cmake
\end{code}
\endgroup 