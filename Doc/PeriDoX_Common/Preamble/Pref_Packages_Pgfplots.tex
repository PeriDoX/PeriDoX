%%%%%%%%%%%%%%%%%%%%%%%%%%%%%%%%%%%%
% Header                           %
%%%%%%%%%%%%%%%%%%%%%%%%%%%%%%%%%%%%
% 
% This file handles all things considering the pgfplots package
% 
% Revisions: 2017-04-10 Martin Raedel <martin.raedel@dlr.de>
%                       Initial draft
%               
% Contact:   Martin Raedel,  martin.raedel@dlr.de
%            DLR Composite Structures and Adaptive Systems
%          
%                                 __/|__
%                                /_/_/_/  
%            www.dlr.de/fa/en      |/ DLR
% 
%%%%%%%%%%%%%%%%%%%%%%%%%%%%%%%%%%%%
% Content                          %
%%%%%%%%%%%%%%%%%%%%%%%%%%%%%%%%%%%%

% ---------------------------
% Load package
% ---------------------------

% Pgfplots
\makeatletter
\@ifpackageloaded{pgfplots}{
  \usepgfplotslibrary{fillbetween}
  \usepgfplotslibrary{groupplots}
  \usepgfplotslibrary{patchplots}
}{
  \usepackage{pgfplots}
  \usepgfplotslibrary{fillbetween}
  \usepgfplotslibrary{groupplots}
  \usepgfplotslibrary{patchplots}
}
\makeatother

% Pgfplotstable
\makeatletter
\@ifpackageloaded{pgfplotstable}{}{
  \usepackage{pgfplotstable}
}
\makeatother


% Temporary fix for pgfplots & forest problems reported in September 2016
% here: http://tex.stackexchange.com/questions/328972/presence-of-pgfplots-package-breaks-forest-environment-w-folder-option-en
% \makeatletter
% \let\pgfmathModX=\pgfmathMod@
% \usepackage{pgfplots}%
% \let\pgfmathMod@=\pgfmathModX
% \makeatother

% ---------------------------
% Pgf version
% ---------------------------

\pgfplotsset{compat=1.14}

% ---------------------------
% colormaps
% ---------------------------

\pgfplotsset{
  colormap={abaqusblueredcolormap}{
    rgb255( 0cm)=(  0,  0,255);
    rgb255( 1cm)=(  0, 93,255);
    rgb255( 2cm)=(  0,185,255);
    rgb255( 3cm)=(  0,255,232);
    rgb255( 4cm)=(  0,255,139);
    rgb255( 5cm)=(  0,255,139);
    rgb255( 6cm)=(  0,255, 46);
    rgb255( 7cm)=( 46,255,  0);
    rgb255( 8cm)=(139,255,  0);
    rgb255( 9cm)=(232,255,  0);
    rgb255(10cm)=(255,185,  0);
    rgb255(11cm)=(255, 93,  0);
    rgb255(12cm)=(255,  0,  0);
  }
}

\pgfplotsset{
  colormap={paraviewblueredcolormap}{
    rgb255( 0cm)=(  0,  0,255);
    rgb255( 1cm)=(  0, 93,255);
    rgb255( 2cm)=(  0,185,255);
    rgb255( 3cm)=(  0,255,232);
    rgb255( 4cm)=(  0,255,139);
    rgb255( 5cm)=(  0,255,139);
    rgb255( 6cm)=(  0,255, 46);
    rgb255( 7cm)=( 46,255,  0);
    rgb255( 8cm)=(139,255,  0);
    rgb255( 9cm)=(232,255,  0);
    rgb255(10cm)=(255,185,  0);
    rgb255(11cm)=(255, 93,  0);
    rgb255(12cm)=(255,  0,  0);
  }
}

\pgfplotsset{
  colormap={whiteblack}{color(0cm)=(white);color(1cm)=(black)}
}

% ---------------------------
% pgfplotsset
% ---------------------------

%~~~~~ Number format ~~~~~~~~

% call with e.g.: y tick label style={numberformatfixed={3}}
\pgfplotsset{
    numberformatfixed/.style 2 args={
      /pgf/number format/fixed,
      /pgf/number format/fixed zerofill,% Allow trailing zeros
      /pgf/number format/precision=#1,   % Nr of decimal digits
    },
    numberformatfixed/.default={2}
}

%~~~~~ Colorbar ~~~~~~~~~~~~~

\pgfplotsset{
  basecolorbaraxis style/.style={
    hide axis,
    scale only axis,
    colormap/bluered,                         % Colormap preset
    colorbar sampled,                         % Steps in colorbar
  }
}

\pgfplotsset{
  damageaxis style/.style={
    basecolorbaraxis style,
    point meta min=0.0,                       % Minimum value colorbar
    point meta max=1.0,                       % Maximum value colorbar
  }
}

\pgfplotsset{
  damagefigureaxis style/.style={
    %basecolorbaraxis style,
    %point meta min=0.0,                       % Minimum value colorbar
    %point meta max=1.0,                       % Maximum value colorbar
    damageaxis style,
    enlargelimits=false,                      % Do not add margins
    axis equal image,                         % Keep aspect ratio
  }
}

\pgfplotsset{
  mybarundertensionaxis style/.style={
    hide axis,
    scale only axis,
    %anchor=east,
    anchor=west,
    height=0.125\textheight,
    %xshift=0.25cm,
    xshift=-1.0cm,
    colormap/bluered, % Colormap preset
    colorbar sampled, % Steps in colorbar
    colorbar right, % Activate colorbar
  }
}

\pgfplotsset{
  damagecolorbar style/.style={
    separate axis lines,
    samples=256,                              % Number of steps+1
  }
}

\pgfplotsset{
  abaqusdiscrete12colorbar style/.style={
    separate axis lines,
    samples=13,                               % Number of steps+1
  }
}

\pgfplotsset{
  abaqusdiscrete256colorbar style/.style={
    separate axis lines,
    samples=256,                              % Number of steps+1
  }
}

\pgfplotsset{
  ansysdiscrete9colorbar style/.style={
    separate axis lines,
    samples=10,                               % Number of steps+1
  }
}

\pgfplotsset{
  paraviewdiscrete256colorbar style/.style={
    separate axis lines,
    samples=256,                              % Number of steps+1
  }
}

%~~~~~ Chart Style ~~~~~~~~~~

\pgfplotsset{
  chart style/.style={
    width=0.7\linewidth,
    height=0.3\textheight,
    axis x line=middle,                       % Middle x-axis
    axis y line=left,
    enlarge x limits={auto,upper},            % Add this at positive x
    enlarge y limits,                         % Add this at y
    x label style={                           % xlabel style
      at={(axis description cs:0.5,-0.025)},  % Position
      anchor=north                            % Anchor
    },
    y label style={                           % ylabel style
      anchor=south                            % Anchor
    },
    x tick label style={
      /pgf/number format/fixed,
      /pgf/number format/precision=5,         % Nr of decimal digits
    },
    y tick label style={
      /pgf/number format/fixed,
      /pgf/number format/fixed zerofill,      % Allow trailing zeros
      /pgf/number format/precision=1,         % Nr of decimal digits
    },
    scaled ticks=false,
    legend columns=1,                         % Nr of colums in legend
    legend style={
      draw=none,                              % No border
      fill=none,                              % No fill color
      at={(1.025,0.5)},                       % Position
      anchor=west,                            % Anchor
      /tikz/row 2/.style={
        row sep=5pt,
      }
    }
  }
}

\pgfplotsset{
  xaxis style/.style={
    each nth point=1,
    restrict expr to domain={rawx}{0:5},      % restrict x-axis values
  }
}

%~~~~~ WaveChart Style ~~~~~~~~~~

\pgfplotsset{
  wavechart style/.style={
    chart style,
    xlabel={Time [ms]},
    ylabel={Displacement [mm]},
    ylabel near ticks,
    xticklabel={%
      \pgfkeys{/pgf/fpu=true}%               % Allow numbers > 16383.99
      \pgfmathparse{\tick*1000}%             % Scale s to ms
      \pgfkeys{/pgf/fpu=false}%              % Switch of fpu
      $\pgfmathprintnumber[
        fixed,                               % Number format
        precision=1,                         % Nr of decimal digits
      ]{\pgfmathresult}$%
    },
    yticklabel={%
      \pgfkeys{/pgf/fpu=true}%               % Allow numbers > 16383.99
      \pgfmathparse{\tick*1000}%             % Scale m to mm
      \pgfkeys{/pgf/fpu=false}%              % Switch of fpu
      $\pgfmathprintnumber[
        fixed,                               % Number format
        precision=1,                         % Nr of decimal digits
      ]{\pgfmathresult}$%
    },
  }
}

\pgfplotsset{
  wavexaxis style/.style={
    each nth point=1,
    restrict expr to domain={rawx}{0:0.005},   % restrict x-axis values [ms]
  }
}

%~~~~~ Tension chart ~~~~~~~~~

\pgfplotsset{
  tensionchart style/.style={
    chart style,
    enlarge y limits={auto,upper},           % Add this at positive y
    xlabel={Time [s]},
    ylabel={Displacement [mm]},
    xlabel near ticks,
    ylabel near ticks,
    yticklabel={%
      \pgfkeys{/pgf/fpu=true}%               % Allow numbers > 16383.99
      \pgfmathparse{\tick*1000}%             % Scale m to mm
      \pgfkeys{/pgf/fpu=false}%              % Switch of fpu
      $\pgfmathprintnumber[
        fixed,                               % Number format
        precision=1,                         % Nr of decimal digits
      ]{\pgfmathresult}$%
    },
  }
}

\pgfplotsset{
  tensionchart2 style/.style={
    tensionchart style,
    xlabel={Position bar in $x$-direction [m]},
    ylabel={Displacement [mm]},
  }
}

\pgfplotsset{
  dispxaxis style/.style={
    each nth point=1,
  }
}

%~~~~~ Energy chart ~~~~~~~~~

\pgfplotsset{
  energychart style/.style={
    chart style,
    xlabel={Time [ms]},
    ylabel={Energy [J]},
    xlabel near ticks,
    ylabel near ticks,
    enlarge y limits={auto,upper},            % Add this at positive y
    %x tick label style={
    %  /pgf/number format/fixed,
    %  /pgf/number format/precision=5,         % Nr of decimal digits
    %},
    y tick label style={
      /pgf/number format/fixed,
      %/pgf/number format/fixed zerofill,      % Allow trailing zeros
      /pgf/number format/precision=0,         % Nr of decimal digits
    },
    legend cell align=left,
    cycle list name=color list,               % colors !!! use addplot+ to cycle through colors !!!
    scaled x ticks=false,
    xticklabel={%
      \pgfkeys{/pgf/fpu=true}%               % Allow numbers > 16383.99
      \pgfmathparse{\tick*1000}%             % Scale s to ms
      \pgfkeys{/pgf/fpu=false}%              % Switch of fpu
      $\pgfmathprintnumber[
        fixed,                               % Number format
        precision=1,                         % Nr of decimal digits
      ]{\pgfmathresult}$%
    },
  }
}

\pgfplotsset{
  advenergychart style/.style={
    energychart style,
    scaled y ticks=false,
    yticklabel={%
      \pgfkeys{/pgf/fpu=true}%               % Allow numbers > 16383.99
      \pgfmathparse{\tick/1000}%             % Scale mJ to J
      \pgfkeys{/pgf/fpu=false}%              % Switch of fpu
      $\pgfmathprintnumber[
        fixed,                               % Number format
        precision=0,                         % Nr of decimal digits
      ]{\pgfmathresult}$%
    }
  }
}

%~~~~~ Displacement chart ~~~

\pgfplotsset{
  displchart style/.style={
    chart style,
    xlabel={Time [ms]},
    ylabel={Displacement [mm]},
    xlabel near ticks,
    ylabel near ticks,
    enlarge y limits={auto,upper},            % Add this at positive y
    %x tick label style={
    %  /pgf/number format/fixed,
    %  /pgf/number format/precision=5,         % Nr of decimal digits
    %},
    y tick label style={
      /pgf/number format/fixed,
      %/pgf/number format/fixed zerofill,      % Allow trailing zeros
      /pgf/number format/precision=1,         % Nr of decimal digits
    },
    legend cell align=left,
    scaled x ticks=false,
    xticklabel={%
      \pgfkeys{/pgf/fpu=true}%               % Allow numbers > 16383.99
      \pgfmathparse{\tick*1000}%             % Scale s to ms
      \pgfkeys{/pgf/fpu=false}%              % Switch of fpu
      $\pgfmathprintnumber[
        fixed,                               % Number format
        precision=1,                         % Nr of decimal digits
      ]{\pgfmathresult}$%
    },
  }
}

%~~~~~ uva chart ~~~~~~~~~~~~

\pgfplotsset{
  uvachartstyle/.style={
    axis lines=middle,
    ticks=none,
    domain=\xn:\xf,
    %restrict x to domain=\xn:\xf,
    restrict y to domain=\yn:\yf,
    xmin=\xn,
    xmax=\xf,
    ymin=\yn,
    ymax=\yf,
    width=\linewidth,
    height=\linewidth,
    xlabel=\xlabel,
    ylabel=\ylabel,
    every axis x label/.style={
      at={(ticklabel* cs:1.01)},
      anchor=west,
    },
    every axis y label/.style={
      at={(ticklabel* cs:1.01)},
      anchor=south,
    },
    no markers,				% equivalent to mark=none for all addplots
  },
}

\pgfplotsset{
  unode style/.style={
    each nth point=1,
    filter discard warning = false,
  }
}

\pgfplotsset{
  xfemnode style/.style={
    unode style,
    %mymark={o}{solid},                      % equal spaced marks
  }
}

\pgfplotsset{
  xfemmarkednode style/.style={
    xfemnode style,
    mymark={o}{solid},                      % equal spaced marks
  }
}

\pgfplotsset{
  perinode style/.style={
    unode style,
    dashed,
  }
}

\pgfplotsset{
  perimarkednode style/.style={
    perinode style,
    mymark={x}{solid},                      % equal spaced marks
  }
}

%~~~~~ InfluenceFunction ~~~~

\pgfplotsset{
  influencefunctionaxis style/.style={
    xlabel=\figxlabel,
    ylabel=\figylabel,
    width=\figwidth,
    height=\figheight,
    xmin=0,
    xmax=1.1,
    ymin=0,
    ymax=1.1,
    domain=0:1,
    samples=25,
    no markers,
    thick,
    label style={font=\figurefontsize},
  }
}