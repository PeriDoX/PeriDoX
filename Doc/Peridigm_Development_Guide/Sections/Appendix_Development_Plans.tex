%%%%%%%%%%%%%%%%%%%%%%%%%%%%%%%%%%%%
% Header                           %
%%%%%%%%%%%%%%%%%%%%%%%%%%%%%%%%%%%%
% 
% Revisions: 2017-04-10 Martin R�del <martin.raedel@dlr.de>
%                       Initial draft
%               
% Contact:   Martin R�del,  martin.raedel@dlr.de
%            DLR Composite Structures and Adaptive Systems
%          
%                                 __/|__
%                                /_/_/_/  
%            www.dlr.de/fa/en      |/ DLR
%
%%%%%%%%%%%%%%%%%%%%%%%%%%%%%%%%%%%%
% Content                          %
%%%%%%%%%%%%%%%%%%%%%%%%%%%%%%%%%%%%

% \chapter{Development plan}
% \setcounter{currentlevel}{6}

An issue tracker has been setup:

\href{https://free-redmine.saas-secure.com/projects/peridox}{https://free-redmine.saas-secure.com/projects/peridox}

Contact the author of this document about information how to access and edit the issue tracker.

% \begin{table}[htbp]
% \caption{Requirement list}
% \label{tab:requirementList}
% \begin{tabularx}{\linewidth}{XXX}
% \toprule
% What&Why & Where \\
% \midrule
% Read coordinate system&read local coordinate systems from exodus&io/Peridigm\_DiscretizationFactory.cpp;io/Peridigm\_ExodusDiscretization.cpp	\\
% local material coordinate transformation&describe anisotropic material accurately &material/Peridigm\_MaterialFactory.cpp; core/\\
% delete bonds in correspondence material&simulate crack propagation with correspondence material&material/Peridigm\_MaterialFactory.cpp	\\
% \bottomrule
% \end{tabularx}
% \end{table}
