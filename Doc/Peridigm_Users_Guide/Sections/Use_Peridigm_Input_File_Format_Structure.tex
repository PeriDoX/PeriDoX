%%%%%%%%%%%%%%%%%%%%%%%%%%%%%%%%%%%%
% Header                           %
%%%%%%%%%%%%%%%%%%%%%%%%%%%%%%%%%%%%
% 
% Revisions: 2017-04-10 Martin R�del <martin.raedel@dlr.de>
%                       Initial draft
%               
% Contact:   Martin R�del,  martin.raedel@dlr.de
%            DLR Composite Structures and Adaptive Systems
%          
%                                 __/|__
%                                /_/_/_/  
%            www.dlr.de/fa/en      |/ DLR
% 
%%%%%%%%%%%%%%%%%%%%%%%%%%%%%%%%%%%%
% Content                          %
%%%%%%%%%%%%%%%%%%%%%%%%%%%%%%%%%%%%

\leveldown{Input deck structure}
\label{sec:Peridigm:QRG:InputFileFormat:Structure}

This section is basically taken from section 3.2 in \cite{PeridigmUserGuide100}. Independent of the actual input file format, a \toolname{} input deck takes the form:

\begin{itemize}[noitemsep]
  \item \texttt{[Discretization Section]}           \tabto{0.75\linewidth} required
  \item \texttt{[Materials Section]}                \tabto{0.75\linewidth} required
  \item \texttt{[Damage Models Section]}            \tabto{0.75\linewidth} optional
  \item \texttt{[Blocks Section]}                   \tabto{0.75\linewidth} required
  \item \texttt{[Contact Section]}                  \tabto{0.75\linewidth} optional
  \item \texttt{[Boundary Conditions Section]}      \tabto{0.75\linewidth} required
  \item \texttt{[Compute Class Parameters Section]} \tabto{0.75\linewidth} optional
  \item \texttt{[Solver Section]}                   \tabto{0.75\linewidth} required
  \item \texttt{[Output Section]}                   \tabto{0.75\linewidth} required
\end{itemize}

We elaborate on each of these sections below:

\begin{filecontents}{\tabledir\jobname-inputfilestructureltxtable.tex}
\begin{longtable}{@{}lX@{}}
\caption{Input file section overview}\\
\toprule
Section & Description   \\
\midrule
\endfirsthead
%
\multicolumn{2}{@{}l}{\ldots continued}\\
\toprule
Section & Description   \\
\midrule
\endhead
%
\bottomrule
\multicolumn{2}{r@{}}{continued \ldots}\\
\endfoot
\bottomrule
\endlastfoot
%
\texttt{[Discretization Section]}           & Contains filename of input mesh, or arguments to Peridigm internal mesh generator.\\
\texttt{[Materials Section]}                & Contains  the  names  of  the  material models  used  and  arguments  for  their constitutive parameters.\\
\texttt{[Damage Models Section]}            & Contains  the  names  of  the  damage  models  used  and  arguments  for their constitutive parameters.\\
\texttt{[Blocks Section]}                   & Contains  a  listing  of  the  model properties,  associating  each  block  with  material and damage models as well as a horizon.\\
\texttt{[Contact Section]}                  & Contains a listing of the contact models and the kind of contact model to be applied when model regions come into contact in a simulation.\\
\texttt{[Boundary Conditions Section]}      & Contains the initial and boundary conditions and loads for the simulation.\\
\texttt{[Compute Class Parameters Section]} & Contains user defined calculation data which shall be output in a simulation.\\
\texttt{[Solver Section]}                   & Contains the solver to be used along with solver parameters.\\
\texttt{[Output Section]}                   & Contains the output filename and output frequency along with a listing of the variables to be output.
\end{longtable}
\end{filecontents}

\begingroup
\LTXtable{\linewidth}{\tabledir\jobname-inputfilestructureltxtable.tex}
\endgroup

Each of these sections is described in a separate part of chapter \ref{sec:Peridigm:QRG}.
