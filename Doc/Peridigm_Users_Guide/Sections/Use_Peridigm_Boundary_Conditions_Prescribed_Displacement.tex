%%%%%%%%%%%%%%%%%%%%%%%%%%%%%%%%%%%%
% Header                           %
%%%%%%%%%%%%%%%%%%%%%%%%%%%%%%%%%%%%
% 
% Revisions: 2017-04-10 Martin R�del <martin.raedel@dlr.de>
%                       Initial draft
%               
% Contact:   Martin R�del,  martin.raedel@dlr.de
%            DLR Composite Structures and Adaptive Systems
%          
%                                 __/|__
%                                /_/_/_/  
%            www.dlr.de/fa/en      |/ DLR
% 
%%%%%%%%%%%%%%%%%%%%%%%%%%%%%%%%%%%%
% Content                          %
%%%%%%%%%%%%%%%%%%%%%%%%%%%%%%%%%%%%

\levelup{\idxPDKwPrescribedDisplacement}
\label{sec:Peridigm:QRG:BoundaryConditions:PrescribedDisplacement}
\myindex[\idxPDKeywordName]{\idxPDKwPrescribedDisplacement}
\myindex[\idxPDKeywordName]{\idxPDKwBoundaryConditions!\idxPDKwPrescribedDisplacement|see{\idxPDKwPrescribedDisplacement}}

\leveldown{Description}
\label{sec:Peridigm:QRG:BoundaryConditions:PrescribedDisplacement:Description}

Definition of a prescribed displacement on a nodeset. The prescribed displacement might be dependent of the scalar coordinates of the nodeset members and/or the time. Thus, displacement and acceleration boundary conditions can be modeled with this card as well.

\levelstay{Sketch}

% General variables for following plots
\def\xn{0.0}
\def\xi{4.0}
\def\xf{11.0}
\def\yn{0.0}
\def\yi{10.0}
\def\yf{11.0}
% Charts
\begin{tabularx}{\linewidth}{YYY}
\begin{minipage}{\linewidth}
\begin{tikzpicture}
  % General variables
  \def\xlabel{$t$}
  \def\ylabel{$u$}
  % Axis
  \begin{axis}[uvachartstyle]
    % Function
    \addplot {\yi}; 
  \end{axis}
\end{tikzpicture}
\end{minipage}
&
\begin{minipage}{\linewidth}
\begin{tikzpicture}
  % General variables
  \def\xlabel{$t$}
  \def\ylabel{$v$}
  % Axis
  \begin{axis}[uvachartstyle]
    % Function
    \addplot {0}; 
  \end{axis}
\end{tikzpicture}
\end{minipage}
&
\begin{minipage}{\linewidth}
\begin{tikzpicture}
  % General variables
  \def\xlabel{$t$}
  \def\ylabel{$a$}
  % Axis
  \begin{axis}[uvachartstyle]
    % Function
    \addplot {0}; 
  \end{axis}
\end{tikzpicture}
\end{minipage}
\end{tabularx}

% \begin{figure}[htbp]
%   \centering
%   \tikzexternalenable
%   \tikzsetnextfilename{Test}
%   \begin{tikzpicture}
%     \node (image) at (0,0) {\includegraphics[width=0.5\linewidth]{example-image-a}};
%   \end{tikzpicture}
%   \tikzexternaldisable
%   \caption{A figure}
%   \label{fig:Figure}
% \end{figure}

\levelstay{Code}
\label{sec:Peridigm:QRG:BoundaryConditions:PrescribedDisplacement:Code}

\begin{itemize}[noitemsep]
  \item from \texttt{src/core}:
  \begin{itemize}[noitemsep]
    \item \texttt{Peridigm\_BoundaryAndInitialConditionManager.cpp}
    \item \texttt{Peridigm\_BoundaryAndInitialConditionManager.hpp}
    \item \texttt{Peridigm\_BoundaryCondition.cpp}
    \item \texttt{Peridigm\_BoundaryCondition.hpp}
    \item \texttt{Peridigm\_Enums.cpp}
    \item \texttt{Peridigm\_Enums.hpp}
  \end{itemize}
\end{itemize}

\levelstay{Input parameters}
\label{sec:Peridigm:QRG:BoundaryConditions:PrescribedDisplacement:InputParameters}

\paragraph{List}

\begin{tabularx}{\linewidth}{lcccX}
\toprule
Name		& Type		& Required	& Default	& Description		\\
\midrule
Type		& string	& \checkmark	& -		& ``\idxPDKwPrescribedDisplacement''	\\
Node Set        & string        & \checkmark\textsuperscript{\ref{enm:Peridigm:QRG:BoundaryConditions:PrescribedDisplacement:Remark:NodeSet}}    & -             & Application region name | ``Full Domain'' | ``All Sets''      \\
Coordinate	& string	& \checkmark	& -		& ``x'' | ``y'' | ``z'' 	\\
Value		& string	& \checkmark\textsuperscript{\ref{enm:Peridigm:QRG:BoundaryConditions:PrescribedDisplacement:Remark:Value}}	& -		& String with function for function parser	\\
\bottomrule
\end{tabularx}

\paragraph{Remarks}

\begin{enumerate}[noitemsep]
  \item \label{enm:Peridigm:QRG:BoundaryConditions:PrescribedDisplacement:Remark:NodeSet} Any string that is not ``Full Domain'' or ``All Sets'' will be considered a custom node set name in the model or mesh file, dependent of the discretization type.
  \item \label{enm:Peridigm:QRG:BoundaryConditions:PrescribedDisplacement:Remark:Value} The string in the variable \textit{Value} should start with \verb+value = +. If it does not, \toolname{} will automatically add it for the function parser to work.
\end{enumerate}

\levelstay{Exemplary input section}

\leveldown{XML format}

\begin{code}
<ParameterList name="Boundary Conditions">
  <ParameterList name="Prescr. Displacement Suppress Rigid Body Modes 1">
    <Parameter name="Type" type="string" value="Prescribed Displacement"/>
    <Parameter name="Node Set" type="string" value="nodelist_3"/>
    <Parameter name="Coordinate" type="string" value="y"/>
    <Parameter name="Value" type="string" value="0.0"/>
  </ParameterList>
  <ParameterList name="Prescr. Displacement Suppress Rigid Body Modes 2">
    <Parameter name="Type" type="string" value="Prescribed Displacement"/>
    <Parameter name="Node Set" type="string" value="nodelist_5"/>
    <Parameter name="Coordinate" type="string" value="y"/>
    <Parameter name="Value" type="string" value="0.0"/>
  </ParameterList>
  <ParameterList name="Prescr. Displacement Suppress Rigid Body Modes 3">
    <Parameter name="Type" type="string" value="Prescribed Displacement"/>
    <Parameter name="Node Set" type="string" value="nodelist_4"/>
    <Parameter name="Coordinate" type="string" value="z"/>
    <Parameter name="Value" type="string" value="0.0"/>
  </ParameterList>
  <ParameterList name="Prescr. Displacement Suppress Rigid Body Modes 4">
    <Parameter name="Type" type="string" value="Prescribed Displacement"/>
    <Parameter name="Node Set" type="string" value="nodelist_6"/>
    <Parameter name="Coordinate" type="string" value="z"/>
    <Parameter name="Value" type="string" value="0.0"/>
  </ParameterList>
</ParameterList>
\end{code}

\levelstay{Free format}

\begin{code}
Boundary Conditions
  Prescribed Displacement Fix Bottom Rigid Body Motion In X
    Type "Prescribed Displacement"
    Node Set "nodelist_3"
    Coordinate "x"
    Value "0.0"
  Prescribed Displacement Fix Bottom Rigid Body Motion In Z
    Type "Prescribed Displacement"
    Node Set "nodelist_4"
    Coordinate "z"
    Value "0.0"
  Prescribed Displacement Fix Top Rigid Body Motion In X
    Type "Prescribed Displacement"
    Node Set "nodelist_5"
    Coordinate "x"
    Value "0.0"
  Prescribed Displacement Fix Top Rigid Body Motion In Z
    Type "Prescribed Displacement"
    Node Set "nodelist_6"
    Coordinate "z"
    Value "0.0"
\end{code}

\levelstay{YAML format}

\begin{code}
Boundary Conditions:
  My Prescribed Displacement:
    Type: "Prescribed Displacement"
    Node Set: "My Node Set"
    Coordinate: "x"
    Value: "value = 5.0"
\end{code}

\levelup{List of examples}

\begin{itemize}[noitemsep]
%   \item From \texttt{Models/}:
%   \begin{itemize}[noitemsep]
%     \item \texttt{Models/Dogbone}
%   \end{itemize}
  \item From \texttt{examples/}:
  \begin{itemize}[noitemsep]
    \item \texttt{tensile\_test/tensile\_test.peridigm}
  \end{itemize}
  \item From \texttt{test/regression/}:
  \begin{itemize}[noitemsep]
    \item \texttt{Bar\_OneBlock\_OneMaterial\_QS/Bar.xml}
  \end{itemize}
%   \item From \texttt{test/verification/}:
%   \begin{itemize}[noitemsep]
%     \item \texttt{BondBreakingInitialVelocity\_TimeDependentCS/BondBreakingInitialVelocity.xml} 
%   \end{itemize}
\end{itemize}

