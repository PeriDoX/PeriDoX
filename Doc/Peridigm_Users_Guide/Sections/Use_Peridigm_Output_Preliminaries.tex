%%%%%%%%%%%%%%%%%%%%%%%%%%%%%%%%%%%%
% Header                           %
%%%%%%%%%%%%%%%%%%%%%%%%%%%%%%%%%%%%
% 
% Revisions: 2017-04-10 Martin R�del <martin.raedel@dlr.de>
%                       Initial draft
%               
% Contact:   Martin R�del,  martin.raedel@dlr.de
%            DLR Composite Structures and Adaptive Systems
%          
%                                 __/|__
%                                /_/_/_/  
%            www.dlr.de/fa/en      |/ DLR
% 
%%%%%%%%%%%%%%%%%%%%%%%%%%%%%%%%%%%%
% Content                          %
%%%%%%%%%%%%%%%%%%%%%%%%%%%%%%%%%%%%

\leveldown{Preliminaries}

Basically, there are five types of distinguished output data in \marktool{\toolname}:

\begin{itemize}[noitemsep]
  \item Global scalar fieldspecs \tab scalar fields defined over entire simulation
  \item Multiphysics fieldspecs
  \item Element scalar fieldspecs \tab scalar fields defined over elements
  \item Nodal Vector3D fieldspecs \tab vector fields defined at nodes
  \item Bond scalar fieldspecs
\end{itemize}

All identifiers for possible output values are defined in \verb+.\src\io\mesh_output\Field.h+.

% \newpage
%%%%%%%%%%%%%%%%%%%%%%%%%%%%%%%%%%%%
% Header                           %
%%%%%%%%%%%%%%%%%%%%%%%%%%%%%%%%%%%%
% 
% Revisions: 2017-04-10 Martin R�del <martin.raedel@dlr.de>
%                       Initial draft
%               
% Contact:   Martin R�del,  martin.raedel@dlr.de
%            DLR Composite Structures and Adaptive Systems
%          
%                                 __/|__
%                                /_/_/_/  
%            www.dlr.de/fa/en      |/ DLR
% 
%%%%%%%%%%%%%%%%%%%%%%%%%%%%%%%%%%%%
% Content                          %
%%%%%%%%%%%%%%%%%%%%%%%%%%%%%%%%%%%%

\leveldown{Output variables}
\label{sec:Peridigm:QRG:Output:Variables}

\paragraph{Global scalar}

\begin{multicols}{2}
\begin{itemize}[noitemsep]
  \item Global\_Angular\_Momentum
  \item Global\_Kinetic\_Energy
  \item Global\_Linear\_Momentum
  \item Global\_Strain\_Energy
  \item Global\_Strain\_Energy\_Density
\end{itemize}
\end{multicols}

\paragraph{Multiphysics}

Not considered here.

\begin{multicols}{2}
\begin{itemize}[noitemsep]
  \item Fluid\_Pressure\_U
  \item Fluid\_Pressure\_V
  \item Fluid\_Pressure\_Y
  \item Flux
  \item Flux\_Density
\end{itemize}
\end{multicols}

\paragraph{Element scalar}

\begin{multicols}{2}
\begin{itemize}[noitemsep]
  \item BC\_MASK
  \item Block\_Id
  \item Critical\_Stretch
  \item Critical\_Time\_Step
  \item Damage
  \item Density
  \item Dilatation
  \item Element\_Id
  \item Interface\_Proximity
  \item Kinetic\_Energy
  \item Lambda
  \item Neighborhood\_Volume
  \item Norm\_td
  \item Num\_Neighbors
  \item Number\_Of\_Neighbors
  \item Proc\_Num
  \item Radius
  \item Strain\_Energy
  \item Strain\_Energy\_Density
  \item Volume
  \item Weighted\_Volume
\end{itemize}
\end{multicols}

\paragraph{Nodal Vector3D}

\begin{multicols}{2}
\begin{itemize}[noitemsep]
  \item Acceleration
  \item Angular\_Momentum
  \item Contact\_Force
  \item Contact\_Force\_Density
  \item Coordinates
  \item Displacement
  \item Force
  \item Force\_Density
  \item Linear\_Momentum
  \item Model\_Coordinates
  \item Residual
  \item Tangent\_Reference\_Coordinates
  \item Velocity
\end{itemize}
\end{multicols}

\paragraph{Bond scalar}

\begin{multicols}{2}
\begin{itemize}[noitemsep]
  \item Bond\_Damage
  \item Deviatoric\_Plastic\_Extension
  \item Deviatoric\_Back\_Extension
  \item Partial\_Volume
\end{itemize}
\end{multicols}

The different entities are written to different output files:

\begin{itemize}[noitemsep]
  \item .e:	\tab Element scalar, Nodal Vector3D, Bond scalar
  \item .h:	\tab Global scalar
\end{itemize}

It is not differentiated between cellular data and point data. Also it is not important if the calculation supports the output. If the simulation has no results for a specific output the data will be set to zero. However, the size of the output file itself is influenced. All variables can be found with the defined names in \marktool{Paraview}. 

\paragraph{Additional output values from classes}

The following classes have their own output values. The specific output variables are stated in the class description in this document.

\begin{itemize}[noitemsep]
  \item Materials
  \item Compute classes in \texttt{src/compute/}
\end{itemize}

\subparagraph{Compute class output values}\mbox{}\\

\begin{itemize}[noitemsep]
  \item Stored\_Elastic\_Energy			\tab$\rightarrow$ Peridigm\_Compute\_Stored\_Elastic\_Energy.cpp
  \item Stored\_Elastic\_Energy\_Density	\tab$\rightarrow$ Peridigm\_Compute\_Stored\_Elastic\_Energy\_Density.cpp
\end{itemize}
% \begin{tabularx}{\linewidth}{@{\dlrhook{ }}l@{ $\rightarrow$ }X}
% Stored\_Elastic\_Energy			& Peridigm\_Compute\_Stored\_Elastic\_Energy.cpp	\\
% Stored\_Elastic\_Energy\_Density	& Peridigm\_Compute\_Stored\_Elastic\_Energy\_Density.cpp
% \end{tabularx}



% \begin{itemize}[noitemsep]
%   \item Correspondence Materials:
%   \begin{itemize}[noitemsep]
%     \item Cauchy\_Stress
%     \item Hourglass\_Force\_Density
%     \item Left\_Stretch\_Tensor
%     \item Partial\_Stress
%     \item Rotation\_Tensor
%     \item Shape\_Tensor\_Inverse
%     \item Unrotated\_Cauchy\_Stress
%     \item Unrotated\_Rate\_Of\_Deformation
%   \end{itemize}
% \end{itemize}
\newpage
%%%%%%%%%%%%%%%%%%%%%%%%%%%%%%%%%%%%
% Header                           %
%%%%%%%%%%%%%%%%%%%%%%%%%%%%%%%%%%%%
% 
% Revisions: 2017-04-10 Martin R�del <martin.raedel@dlr.de>
%                       Initial draft
%               
% Contact:   Martin R�del,  martin.raedel@dlr.de
%            DLR Composite Structures and Adaptive Systems
%          
%                                 __/|__
%                                /_/_/_/  
%            www.dlr.de/fa/en      |/ DLR
%
%%%%%%%%%%%%%%%%%%%%%%%%%%%%%%%%%%%%
% Content                          %
%%%%%%%%%%%%%%%%%%%%%%%%%%%%%%%%%%%%

\levelstay{Output data acquisition}

\paragraph{Output data acquisition basics}

The output data acquisition block starts with information about how and where to store the entities. The following entries are not mandatory:

\begin{itemize}[noitemsep]
  \item Output Format	\tab: default: binary
  \item Parallel Write	\tab: default: true
\end{itemize}

\paragraph{Output data acquisition without global scalar entities}

In case no global scalar value is required, it is sufficient to create one output in the input deck. OUTPUT PARAMETER LIST can contain any of the aforementioned entities for the .e-results file. By default the output is set to false for all entities. The result file will be \verb+FILENAME.e+.

\subparagraph{Free-format}\mbox{}\\

\begin{code}
Output
  Output File Type ExodusII
  Output Format "BINARY"
  Output Filename "FILENAME"
  Output Frequency "1"
  Parallel Write "true"
  Output Variables
    OUTPUT PARAMETER LIST, e.g.
    Displacement "true"
    Velocity "true"
\end{code}

\subparagraph{XML-format}\mbox{}\\

\begin{code}
<ParameterList name="Output">
  <Parameter name="Output File Type" type="string" value="ExodusII"/>
  <Parameter name="Output Format" type="string" value="BINARY"/>
  <Parameter name="Output Filename" type="string" value="Filename"/>
  <Parameter name="Output Frequency" type="int" value="1"/>
  <Parameter name="Parallel Write" type="bool" value="true"/>
  <ParameterList name="Output Variables">
    OUTPUT PARAMETER LIST, e.g.
    <Parameter name="Displacement" type="bool" value="true"/>
    <Parameter name="Velocity" type="bool" value="true"/>
  </ParameterList>
</ParameterList>
\end{code}

\paragraph{Output data acquisition including global scalar entities}

Two different sets of outputs have to created, one for the .e-  and one for the .h-entities. The result files will be \verb+FILENAME.e+ for \verb+Output_1+ and \verb+FILENAME.h+ for \verb+Output_2+. The output frequencies can vary between the two outputs. The non-mandatory items are left out in this example.

\subparagraph{Free-format}\mbox{}\\

\begin{code}
Output1
  Output File Type ExodusII
  Output Filename "FILENAME"
  Output Frequency "10"
  Output Variables
    .e OUTPUT PARAMETER LIST, e.g.
    Displacement "true"
    Velocity "true"

Output2
  Output File Type ExodusII
  Output Filename "FILENAME"
  Output Frequency "5"
  Output Variables
    .h OUTPUT PARAMETER LIST, e.g.
    Global_Kinetic_Energy "true"
    Global_Linear_Momentum "true"
\end{code}

\subparagraph{XML-format}\mbox{}\\

equivalent
\newpage
%%%%%%%%%%%%%%%%%%%%%%%%%%%%%%%%%%%%
% Header                           %
%%%%%%%%%%%%%%%%%%%%%%%%%%%%%%%%%%%%
% 
% Revisions: 2017-04-10 Martin R�del <martin.raedel@dlr.de>
%                       Initial draft
%               
% Contact:   Martin R�del,  martin.raedel@dlr.de
%            DLR Composite Structures and Adaptive Systems
%          
%                                 __/|__
%                                /_/_/_/  
%            www.dlr.de/fa/en      |/ DLR
% 
%%%%%%%%%%%%%%%%%%%%%%%%%%%%%%%%%%%%
% Content                          %
%%%%%%%%%%%%%%%%%%%%%%%%%%%%%%%%%%%%

\levelstay{Output sequence}

A sequencially different output, e.g. for different solver steps, can be obtained via the specification of \textit{Initial Output Step} and \textit{Final Output Step} for each \textit{Output} block. For the first output block \textit{Initial Output Step} and for the last output block \textit{Final Output Step} do not have to be specified if these coincide with 0.0 and the end of the simulation.

\paragraph{Exemplary input section}

\subparagraph{XML-format}\mbox{}\\

-

\subparagraph{Free format}\mbox{}\\

\begin{code}
Output1
  Output File Type "ExodusII"
  Output Format "BINARY"
  Output Filename "model_implicit"
  Output Frequency 1
  Final Output Step 10
  Output Variables
    Block_Id "true"
    ...
    Displacement "true"

Output2
  Output File Type "ExodusII"
  Output Filename "model_explicit"
  Output Frequency 50
  Initial Output Step 11
  Output Variables
    Block_Id "true"
    ...
    Displacement "true"
\end{code}

\subparagraph{YAML format}\mbox{}\\

-

\paragraph{List of examples}

\begin{itemize}[noitemsep]
%   \item From \texttt{Models/}:
%   \begin{itemize}[noitemsep]
%     \item \texttt{Models/Dogbone}
%   \end{itemize}
  \item From \texttt{examples/}:
  \begin{itemize}[noitemsep]
    \item \texttt{twist\_and\_pull/twist\_and\_pull.peridigm}
  \end{itemize}
%   \item From \texttt{test/regression/}:
%   \begin{itemize}[noitemsep]
%     \item \texttt{Bar\_OneBlock\_OneMaterial\_QS/Bar.xml}
%   \end{itemize}
%   \item From \texttt{test/verification/}:
%   \begin{itemize}[noitemsep]
%     \item \texttt{BondBreakingInitialVelocity\_TimeDependentCS/BondBreakingInitialVelocity.xml} 
%   \end{itemize}
\end{itemize}
\newpage
%%%%%%%%%%%%%%%%%%%%%%%%%%%%%%%%%%%%
% Header                           %
%%%%%%%%%%%%%%%%%%%%%%%%%%%%%%%%%%%%
% 
% Revisions: 2017-04-10 Martin R�del <martin.raedel@dlr.de>
%                       Initial draft
%               
% Contact:   Martin R�del,  martin.raedel@dlr.de
%            DLR Composite Structures and Adaptive Systems
%          
%                                 __/|__
%                                /_/_/_/  
%            www.dlr.de/fa/en      |/ DLR
% 
%%%%%%%%%%%%%%%%%%%%%%%%%%%%%%%%%%%%
% Content                          %
%%%%%%%%%%%%%%%%%%%%%%%%%%%%%%%%%%%%

\levelstay{User-defined output entities}

So called \verb+Compute Class Parameters+ can be defined for the acquisition of result data for nodesets, blocks or the nearest point next to a specific location.

The respective keywords are explained in sections \ref{sec:Peridigm:QRG:ComputeClassParameters:Nodeset},  \ref{sec:Peridigm:QRG:ComputeClassParameters:Block} and \ref{sec:Peridigm:QRG:ComputeClassParameters:NearestPoint} 