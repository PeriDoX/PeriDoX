%%%%%%%%%%%%%%%%%%%%%%%%%%%%%%%%%%%%
% Header                           %
%%%%%%%%%%%%%%%%%%%%%%%%%%%%%%%%%%%%
% 
% Preference file for everything considering glossaries
% 
% Revisions: 2017-04-10 Martin Raedel <martin.raedel@dlr.de>
%                       Initial draft
%               
% Contact:   Martin Raedel,  martin.raedel@dlr.de
%            DLR Composite Structures and Adaptive Systems
%          
%                                 __/|__
%                                /_/_/_/  
%            www.dlr.de/fa/en      |/ DLR
% 
%%%%%%%%%%%%%%%%%%%%%%%%%%%%%%%%%%%%
% Content                          %
%%%%%%%%%%%%%%%%%%%%%%%%%%%%%%%%%%%%

%%%%%%%%%%%%%%%%%%%%%%%%%%%%%%%%%%%%
% Package                          %
%%%%%%%%%%%%%%%%%%%%%%%%%%%%%%%%%%%%

\usepackage[
%   nonumberlist,   %keine Seitenzahlen anzeigen
  acronym,        %ein Abk�rzungsverzeichnis erstellen
  nomain,         %don�t use the main glossary
  toc,            %Eintr�ge im Inhaltsverzeichnis
  %translate=babel,
  %section=\glossarytoclevel   % kapitelweise Verzeichniserstellung
%   section,        %im Inhaltsverzeichnis auf section-Ebene erscheinen
%   savewrites,     % minimise the number of write registers used
]{glossaries}

% \usepackage{scrwfile} % No \newwrite and therefore limitations on registers

%%%%%%%%%%%%%%%%%%%%%%%%%%%%%%%%%%%%
% Redefine package options         %
%%%%%%%%%%%%%%%%%%%%%%%%%%%%%%%%%%%%

%Den Punkt am Ende jeder Beschreibung deaktivieren
\renewcommand*{\glspostdescription}{}
% \renewcommand*{\glspostdescription}{\dotfill}

%%%%%%%%%%%%%%%%%%%%%%%%%%%%%%%%%%%%
% Own styles                       %
%%%%%%%%%%%%%%%%%%%%%%%%%%%%%%%%%%%%

% -----------------
% Acronym-styles
% -----------------

\newglossarystyle{myacronymstyle}{%
  \renewenvironment{theglossary}%
    {\begin{longtabu} to \linewidth {lX}}%
    {\end{longtabu}}%
  % Header line
  \renewcommand*{\glossaryheader}{%
%     % Requires booktabs
%     \toprule%
%     Abbreviation & Description%
%     \tabularnewline%
%     \midrule%
%     \endhead%
%     \bottomrule%
%     \endfoot%
  }%
  % indicate what to do at the start of each logical group
  %\renewcommand*{\glsgroupheading}[1]{}%
  %\renewcommand*{\glsgroupskip}{}% What to do between groups
  \renewcommand*{\glsgroupskip}{\tabularnewline}% What to do between groups
  \renewcommand*{\glossaryentryfield}[5]{%
    \glsentryitem{##1}\glstarget{##1}{##2} 
     %\glstarget{##2}{##2}% Name
      & ##3\glspostdescription ##5% Description
      \\% end of row
  }
}


    \newglossarystyle{myglostyle}{%
      \renewcommand*{\glsclearpage}{}%
      \renewenvironment{theglossary}%
        {\begin{longtabu} to \linewidth {cX}}%
        {\end{longtabu}}%
      % Header line
      \renewcommand*{\glossaryheader}{%
        \textbf{Symbol} & \textbf{Description}%
        \tabularnewline%
        \tabularnewline%
        \endhead%
        \endfoot%
      }%
      \renewcommand*{\glsgroupskip}{\tabularnewline}
      \renewcommand*{\glossentry}[1]{%
        \glsentryitem{##1}
        \glstarget{##1}{\glossentrysymbol{##1}} &
        \glossentrydesc{##1}        %& % Description
        \tabularnewline%
      }%
    }

% -----------------
% Coordinate-system style
% -----------------

\newglossarystyle{mycoordinatesystemstyle}{%
  %\renewcommand{\glossarysection}[2][]{}% no title
  \renewcommand*{\glsclearpage}{}% avoid page break before glossary
  \renewenvironment{theglossary}%
    % \extrarowsep=1mm
    {\begin{longtabu} to \linewidth {cX}}%
    {\end{longtabu}}%
  % Header line
  \renewcommand*{\glossaryheader}{%
    % Requires booktabs
    %\toprule%
    \textbf{Symbol} & \textbf{Description}%
    \tabularnewline%
    \tabularnewline%
    %\midrule%
    \endhead%
    %\bottomrule%
    \endfoot%
  }%
  % indicate what to do at the start of each logical group
  %\renewcommand*{\glsgroupheading}[1]{}%
  %\renewcommand*{\glsgroupskip}{}% What to do between groups
  \renewcommand*{\glsgroupskip}{\tabularnewline}% What to do between groups
%   \renewcommand*{\glossaryentryfield}[5]{%
%     \glsentryitem{##1}\glstarget{##1}{##2} 
%      %\glstarget{##2}{##2}% Name
%       & ##3\glspostdescription ##5% Description
%       \\% end of row
%   }
  \renewcommand*{\glossentry}[1]{%
    \glsentryitem{##1}% Entry number if required
    \glstarget{##1}{\glossentrysymbol{##1}} &
    %\glossentrysymbol{##1}	& % Symbol
    %\glossentryname{##1}	& % Name
    \glossentrydesc{##1}	%& % Description
    %\glsentryuseri{##1}%	  % Unit in User1-Variable
    \tabularnewline%
  }%
}

% -----------------
% Symbols-styles
% -----------------

\newglossarystyle{mysymbolstyle}{%
  %\renewcommand{\glossarysection}[2][]{}% no title
  \renewcommand*{\glsclearpage}{}% avoid page break before glossary
  \renewenvironment{theglossary}%
    % \extrarowsep=1mm
    {\begin{longtabu} to \linewidth {clX}}%c}}%
    {\end{longtabu}}%
%     {\begin{longtable}{@{}p{0.1\linewidth}p{0.8\linewidth}p{0.1\linewidth}@{}}}%
%     {\end{longtable}}%
  % Header line
  \renewcommand*{\glossaryheader}{%
    % Requires booktabs
    %\toprule%
    \textbf{Symbol} & \textbf{Name} & \textbf{Description}% & \textbf{Unit}%
    \tabularnewline%
    \tabularnewline%
    %\midrule%
    \endhead%
    %\bottomrule%
    \endfoot%
  }%
  % indicate what to do at the start of each logical group
  %\renewcommand*{\glsgroupheading}[1]{}%
  %\renewcommand*{\glsgroupskip}{}% What to do between groups
  \renewcommand*{\glsgroupskip}{\tabularnewline}% What to do between groups
%   \renewcommand*{\glossaryentryfield}[5]{%
%     \glsentryitem{##1}\glstarget{##1}{##2} 
%      %\glstarget{##2}{##2}% Name
%       & ##3\glspostdescription ##5% Description
%       \\% end of row
%   }
  \renewcommand*{\glossentry}[1]{%
    \glsentryitem{##1}% Entry number if required
    \glstarget{##1}{\glossentrysymbol{##1}} &
    %\glossentrysymbol{##1}	& % Symbol
    \glossentryname{##1}	& % Name
    \glossentrydesc{##1}	%& % Description
    %\glsentryuseri{##1}%	  % Unit in User1-Variable
    \tabularnewline%
  }%
}

% -----------------
% Symbols-styles for papers
% -----------------

\newglossarystyle{myonecolpapersymbolstyle}{%
  %\renewcommand{\glossarysection}[2][]{}% no title
  \renewcommand*{\glsclearpage}{}% avoid page break before glossary
  \renewenvironment{theglossary}%
    {\begin{longtabu} to \linewidth {clXcl}}%c}}%
    {\end{longtabu}}%
  % Header line
  \renewcommand*{\glossaryheader}{}%
  %\renewcommand*{\glsgroupheading}[1]{}% indicate what to do at the start of each logical group
  \renewcommand*{\glsgroupskip}{}% What to do between groups -> no skip
  \renewcommand*{\glossentry}[1]{% How the entry looks like
    \glsentryitem{##1}% Entry number if required
    \glstarget{##1}{\glossentrysymbol{##1}} & % Symbol
    \glossentryname{##1}        %& % Name
    \tabularnewline%
  }%
}

% https://tex.stackexchange.com/a/216434/44634
% needs: \usepackage{multicol}
\newglossarystyle{mytwocolpapersymbolstyle}{%
  %\renewcommand{\glossarysection}[2][]{}% no title
  \renewenvironment{theglossary}%
    {\begin{multicols}{2}\raggedright}
    {\end{multicols}}
  % Header line
  \renewcommand*{\glossaryheader}{}%
  \renewcommand*{\glsgroupheading}[1]{}% indicate what to do at the start of each logical group
  \renewcommand*{\glsgroupskip}{}% What to do between groups -> no skip
  \renewcommand*{\glsclearpage}{}% avoid page break before glossary 
  % set how each entry should appear:
  \renewcommand*{\glossentry}[2]{
    \noindent\makebox[2.5em][c]{\glstarget{##1}{\glossentrysymbol{##1}}}% Symbol
    \glossentryname{##1}% Name
    \newline
  }
%   \renewcommand*{\subglossentry}[3]{%
%     \glossentry{##2}{##3}
%   }
}

% -----------------
% Exponent-styles
% -----------------

\newglossarystyle{myexponentstyle}{%
  %\renewcommand{\glossarysection}[2][]{}% no title
  \renewcommand*{\glsclearpage}{}% avoid page break before glossary
  \renewenvironment{theglossary}%
    % \extrarowsep=1mm
    {%
      \begingroup
      \renewcommand{\arraystretch}{1.4}
      \begin{longtabu} to \linewidth {@{\ \ }r@{}lX}
    }{%
      \end{longtabu}
      \endgroup
    }%
%     {\begin{longtable}{@{}p{0.1\linewidth}p{0.8\linewidth}p{0.1\linewidth}@{}}}%
%     {\end{longtable}}%
  % Header line
  \renewcommand*{\glossaryheader}{%
    % Requires booktabs
    %\toprule%
    %\textbf{Symbol} & \textbf{Name} & \textbf{Description}%
    \multicolumn{2}{@{}c@{}}{\textbf{Symbol}} & \textbf{Description}%
    \tabularnewline%
    \tabularnewline%
    %\midrule%
    \endhead%
    %\bottomrule%
    \endfoot%
  }%
  % indicate what to do at the start of each logical group
  %\renewcommand*{\glsgroupheading}[1]{}%
  %\renewcommand*{\glsgroupskip}{}% What to do between groups
  \renewcommand*{\glsgroupskip}{\tabularnewline}% What to do between groups
%   \renewcommand*{\glossaryentryfield}[5]{%
%     \glsentryitem{##1}\glstarget{##1}{##2} 
%      %\glstarget{##2}{##2}% Name
%       & ##3\glspostdescription ##5% Description
%       \\% end of row
%   }
  \renewcommand*{\glossentry}[1]{%
    \glsentryitem{##1}% Entry number if required
    \protect\ensuremath{\protect\left(\protect\phantom{a}\protect\right)} &
    \glstarget{##1}{\protect\ensuremath{\protect\vphantom{a}^{\glossentrysymbol{##1}}}} &
    %\glossentrysymbol{##1}     & % Symbol
    %\glossentryname{##1}       & % Name
    \glossentrydesc{##1}        %& % Description
    %\glsentryuseri{##1}%         % Unit in User1-Variable
    \tabularnewline%
  }%
}

% -----------------
% Index-styles
% -----------------

\newglossarystyle{myindexstyle}{%
  %\renewcommand{\glossarysection}[2][]{}% no title
  \renewcommand*{\glsclearpage}{}% avoid page break before glossary
  \renewenvironment{theglossary}%
    % \extrarowsep=1mm
    {%
      \begingroup
      \renewcommand{\arraystretch}{1.4}
      \begin{longtabu} to \linewidth {@{\ \ }r@{}lX}
    }{%
      \end{longtabu}
      \endgroup
    }%
%     {\begin{longtable}{@{}p{0.1\linewidth}p{0.8\linewidth}p{0.1\linewidth}@{}}}%
%     {\end{longtable}}%
  % Header line
  \renewcommand*{\glossaryheader}{%
    % Requires booktabs
    %\toprule%
    %\textbf{Symbol} & \textbf{Name} & \textbf{Description}%
    \multicolumn{2}{@{}c@{}}{\textbf{Symbol}} & \textbf{Description}%
    \tabularnewline%
    \tabularnewline%
    %\midrule%
    \endhead%
    %\bottomrule%
    \endfoot%
  }%
  % indicate what to do at the start of each logical group
  %\renewcommand*{\glsgroupheading}[1]{}%
  %\renewcommand*{\glsgroupskip}{}% What to do between groups
  \renewcommand*{\glsgroupskip}{\tabularnewline}% What to do between groups
%   \renewcommand*{\glossaryentryfield}[5]{%
%     \glsentryitem{##1}\glstarget{##1}{##2} 
%      %\glstarget{##2}{##2}% Name
%       & ##3\glspostdescription ##5% Description
%       \\% end of row
%   }
  \renewcommand*{\glossentry}[1]{%
    \glsentryitem{##1}% Entry number if required
    \protect\ensuremath{\protect\left(\protect\phantom{a}\protect\right)} &
    %\glstarget{##1}{\glossentrysymbol{##1}} &
    \glstarget{##1}{\protect\ensuremath{\protect\vphantom{a}_{\glossentrysymbol{##1}}}} &
    %\glossentrysymbol{##1}	& % Symbol
    %\glossentryname{##1}	& % Name
    \glossentrydesc{##1}	%& % Description
    %\glsentryuseri{##1}%	  % Unit in User1-Variable
    \tabularnewline%
  }%
}

% -----------------
% Operator style
% -----------------

\newglossarystyle{myoperatorstyle}{%
  %\renewcommand{\glossarysection}[2][]{}% no title
  \renewcommand*{\glsclearpage}{}% avoid page break before glossary
  \renewenvironment{theglossary}%
    % \extrarowsep=1mm
    {%
      \begingroup%
      \renewcommand{\arraystretch}{1.4}%
      %\begin{longtabu} to \linewidth {cX}
      \begin{longtabu} to \linewidth {@{\ \;}r@{}c@{}lX}
    }%
    {%
      \end{longtabu}
      \endgroup
    }%
  % Header line
  \renewcommand*{\glossaryheader}{%
    % Requires booktabs
    %\toprule%
    %\textbf{Symbol} & \textbf{Description}%
    \multicolumn{3}{@{}c@{}}{\textbf{Symbol}} & \textbf{Description}%
    \tabularnewline%
    \tabularnewline%
    %\midrule%
    \endhead%
    %\bottomrule%
    \endfoot%
  }%
  % indicate what to do at the start of each logical group
  %\renewcommand*{\glsgroupheading}[1]{}%
  %\renewcommand*{\glsgroupskip}{}% What to do between groups
  \renewcommand*{\glsgroupskip}{\tabularnewline}% What to do between groups
%   \renewcommand*{\glossaryentryfield}[5]{%
%     \glsentryitem{##1}\glstarget{##1}{##2} 
%      %\glstarget{##2}{##2}% Name
%       & ##3\glspostdescription ##5% Description
%       \\% end of row
%   }
  \renewcommand*{\glossentry}[1]{%
    \glsentryitem{##1}% Entry number if required
    %\glstarget{##1}{\glossentrysymbol{##1}} &
    %\glstarget{##1}{\glossentrysymbol{##1}}&
    \glsentryuseri{##1} &
    \glsentryuserii{##1} &
    \glsentryuseriii{##1} &
    %\glossentrysymbol{##1}	& % Symbol
    %\glossentryname{##1}	& % Name
    \glossentrydesc{##1}	%& % Description
    %\glsentryuseri{##1}%	  % Unit in User1-Variable
    \tabularnewline%
  }%
}