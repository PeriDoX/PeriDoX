%%%%%%%%%%%%%%%%%%%%%%%%%%%%%%%%%%%%
% Header                           %
%%%%%%%%%%%%%%%%%%%%%%%%%%%%%%%%%%%%
% 
% Revisions: 2017-04-10 Martin Rädel <martin.raedel@dlr.de>
%                       Initial draft
%               
% Contact:   Martin Rädel,  martin.raedel@dlr.de
%            DLR Composite Structures and Adaptive Systems
%          
%                                 __/|__
%                                /_/_/_/  
%            www.dlr.de/fa/en      |/ DLR
% 
%%%%%%%%%%%%%%%%%%%%%%%%%%%%%%%%%%%%
% Content                          %
%%%%%%%%%%%%%%%%%%%%%%%%%%%%%%%%%%%%

\levelstay{\marktool{MPICH}}

\marktool{MPICH} is a high performance and widely portable implementation of the Message Passing Interface (MPI) standard.

\paragraph{Use the operating system distribution}

ToDo

\paragraph{Use 1-click install}

Go to

\href{https://software.opensuse.org/package/mpich}{https://software.opensuse.org/package/mpich}

and choose 1-Click-Install or download the rpm-file from the source.

\paragraph{Installation with \marktool{\opensusename}-repository}

ToDo

\paragraph{Installation from source}

Try

\begingroup
\lstset{breaklines=true}
\begin{code}
cd $DOWNLOAD_DIR
zypper si -d mpich2		%\lstcomment{\# install the build deps}%
				%\lstcomment{\# for the previous version}%
wget http://www.mpich.org/static/downloads/3.2/mpich-3.2.tar.gz	
tar xvfz mpich-3.2.tar.gz 		%\lstcomment{\# unzip}%
cd mpich-3.2			%\lstcomment{\# go into directory}%
./configure > configure_mpich.log 2>&1
make > make_mpich.log 2>&1 %\lstcomment{\# build}%
make altinstall > make_install_mpich.log 2>&1 %\lstcomment{\# install}%
\end{code}
\endgroup 