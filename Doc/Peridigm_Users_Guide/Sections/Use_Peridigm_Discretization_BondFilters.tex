%%%%%%%%%%%%%%%%%%%%%%%%%%%%%%%%%%%%
% Header                           %
%%%%%%%%%%%%%%%%%%%%%%%%%%%%%%%%%%%%
% 
% Revisions: 2017-04-10 Martin R�del <martin.raedel@dlr.de>
%                       Initial draft
%               
% Contact:   Martin R�del,  martin.raedel@dlr.de
%            DLR Composite Structures and Adaptive Systems
%          
%                                 __/|__
%                                /_/_/_/  
%            www.dlr.de/fa/en      |/ DLR
% 
%%%%%%%%%%%%%%%%%%%%%%%%%%%%%%%%%%%%
% Content                          %
%%%%%%%%%%%%%%%%%%%%%%%%%%%%%%%%%%%%

\levelstay{\idxPDKwBondFilters}
\label{sec:Peridigm:QRG:Discretization:Tools:BondFilters}
\myindex[\idxPDKeywordName]{\idxPDKwBondFilters}

\leveldown{Description}

A method to omit bonds in the discretization.

\levelstay{Code}

Dependent on the type of discretization the following classes are used:

\begin{itemize}[noitemsep]
  \item from \texttt{src/io/discretization/}:
  \begin{itemize}[noitemsep]
    \item \texttt{Peridigm\_AlbanyDiscretization.cpp}
    \item \texttt{Peridigm\_AlbanyDiscretization.hpp}
    \item \texttt{Peridigm\_ExodusDiscretization.cpp}
    \item \texttt{Peridigm\_ExodusDiscretization.hpp}
    \item \texttt{Peridigm\_PdQuickGridDiscretization.hpp}
    \item \texttt{Peridigm\_PdQuickGridDiscretization.cpp}
    \item \texttt{Peridigm\_TextFileDiscretization.hpp}
    \item \texttt{Peridigm\_TextFileDiscretization.cpp}
  \end{itemize}
\end{itemize}

All extend the super-class

\begin{itemize}[noitemsep]
  \item \verb+/src/io/discretization/Peridigm_Discretization.cpp+
  \item \verb+/src/io/discretization/Peridigm_Discretization.hpp+
\end{itemize}

Each class uses the method from the super-class :

\verb+PeridigmNS::Discretization::createBondFilters+

\levelstay{Input parameters}

\leveldown{List}

\begin{tabularx}{\linewidth}{lcccX}
\toprule
Name                    & Type   & Required   & Default & Description           \\
\midrule
Type                    & string & \checkmark & - & ``\idxPDKwRectangularPlane'' \\
Normal\_X               & double & \checkmark & - & $x$-component of the cutting plane normal vector\\
Normal\_Y               & double & \checkmark & - & $y$-component of the cutting plane normal vector\\
Normal\_Z               & double & \checkmark & - & $z$-component of the cutting plane normal vector\\
Lower\_Left\_Corner\_X  & double & \checkmark & - & \\
Lower\_Left\_Corner\_Y  & double & \checkmark & - & \\
Lower\_Left\_Corner\_Z  & double & \checkmark & - & \\
Bottom\_Unit\_Vector\_X & double & \checkmark & - & \\
Bottom\_Unit\_Vector\_Y & double & \checkmark & - & \\
Bottom\_Unit\_Vector\_Z & double & \checkmark & - & \\
Bottom\_Length          & double & \checkmark & - & \\
Side\_Length            & double & \checkmark & - & \\
\bottomrule
\end{tabularx}

\levelstay{Remarks}

\begin{enumerate}[noitemsep]
  \item \idxPDKwBondFilters{} cause the bonds to not be created in the first place
  \item For type ``\idxPDKwRectangularPlane'':
  \begin{itemize}[noitemsep]
    \item Define a rectangular cut plane where all bonds are released prior to the simulation.
  \end{itemize}
  \item It is suggested to define the output variable Number\_Of\_Neighbors in order to confirm that the bond filter is in the right place. It always seems to take me multiple tries to get it right.
\end{enumerate}

\levelup{Exemplary input section}

\leveldown{XML-format}

from \verb+/test/regression/PrecrackedPlate/PrecrackedPlate.xml+:

\begingroup
\lstset{breaklines=true}
\begin{code}
<ParameterList name="Discretization">
  <Parameter name="Type" type="string" value="Exodus" />
  <Parameter name="Input Mesh File" type="string" value="Plate.g"/>
  <ParameterList name="Bond Filters">
    <ParameterList name="My First Bond Filter">
      <Parameter name="Type" type="string" value="Rectangular_Plane"/>
      <Parameter name="Normal_X" type="double" value="0.0"/>
      <Parameter name="Normal_Y" type="double" value="1.0"/>
      <Parameter name="Normal_Z" type="double" value="0.0"/>
      <Parameter name="Lower_Left_Corner_X" type="double" value="-10.0"/>
      <Parameter name="Lower_Left_Corner_Y" type="double" value="0.0"/>
      <Parameter name="Lower_Left_Corner_Z" type="double" value="-10.0"/>
      <Parameter name="Bottom_Unit_Vector_X" type="double" value="1.0"/>
      <Parameter name="Bottom_Unit_Vector_Y" type="double" value="0.0"/>
      <Parameter name="Bottom_Unit_Vector_Z" type="double" value="0.0"/>
      <Parameter name="Bottom_Length" type="double" value="10"/>
      <Parameter name="Side_Length" type="double" value="20"/>
    </ParameterList>
  </ParameterList>
</ParameterList>
\end{code}
\endgroup

\levelstay{Free format}

-

\levelstay{YAML format}

-

\levelup{List of examples}

\begin{itemize}[noitemsep]
%   \item From \texttt{examples/}:
%   \begin{itemize}[noitemsep]
%     \item \texttt{examples/tensile_test/tensile_test.peridigm}
%   \end{itemize}
  \item From \texttt{test/regression/}:
  \begin{itemize}[noitemsep]
    \item \texttt{PrecrackedPlate/PrecrackedPlate.xml}
    \item \texttt{PrecrackedPlateTwoCracks/PrecrackedPlateTwoCracks.xml}
  \end{itemize}
%   \item From \texttt{test/verification/}:
%   \begin{itemize}[noitemsep]
%     \item \texttt{Contact_Friction/Contact_Friction.xml}
%   \end{itemize}
\end{itemize}
% \begin{itemize}[noitemsep]
%   \item \texttt{/test/regression/PrecrackedPlate/PrecrackedPlate.xml}
%   \item \texttt{/test/regression/PrecrackedPlateTwoCracks/PrecrackedPlateTwoCracks.xml}
% \end{itemize}