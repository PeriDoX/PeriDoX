\levelstay{Tool-compliant legends in pgfplots}

To create a tool-compliant legend two things are required: the colormap and the colorbar style.

\leveldown{\texorpdfstring{\protect\marktool{\paraviewname}}{\paraviewname}, \texorpdfstring{\protect\marktool{\abaqusname}}{\abaqusname}, $\ldots$ colormaps in pgfplots}

To use a legend compliant with any free or commercial tool, a contiuous colormap, so the color gradient, has to be defined first. E.g. the definition of the (original?) \marktool{\paraviewname} colormap is specified here:

\href{http://www.paraview.org/Wiki/images/b/be/All_idl_cmaps.xml}{http://www.paraview.org/Wiki/images/b/be/All\_idl\_cmaps.xml}

It should be possible to convert the exact definition to a \verb+pgfplots+ colormap. A rough approximation can be defined in the preamble with the following code-snippet. The colormaps for \marktool{\paraviewname} and \marktool{\abaqusname} seem to be identical, at least according to color analysis in Gimp.

\begin{code}
\pgfplotsset{
  colormap={abaqusblueredcolormap}{
    rgb255( 0cm)=(  0,  0,255);
    rgb255( 1cm)=(  0, 93,255);
    rgb255( 2cm)=(  0,185,255);
    rgb255( 3cm)=(  0,255,232);
    rgb255( 4cm)=(  0,255,139);
    rgb255( 5cm)=(  0,255,139);
    rgb255( 6cm)=(  0,255, 46);
    rgb255( 7cm)=( 46,255,  0);
    rgb255( 8cm)=(139,255,  0);
    rgb255( 9cm)=(232,255,  0);
    rgb255(10cm)=(255,185,  0);
    rgb255(11cm)=(255, 93,  0);
    rgb255(12cm)=(255,  0,  0);
  }
}

\pgfplotsset{
  colormap={paraviewblueredcolormap}{
    rgb255( 0cm)=(  0,  0,255);
    rgb255( 1cm)=(  0, 93,255);
    rgb255( 2cm)=(  0,185,255);
    rgb255( 3cm)=(  0,255,232);
    rgb255( 4cm)=(  0,255,139);
    rgb255( 5cm)=(  0,255,139);
    rgb255( 6cm)=(  0,255, 46);
    rgb255( 7cm)=( 46,255,  0);
    rgb255( 8cm)=(139,255,  0);
    rgb255( 9cm)=(232,255,  0);
    rgb255(10cm)=(255,185,  0);
    rgb255(11cm)=(255, 93,  0);
    rgb255(12cm)=(255,  0,  0);
  }
}
\end{code}

\levelstay{\texorpdfstring{\protect\marktool{\paraviewname}}{\paraviewname}, \texorpdfstring{\protect\marktool{\abaqusname}}{\abaqusname}, $\ldots$ colorbars in pgfplots}

The colorbar style defines the number of discrete steps in the colorbar. Following are the definition of the default colorbars for \marktool{\abaqusname} and \marktool{ANSYS} as well as an equivalent, but semi-continous, variant for \marktool{\paraviewname}.

\begin{texcode}
\pgfplotsset{
  abaqusdiscrete12colorbar style/.style={
    separate axis lines,
    samples=13,                               % Number of steps+1
  }
}

\pgfplotsset{
  ansysdiscrete9colorbar style/.style={
    separate axis lines,
    samples=10,                               % Number of steps+1
  }
}

\pgfplotsset{
  paraviewdiscrete256colorbar style/.style={
    separate axis lines,
    samples=257,                              % Number of steps+1
  }
}
\end{texcode}

\levelstay{Application of colormaps and colorbars in pgfplots}

Use the colormap with:

\begin{texcode}
\begin{tikzpicture}
  \begin{axis}[
    ...
    colorbar,
    colormap name=paraviewblueredcolormap,
    colorbar style={
      paraviewdiscrete256colorbar style,
      ...
    },
    ...
  \end{axis}
  ...
\end{tikzpicture}
\end{texcode}