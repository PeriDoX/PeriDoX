%%%%%%%%%%%%%%%%%%%%%%%%%%%%%%%%%%%%
% Header                           %
%%%%%%%%%%%%%%%%%%%%%%%%%%%%%%%%%%%%
% 
% Revisions: 2017-04-10 Martin R�del <martin.raedel@dlr.de>
%                       Initial draft
%               
% Contact:   Martin R�del,  martin.raedel@dlr.de
%            DLR Composite Structures and Adaptive Systems
%          
%                                 __/|__
%                                /_/_/_/  
%            www.dlr.de/fa/en      |/ DLR
% 
%%%%%%%%%%%%%%%%%%%%%%%%%%%%%%%%%%%%
% Content                          %
%%%%%%%%%%%%%%%%%%%%%%%%%%%%%%%%%%%%

% \levelstay{Format-specific remarks}
\levelstay{Formats}
\label{sec:Peridigm:QRG:InputFileFormat:FormatSpecificRemarks}

\toolname{} currently supports three different input file formats:

\begin{itemize}[noitemsep]
  \item XML format
  \item \textcolor{gray}{Peridigm/Free format}
  \item YAML format
\end{itemize}

Both, XML and free format are more or less obsolete and will not be part of future developments. For version 1.5 and beyond the focus is on the YAML format.

\leveldown{XML format}
\label{sec:Peridigm:QRG:InputFileFormat:FormatSpecificRemarks:XML}

\paragraph{Comment sign}

A comment line or block is started with \lstinline[style=inlinecodestyle]+<!--+ and ends with \lstinline[style=inlinecodestyle]+-->+

\subparagraph{Single-line comment} \lstinline[style=inlinecodestyle]+<!-------Your comment----->+

\subparagraph{Multi-line comment example} from a generic xml file

\begin{code}
<detail>
  <band height="20">
  <!--
    Hello,
       I am a multi-line XML comment
    <staticText>
      <reportElement x="180" y="0" width="200" height="20"/>
      <text><![CDATA[Hello World!]]></text>
    </staticText>
  -->
  </band>
</detail>
\end{code}

\paragraph{Format interference with the function parser}

When you write a xml format document, you cannot use ``>'' or ``<'' directly. When you edit a xml document, you should use ``\&gt;'' replace ``>'' and ``\&lt;'' replace ``<'', e.g.

\begingroup
\lstset{breaklines=true}
\begin{code}
<Parameter name="Value" type="string" value="if(t &lt;= 10.0e-6)\{value=45.8216*t;\} else \{value=0;\}"/>
\end{code}
\endgroup

instead of

\begingroup
\lstset{breaklines=true}
\begin{code}
<Parameter name="Value" type="string" value="if(t <= 10.0e-6)\{value=45.8216*t;\} else \{ value=0;\}"/>
\end{code}
\endgroup

\levelstay{Free format}
\label{sec:Peridigm:QRG:InputFileFormat:FormatSpecificRemarks:Free}

\begin{warning}
This format is no longer supported in the GitHub master.

For users that have existing .peridigm input decks, a utility for converting them to .yaml is available at

\texttt{scripts/peridigm\_to\_yaml.py}

This script was used for conversion of all the .peridigm files in the repository, with the exception of
twist\_and\_pull.peridigm, which required manual editing to remove aprepro commands.  Aprepro is not supported
for .yaml or .xml input decks, and all aprepro support will go away when support for .peridigm files is removed.
\end{warning}

\paragraph{Comment sign}

A comment line is started with the hash sign: \#.

\levelstay{YAML format}
\label{sec:Peridigm:QRG:InputFileFormat:FormatSpecificRemarks:YAML}

\paragraph{Build requirements}

Note that using the new .yaml input files requires that Trilinos be built with YAML support. See the installation guide for instructions.

\paragraph{Comment sign}

A comment line is started with the hash sign: \#. Comments can start anywhere on a line, and continue until the end of the line. YAML does not support block comments.
