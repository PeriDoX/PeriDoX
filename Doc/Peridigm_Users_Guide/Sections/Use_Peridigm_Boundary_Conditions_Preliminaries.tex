%%%%%%%%%%%%%%%%%%%%%%%%%%%%%%%%%%%%
% Header                           %
%%%%%%%%%%%%%%%%%%%%%%%%%%%%%%%%%%%%
% 
% Revisions: 2017-04-10 Martin R�del <martin.raedel@dlr.de>
%                       Initial draft
%               
% Contact:   Martin R�del,  martin.raedel@dlr.de
%            DLR Composite Structures and Adaptive Systems
%          
%                                 __/|__
%                                /_/_/_/  
%            www.dlr.de/fa/en      |/ DLR
% 
%%%%%%%%%%%%%%%%%%%%%%%%%%%%%%%%%%%%
% Content                          %
%%%%%%%%%%%%%%%%%%%%%%%%%%%%%%%%%%%%

\leveldown{Preliminaries}

\leveldown{Types}

The following boundary condition types are defined in \toolname{}:

\begin{itemize}[noitemsep]
  \item Prescribed\_Displacement
  \item Prescribed\_Fluid\_Pressure\_U
  \item Prescribed\_Temperature
  \item Initial\_Temperature
  \item Initial\_Displacement
  \item Initial\_Velocity
  \item Initial\_Fluid\_Pressure\_U
  \item Body\_Force
\end{itemize}

They are parsed in \verb+~\src\core\Peridigm_BoundaryAndInitialConditionManager.cpp+. The names are defined in \verb+~\src\core\Peridigm_Enums.cpp+. The input of keywords seems to be case-insensitive.

In general only ``Prescribed Displacement'' can be applied in \toolname{} for the application of homogeneous or inhomogeneous kinematic boundary conditions. There is no discrete command for prescribed velocities or accelerations, despite ``Initial Velocity''. However, using the function parser it is still possible to apply these kinds of boundary conditions using the time variable $t$.

Generally it is possible to apply any kind of velocity $\glssymbol{symb:scalar:velocity}$ or acceleration $\glssymbol{symb:scalar:acceleration}$ boundary conditions by considering the relationship to the displacement $\glssymbol{symb:scalar:displacement}$:

\begin{equation}
\label{eq:Peridigm:QRG:BoundaryConditions:Preliminaries:uva}
\glssymbol{symb:scalar:acceleration}(\glssymbol{symb:scalar:time})=\dot{\glssymbol{symb:scalar:velocity}}(\glssymbol{symb:scalar:time})=\ddot{\glssymbol{symb:scalar:displacement}}(\glssymbol{symb:scalar:time})
\end{equation}

\levelstay{Application region}
\myindex[\idxPDKeywordName]{\idxPDKwNodeSet}

Usually the boundary conditions have to be applied to particular material points for example at the free end of the body. Therefore, a nodeset has to be defined in order to list the material points to which the boundary condition has to be applied to.

If an \exodusname{} finite element mesh is used as a discretisation the nodesets might already be defined in the mesh file. However, it is possible to define new nodesets in the \toolname{} input deck as well as it is required for the discretisation via text file.

In order to define nodesets for the discretisation via a .txt file there are two options described in the next two segments

\paragraph{Definition in individual text file}

\subparagraph{Description}

A .txt file has to be generated which contains the numbers of the desired boundary material points from the discretisation .txt file in a column.

% \paragraph{Example}
% \subparagraph{Description}

Let the .txt files "nodeset\_1.txt" and "nodeset\_2.txt" be given as

% \begin{tabularx}{\linewidth}{XcX}
\begin{minipage}{0.425\linewidth}
\begin{code}
1
2
3
4
\end{code}
\end{minipage}
% &%
\hfill%
and %
\hfill%
% &%
\begin{minipage}{0.425\linewidth}
\begin{code}
9
10
11
12
\end{code}
\end{minipage}
% \end{tabularx}

and define the Parameter names ``Node Set One'' and ``Node Set Two'' as

\begin{code}
<Parameter name="Node Set One" type="string" value="nodeset_1.txt"/>
<Parameter name="Node Set Two" type="string" value="nodeset_2.txt"/>
\end{code}

Thus, the boundary condition can be applied to these particular material points by entering the name of the nodeset into the input deck section "Boundary Conditions".

\subparagraph{XML format}%\mbox{}\\

\begingroup
\lstset{breaklines=true}
\begin{code}
<ParameterList name="Boundary Conditions">
  <Parameter name="Node Set One" type="string" value="nodeset_1.txt"/>
  <Parameter name="Node Set Two" type="string" value="nodeset_2.txt"/>
  <ParameterList name="Prescribed Displacement Min X Face">
    <Parameter name="Type" type="string" value="Prescribed Displacement"/>
    <Parameter name="Node Set" type="string" value="Node Set One"/>
    <Parameter name="Coordinate" type="string" value="x"/>
    <Parameter name="Value" type="string" value="0.0"/>
  </ParameterList>
  <ParameterList name="Prescribed Displacement Max X Face">
    <Parameter name="Type" type="string" value="Prescribed Displacement"/>
    <Parameter name="Node Set" type="string" value="Node Set Two"/>
    <Parameter name="Coordinate" type="string" value="x"/>
    <Parameter name="Value" type="string" value="-0.1*t/0.00005"/>
  </ParameterList>
</ParameterList>
\end{code}
\endgroup

\subparagraph{Free format}%\mbox{}\\

In the free format, it is important, that the node set name ends with the term \texttt{Node Set}:

\begingroup
\lstset{breaklines=true}
\begin{code}
Boundary Conditions
  nset_NSET-BC_BOT_SIDE Node Set "nset_NSET-BC_BOT_SIDE.txt"
  Displacement-1-D-x
    Type "Prescribed Displacement"
    Node Set "nset_NSET-BC_BOT_SIDE Node Set"
    Coordinate "x"
    Value "0.0"
\end{code}
\endgroup

\paragraph{Definition inside the model file}

The other possibility to define the nodeset is to list the number of the desired nodes explicitly, i. e. as

\begin{code}
<Parameter name="Node Set Three" type="string" value="2 4"/>
<Parameter name="Node Set Four" type="string" value="1 2"/>
\end{code}

The value of \texttt{Parameter name} can now be used as shown in the previous example.

Free format example:

\begin{code}
Boundary Conditions
  Min X Node Set "1"
  Max X Node Set "2"
  Initial Velocity Min X Face
    Type "Initial Velocity"
    Node Set "Min X Node Set"
    Coordinate "x"
    Value "1.0"
  Initial Velocity Max X Face
    Type "Initial Velocity"
    Node Set "Max X Node Set"
    Coordinate "x"
    Value "-1.0"
\end{code}

\paragraph{Remarks}

\begin{enumerate}[noitemsep]
%  \item The parameter names for the nodesets have to be ``Node Set One'', ``Node Set Two'', \dots
 \item Dependent on the chosen discretisation type, a base FE mesh or the direct specification of the peridynamic collocation points in a textfile, the application region is either the finite element node id or the peridynamic collocation point id.
 \item In case you use a FE mesh as discretisation type and try to define a nodeset in the model file, the base mesh node numbers do not necessarily represent the peridynamic representation collocation point numbers
\end{enumerate}