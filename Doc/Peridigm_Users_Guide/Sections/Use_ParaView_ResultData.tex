%%%%%%%%%%%%%%%%%%%%%%%%%%%%%%%%%%%%
% Header                           %
%%%%%%%%%%%%%%%%%%%%%%%%%%%%%%%%%%%%
% 
% Revisions: 2017-04-10 Martin R�del <martin.raedel@dlr.de>
%                       Initial draft
%               
% Contact:   Martin R�del,  martin.raedel@dlr.de
%            DLR Composite Structures and Adaptive Systems
%          
%                                 __/|__
%                                /_/_/_/  
%            www.dlr.de/fa/en      |/ DLR
% 
%%%%%%%%%%%%%%%%%%%%%%%%%%%%%%%%%%%%
% Content                          %
%%%%%%%%%%%%%%%%%%%%%%%%%%%%%%%%%%%%

\levelstay{Result data}

\marktool{\paraviewname} distinguishes three types of result data:

\begin{multicols}{3}
\begin{itemize}
\addtolength\itemsep{-2ex}
 \item point data
 \item cell data
 \item field data
\end{itemize}
\end{multicols}

Point data is specified at each grid point. Cell data is specified per cell. Field data occurs for user-defined output values. Several linear and non-linear cell types currently exist for one-, two- or three-dimensional applications. E.g. a cell can be a quadrilateral between four nodes in two dimensions, a hexahedron volume between eight nodes in three dimensions or others. All cell types are shown in \cite{vtkfileformats}.

Following is a list of \marktool{\toolname} post-processing items and their classification as point data or cell data.

\begin{multicols}{2}
\begin{itemize}
\addtolength\itemsep{-2ex}
\item Point data results:
  \begin{itemize}
  \addtolength\itemsep{-2ex}
  \item Contact\_Force
  \item Displacement
  \item Force
  \item Global Node Ids
  \item Velocity
  \item $\ldots$
  \end{itemize}
\item Field data results:
  \begin{itemize}
  \addtolength\itemsep{-2ex}
  \item Compute Class Parameter results
  \end{itemize}
\columnbreak
\item Cell data results:
  \begin{itemize}
  \addtolength\itemsep{-2ex}
  \item Damage
  \item Dilatation
  \item Element\_Id
  \item Global Element Ids
  \item Number\_Of\_Neighbors
  \item Object IDs
  \item Proc\_num
  \item Radius
  \item Weighted\_Volume
  \item $\ldots$
  \end{itemize}
\end{itemize}
\end{multicols}