%%%%%%%%%%%%%%%%%%%%%%%%%%%%%%%%%%%%
% Header                           %
%%%%%%%%%%%%%%%%%%%%%%%%%%%%%%%%%%%%
% 
% Revisions: 2017-04-10 Martin Rädel <martin.raedel@dlr.de>
%                       Initial draft
%               
% Contact:   Martin Rädel,  martin.raedel@dlr.de
%            DLR Composite Structures and Adaptive Systems
%          
%                                 __/|__
%                                /_/_/_/  
%            www.dlr.de/fa/en      |/ DLR
% 
%%%%%%%%%%%%%%%%%%%%%%%%%%%%%%%%%%%%
% Content                          %
%%%%%%%%%%%%%%%%%%%%%%%%%%%%%%%%%%%%

\marktool{\fetranslatorname} is a \marktool{\javaname}-based tool to translate models between finite element software. \marktool{\fetranslatorname} implements the conversion of meshes from commercial FE tools into the format that \marktool{\toolname} is capable of using as a discretization for the creation of peridynamic collocation points.

In order to use the \marktool{\fetranslatorname} an implementation of the \marktool{\javaname} Runtime Environment (JRE) is necessary. To translate the mesh into binary format the tool \marktool{\ncgenname} from \marktool{\netcdfname} is required.

\leveldown{Linux}

\leveldown{Java}

\marktool{\opensusename} comes with a pre-installed version of the \marktool{openJDK} which is a free and open source implementation of the Java Platform, Standard Edition (Java SE). \marktool{openJDK} should be perfectly capable of running \marktool{\fetranslatorname}. To see if and which version of \marktool{\javaname} is installed on your system open a shell and type:

\begin{code}
java -version
\end{code}

However, since additions and changes to the \marktool{\fetranslatorname} can be necessary, a \marktool{\javaname}-capable IDE is required. The Oracle \marktool{\javaname} Development Kit (JDK) offers an integrated solution with the JRE and \marktool{\netbeansname} as IDE.

\leveldown{Install only \marktool{\javaname} development kit (JDK)}

\begin{enumerate}[noitemsep]
 \item Go to: \href{http://www.oracle.com/technetwork/java/javase/downloads/index.html}{http://www.oracle.com/technetwork/java/javase/downloads/index.html}
 \item Click on \textit{Java Platform (JDK)}
 \item Accept the License Agreement
 \item Open a shell and type \lstinline[style=inlinecodestyle]+lsb_release -a+ and check your operating system architecture (32bit: i586; 64bit: x86\_64)
 \item Click on the according \verb+rpm+ file for your Linux version (here: jdk-8u91-linux-i586.rpm for 32bit or jdk-8u91-linux-x64.rpm for 64bit, we use 64bit)
 \item In the dialog choose \textit{Save File} and save the file somewhere convenient on your system
 \item Open a root shell or a normal shell and switch to root user with
\begin{code}
su -
\end{code}
 \item Change directory to the folder where the RPM file is located
 \item Type
\begingroup
\lstset{breaklines = true}
\begin{code}
zypper install jdk-8u91-linux-x64.rpm
zypper install update-alternatives
update-alternatives --install /usr/bin/java java /usr/java/jdk1.8.0_91/bin/java 1065
update-alternatives --install /usr/bin/javac javac /usr/java/jdk1.8.0_91/bin/javac 1065
update-alternatives --install /usr/bin/jar jar /usr/java/jdk1.8.0_91/bin/jar 1065
update-alternatives --install /usr/bin/javaws javaws /usr/java/jdk1.8.0_91/bin/javaws 1065
update-alternatives --config java
java -version
nedit /home/$USERNAME/.bashrc &
\end{code}
\endgroup
 \item Add \lstinline[style=inlinecodestyle]+export JAVA_HOME=/usr/java/jdk1.8.0_91/+ to the \verb+.bashrc+-file
\end{enumerate}

\levelstay{Install \marktool{\javaname} development kit (JDK) with \marktool{\netbeansname}}

\begin{enumerate}[noitemsep]
 \item Go to: \href{http://www.oracle.com/technetwork/java/javase/downloads/index.html}{http://www.oracle.com/technetwork/java/javase/downloads/index.html}
 \item Click on \textit{NetBeans with JDK}
 \item Accept the License Agreement
 \item Open a shell and type \lstinline[style=inlinecodestyle]+lsb_release -a+ and check your operating system architecture (32bit: i586; 64bit: x86\_64)
 \item Click on the according \verb+sh+ file for your Linux version\\
 (here: \verb+jdk-8u91-nb-8_1-linux-x64.sh+)
 \item In the dialog choose \textit{Save File} and save the file somewhere convenient on your system
 \item Open a root shell or a normal shell and switch to root user with
\begin{code}
su -
\end{code}
 \item Change directory to the folder where the .sh file is located
 \item Change the installer file's permissions so it can be executed:
\begin{code}
chmod u+x <installer-file-name>
\end{code}
 \item Type
\begin{code}
./<installer-file-name>
\end{code}
 \item In the installation wizard:
 \begin{enumerate}
  \item At the Welcome page of the installation wizard, click \textit{Next}.
  \item At the JDK Installation page, specify the directory where to install the JDK, here \verb+/usr/local/java/jdk1.8.0_91+, and click \textit{Next}.
  \item At the \marktool{\netbeansname} IDE Installation page, do the following:
  \begin{itemize}
   \item Specify the directory for the \marktool{\netbeansname} IDE installation\\
   (here \verb+/usr/local/java/netbeans-8.1+)
   \item Accept the default JDK installation to use with the IDE or specify another JDK location.
  \end{itemize}
  \item Accept the default JDK installation to use with the IDE or specify another JDK location.
  \item Click \textit{Next}
  \item Review the Summary page to ensure the software installation locations are correct.
  \item Click \textit{Intall} to begin the installation.
  \item At the Setup Complete page, provide anonymous usage data if desired, and click \textit{Finish}
  \item When the installation is complete, you can view the log file, which resides in the following directory: \verb+~/.nbi/log.+
 \end{enumerate}
 \item Type
\begingroup
\lstset{breaklines = true}
\begin{code}
zypper install update-alternatives
update-alternatives --install /usr/bin/java java /usr/local/java/jdk1.8.0_91/bin/java 1065
update-alternatives --install /usr/bin/javac javac /usr/local/java/jdk1.8.0_91/bin/javac 1065
update-alternatives --install /usr/bin/jar jar /usr/local/java/jdk1.8.0_91/bin/jar 1065
update-alternatives --install /usr/bin/javaws javaws /usr/local/java/jdk1.8.0_91/bin/javaws 1065
update-alternatives --config java
java -version
nedit /home/$USERNAME/.bashrc &
\end{code}
\endgroup
 \item Add
\begin{code}
export JAVA_HOME=/usr/java/jdk1.8.0_91/
\end{code}
 and
\begin{code}
export PATH=$PATH:/usr/local/java/netbeans-8.1/bin
\end{code}
 to the \verb+.bashrc+-file
 \item Start a new shell with \verb+su - $USERNAME+ and type \verb+java -version+ to see if the correct version is active
 \item Start a new shell with \verb+su - $USERNAME+ and type \verb+netbeans &+ to start the IDE
 \item Perform update inside the IDE if asked for
\end{enumerate}

If problems occur during any \verb+update-alternatives --install+ try

% https://hschwarz77.wordpress.com/2015/08/14/install-java-alternatives/
\begingroup
\lstset{breaklines = true}
\begin{code}
update-alternatives --install /usr/bin/java java /usr/local/java/jdk1.8.0_91/bin/java 1
update-alternatives --install /usr/lib64/browser-plugins/javaplugin.so javaplugin /usr/local/java/jdk1.8.0_91/jre/lib/amd64/libnpjp2.so 1 --slave /usr/bin/javaws javaws /usr/local/java/jdk1.8.0_91/bin/javaws
update-alternatives --install /usr/bin/javac javac /usr/local/java/jdk1.8.0_91/bin/javac 1 --slave /usr/bin/jar jar /usr/local/java/jdk1.8.0_91/bin/jar
\end{code}
\endgroup

Now you can set the \marktool{\javaname} priorities with:

\begin{code}
update-alternatives --config java
update-alternatives --config javac
update-alternatives --config javaplugin
\end{code}

\levelup{\texorpdfstring{\protect\marktool{\netcdfname{}}}{\netcdfname{}}}

The \marktool{\netcdfname}-tool \marktool{\ncgenname} is required to convert the ascii mesh file into the binary format readable by \marktool{\toolname}. \marktool{\netcdfname} should already be installed to use \marktool{\toolname}. If the additions to the \verb+PATH+-variable from \autoref{sec:Install_netcdf} are set, no further actions have to be performed.

\levelup{Windows}

\leveldown{Java}

\begin{enumerate}[noitemsep]
 \item Go to: \href{http://www.oracle.com/technetwork/java/javase/downloads/index.html}{http://www.oracle.com/technetwork/java/javase/downloads/index.html}
 \item Click on \textit{NetBeans with JDK} for Development Kit and Netbeans or just \textit{Java Platform (JDK)}
 \item Perform the installation
\end{enumerate}

\levelstay{\texorpdfstring{\protect\marktool{\netcdfname{}}}{\netcdfname{}}}

To test the \marktool{\fetranslatorname} under Windows it is necessary to have \marktool{\ncgenname} available. \marktool{\ncgenname} is available as part of pre-built \marktool{\netcdfname} libraries. To install the latest release

\begin{enumerate}[noitemsep]
 \item Go to: \href{\netcdfaddress}{\netcdfaddress}
 \item Click \textit{Pre-built Windows Binaries for the latest version of \netcdfname}
 \item Go to \textit{Latest Release (\netcdfname{} X.Y.Z)}, here \netcdfname{} 4.4.0
 \item Download the executable matching your system, here netCDF4.4.0-NC4-64.exe
 \item Execute the installer
 \item Add the \verb+bin+ folder of the installation path, here \verb+D:\Programme\netCDF 4.4.0\+ to the Windows \verb+PATH+-Variable:
 \begin{enumerate}
  \item Open the \textit{Windows Control Panel} (Systemsteuerung)
  \item Open \textit{System}
  \item Click \textit{Advanced System Settings} (Erweiterte Systemeinstellungen)
  \item In the \textit{Advanced} tab open \textit{Environment Variables}
  \item Under \textit{User variables for USERNAME} select \verb+PATH+
  \item Click \textit{Edit}
  \item Under \textit{Value of the variable} add the path to the \verb+bin+ folder of the \marktool{\netcdfname} installation separated by a semicolon (\textit{;}), here: \verb+;D:\Programme\netCDF 4.4.0\bin\+
  \item Click \textit{OK} multiple times
 \end{enumerate}
\end{enumerate}

Now you can use \marktool{\ncgenname} in a command-window:

\begin{code}
ncgen.exe -o $OUTPUTFILENAME.g $INPUTFILENAME.g.ascii
\end{code}

