%%%%%%%%%%%%%%%%%%%%%%%%%%%%%%%%%%%%
% Header                           %
%%%%%%%%%%%%%%%%%%%%%%%%%%%%%%%%%%%%
% 
% Revisions: 2017-04-10 Martin Rädel <martin.raedel@dlr.de>
%                       Initial draft
%               
% Contact:   Martin Rädel,  martin.raedel@dlr.de
%            DLR Composite Structures and Adaptive Systems
%          
%                                 __/|__
%                                /_/_/_/  
%            www.dlr.de/fa/en      |/ DLR
% 
%%%%%%%%%%%%%%%%%%%%%%%%%%%%%%%%%%%%
% Content                          %
%%%%%%%%%%%%%%%%%%%%%%%%%%%%%%%%%%%%

\levelstay{\texorpdfstring{\protect\marktool{\hdfname}}{\hdfname{}}}

\marktool{\hdfname} is a data model, library, and file format for storing and managing data. It supports an unlimited variety of datatypes, and is designed for flexible and efficient I/O and for high volume and complex data. For further information visit

\href{\hdfaddress}{\hdfaddress}

\marktool[\hdfaddress]{\hdfname} version \textbf{1.8.9 or newer} is required by \marktool{\netcdfname} and the \marktool{SEACAS} \marktool{\trilinosname} package. \marktool[\hdfaddress]{\hdfname} should be configured with the \verb+--enable-parallel option+.

\paragraph{Installation with \texorpdfstring{\protect\marktool{\opensusename}}{\opensusename}-repository}

\marktool{\hdfname} is available in an \marktool{\opensusename}-repository and can be installed using \marktool{\zyppername}. However, it is recommended to use the manual install with the Gzipped tarball to make sure the correct options are set.

\begingroup
\lstset{breaklines=true}
\begin{code}
zypper addrepo http://download.opensuse.org/repositories/home:ocefpaf/openSUSE_Tumbleweed/home:ocefpaf.repo
zypper refresh
zypper install hdf5
\end{code}
\endgroup

\paragraph{Installation from source} The \marktool{\hdfname} source code is available from

\href{\hdfaddress}{\hdfaddress}

Download the source code for your platform

\begingroup
\lstset{breaklines=true}
\begin{code}
cd $DOWNLOAD_DIR
wget http://www.hdfgroup.org/ftp/HDF5/current/src/hdf5-1.8.16.tar.gz
tar xvfz hdf5-1.8.16.tar.gz 	%\lstcomment{\# unzip}%
cd hdf5-1.8.16			%\lstcomment{\# go into directory}%
\end{code}
\endgroup

Afterwards, create the \marktool{\hdfname} build script as described in section \ref{sec:Build-script_HDF}. In order to use the script make it executable as described in section \ref{sec:Build-script_Executable}. Open a terminal as root, change directory to the created install script and execute it with

\begin{code}
./install_hdf.sh > install.log 2>&1
\end{code}

Afterwards, the \marktool{\hdfname}-directory has to be added to the \verb+PATH+ and \verb+LD_LIBRARY_PATH+ environment variable

\begingroup
\lstset{breaklines = true}
\lstinputlisting[
  style=scriptstyle,
  linerange={47-48},
  nolol,
  ]{Scripts/modified_bashrc.txt}
\endgroup 