%%%%%%%%%%%%%%%%%%%%%%%%%%%%%%%%%%%%
% Header                           %
%%%%%%%%%%%%%%%%%%%%%%%%%%%%%%%%%%%%
% 
% Revisions: 2017-04-10 Martin R�del <martin.raedel@dlr.de>
%                       Initial draft
%               
% Contact:   Martin R�del,  martin.raedel@dlr.de
%            DLR Composite Structures and Adaptive Systems
%          
%                                 __/|__
%                                /_/_/_/  
%            www.dlr.de/fa/en      |/ DLR
% 
%%%%%%%%%%%%%%%%%%%%%%%%%%%%%%%%%%%%
% Content                          %
%%%%%%%%%%%%%%%%%%%%%%%%%%%%%%%%%%%%

\levelup{\idxPDKwViscoelastic}
\label{sec:Peridigm:QRG:Materials:Viscoelastic}
\myindex[\idxPDKeywordName]{\idxPDKwViscoelastic}
\myindex[\idxPDKeywordName]{\idxPDKwMaterials!\idxPDKwViscoelastic|see{\idxPDKwViscoelastic}}

\leveldown{Description}

An isotropic, viscoelastic material model.

\levelstay{Literature}

\begin{itemize}[noitemsep]
  \item \cite{MitchellJA2011b}
  \item \cite{PeridigmUserGuide100}
\end{itemize}

\levelstay{Stiffness model sketch}

\begin{figure}[htbp]
  \begin{subfigure}{0.49\linewidth}
    \centering
    %\begin{tikzpicture}
  % Variable
  \def\modulus{70000}
  \def\hardeningmodulus{4000}
  \def\yieldstress{350}
  \def\failstrain{0.025}
  \def\unloadingfac{0.8}
  \def\xLabel{$\glssymbol{symb:scalar:mech:strain:normal:engineering}$}
  \def\yLabel{$\glssymbol{symb:scalar:mech:stress:normal:engineering}$}
  \def\pinELabel{$\glssymbol{symb:scalar:mat:modulus:bulk}$}
  \def\pinEHLabel{$\glssymbol{symb:scalar:mat:modulus:bulk}_{\glssymbol{symb:index:hardening}}$}
  \def\pinEPos{0.5}
  \def\pinEHPos{0.3}
  \def\yieldlabel{$\glssymbol{symb:scalar:mech:stress:normal:engineering}_{\glssymbol{symb:index:yield}}$}
  \newtoggle{tclabel}
  \toggletrue{tclabel}
    % Math
  \pgfkeys{/pgf/fpu}
  \pgfmathsetmacro{\yieldstrain}{\yieldstress/\modulus}
  \pgfmathsetmacro{\failstress}{\yieldstress+\hardeningmodulus*(\failstrain-\yieldstrain)}
  \pgfmathsetmacro{\unloadingstress}{\yieldstress+\unloadingfac*\hardeningmodulus*(\failstrain-\yieldstrain)}
  \pgfmathsetmacro{\unloadingstrain}{\yieldstrain+\unloadingfac*(\failstrain-\yieldstrain)}
  \pgfkeys{/pgf/fpu=false}
  % Shapes
  \tikzset{%
    myarrowdecoration1/.style={postaction={decorate,decoration={
      markings,
      mark=between positions .4 and .6 step .1pt with {\draw [thin] circle (.1pt);},
      mark=at position .6 with {\arrow[thin,xshift=1pt]{latex}},
      raise=-0.7ex,
    }}},
    myarrowdecoration2/.style={postaction={decorate,decoration={
      markings,
      mark=between positions .4 and .6 step .1pt with {\draw [thin] circle (.1pt);},
      mark=at position .4 with {\arrow[thin,xshift=1pt]{latex reversed}},
      raise=0.7ex,
    }}},
  }
  % Axis
  \begin{axis}[
%     scale only axis,
    axis lines=middle,
    ticks=none,
    %restrict y to domain=-\ultimatestrength:\ultimatestrength,
    xmin=-1.1*\failstrain,
    xmax= 1.1*\failstrain,
    ymin=-1.5*\yieldstress,
    ymax= 1.5*\yieldstress,
    width=0.99\textwidth,
    height=0.99\textwidth,
    xlabel=\xLabel,
    ylabel=\yLabel,
    every axis x label/.style={
      at={(ticklabel* cs:1.005)},
      anchor=west,
    },
    every axis y label/.style={
      at={(ticklabel* cs:1.005)},
      anchor=south,
    },
  ]
    % Coordinates
    \coordinate (origin)     at (0,0);
    \coordinate (yieldt)     at ( \yieldstrain, \yieldstress);
    \coordinate (yieldc)     at (-\yieldstrain,-\yieldstress);
    \coordinate (failuret)   at ( \failstrain, \failstress);
    \coordinate (failurec)   at (-\failstrain,-\failstress);
    % Lines
    \draw[thick,draw=\plotcolor] (origin) -- node[pos=\pinEPos, pin={[pin distance=1ex]-60:{\pinELabel}}](ELabel){} (yieldt);
    \draw[thick,draw=\plotcolor] (yieldt) -- node[pos=\pinEHPos, pin={[pin distance=1ex]120:{\pinEHLabel}}](EHLabel){} (failuret);
    \draw[thick,draw=\plotcolor] (origin) -- (yieldc);
    \draw[thick,draw=\plotcolor] (yieldc) -- (failurec);
    
    \draw[dashed] (origin|-yieldt) node[anchor=east]{\yieldlabel} -- (yieldt);
    % Loading/Unloading
    \draw[dashed,myarrowdecoration1,myarrowdecoration2] (\unloadingstrain,\unloadingstress) -- (\unloadingstrain-\yieldstrain,0);
    % Label
    \iftoggle{tclabel}{%
      \node[anchor=north east] (tensionlabel) at (rel axis cs:1,1) {\footnotesize tension};
      \node[anchor=south west] (compressionlabel) at (rel axis cs:0,0) {\footnotesize compression};
    }{}
  \end{axis}
\end{tikzpicture}
    \includegraphics[width=0.75\textwidth]{example-image-a}
    \caption{Axial}
    \label{fig:Material_Models_ViscoElastic-K}
  \end{subfigure}%
  \begin{subfigure}{0.49\linewidth}
    \centering
    %\input{Figures/Theory/Material_Model_Linear-Elastic-G}
    \includegraphics[width=0.75\textwidth]{example-image-a}
    \caption{Shear}
    \label{fig:Material_Models_ViscoElastic-G}
  \end{subfigure}%
  \caption{Viscoelastic material model}
  \label{fig:Material_Models_ViscoElastic}
\end{figure}

\levelstay{Code}

\leveldown{Release version}

Available from \href{\toolrepoversiononetwo}{version 1.2}.

\levelstay{Required compiler options}

-

\levelstay{Routines}

\begin{itemize}[noitemsep]
  \item IO:
  \begin{itemize}[noitemsep]
    \item \verb+/src/materials/Peridigm_ViscoelasticMaterial.cpp+
    \item \verb+/src/materials/Peridigm_ViscoelasticMaterial.hpp+
  \end{itemize}
  \item Computation:
  \begin{itemize}[noitemsep]
    \item \verb+/src/materials/viscoelastic.cxx+
    \item \verb+/src/materials/viscoelastic.h+
  \end{itemize}
\end{itemize}

\levelup{Input parameters}

\leveldown{List}

\begin{tabularx}{\linewidth}{lcccX}
\toprule
Name			& Type		& Required	& Default	& Description		\\
\midrule
Material Model		& string	& \checkmark	& -		& Material type ``\idxPDKwViscoelastic''		\\
Density			& double	& \checkmark	& -		& Material density	\\
lambda\_i		& double	& \checkmark\textsuperscript{\ref{enm:Peridigm:QRG:Materials:Viscoelastic:Remark:One}}	& -		& Rate of relaxation 	\\
tau b			& double	& \checkmark	& -		& Relaxation time constant		\\
\bottomrule
\end{tabularx}

\levelstay{Remarks}

\begin{enumerate}[noitemsep]
  \item \label{enm:Peridigm:QRG:Materials:Viscoelastic:Remark:One} $\num{0.0}\le\lambda_i\le\num{1.0}$
  \item Automatic Differentiation is not supported for the Viscoelastic material model
  \item Shear Correction Factor is not supported for the Viscoelastic material model
  \item Thermal expansion is not currently supported for the Viscoelastic material model
\end{enumerate}

\levelup{Exemplary input section}

\leveldown{XML-format}

-

\levelstay{Free format}

-

\levelstay{YAML format}

-
  
\levelup{Possible output variables for the material model}

\begin{multicols}{2}
\begin{itemize}[noitemsep]
  \item Bond\_Damage
  \item Coordinates
  \item Damage
  \item Deviatoric\_Back\_Extension
  \item Dilatation
  \item Force\_Density
  \item Model\_Coordinates
  \item Volume
  \item Weighted\_Volume
\end{itemize}
\end{multicols}

\levelstay{List of examples}

% \begin{itemize}[noitemsep]
%   \item From \texttt{examples/}:
%   \begin{itemize}[noitemsep]
%     \item \texttt{examples/tensile\_test/tensile\_test.peridigm}
%   \end{itemize}
%   \item From \texttt{test/regression/}:
%   \begin{itemize}[noitemsep]
%     \item \texttt{Bar_OneBlock_OneMaterial_QS/Bar.xml}
%   \end{itemize}
%   \item From \texttt{test/verification/}:
%   \begin{itemize}[noitemsep]
%     \item \texttt{Contact\_Friction\_Time\_Dependent\_Coefficient/Contact\_Friction\_Time\_Dependent\_Coefficient.xml}
%   \end{itemize}
% \end{itemize}