%%%%%%%%%%%%%%%%%%%%%%%%%%%%%%%%%%%%
% Header                           %
%%%%%%%%%%%%%%%%%%%%%%%%%%%%%%%%%%%%
% 
% Revisions: 2017-04-10 Martin Rädel <martin.raedel@dlr.de>
%                       Initial draft
%               
% Contact:   Martin Rädel,  martin.raedel@dlr.de
%            DLR Composite Structures and Adaptive Systems
%          
%                                 __/|__
%                                /_/_/_/  
%            www.dlr.de/fa/en      |/ DLR
% 
%%%%%%%%%%%%%%%%%%%%%%%%%%%%%%%%%%%%
% Content                          %
%%%%%%%%%%%%%%%%%%%%%%%%%%%%%%%%%%%%

\leveldown{\marktool{\openmpiname}}

\marktool{\openmpiname} is an open source Message Passing Interface implementation that is developed and maintained by a consortium of academic, research, and industry partners. The current version and further information can be found at

\href{https://www.open-mpi.org}{https://www.open-mpi.org}

\paragraph{Installation with \marktool{\yastname}}

To install \marktool{\openmpiname} with the package manager perform the following steps:

\begin{enumerate}[itemsep=-1.5ex]
 \item Open \marktool{\yastname}
 \item Click on \textit{Install software}
 \item Go to the \textit{Search} tab
 \item Search for \openmpiname
 \item Check \openmpiname
 \item Click on apply
\end{enumerate}

\paragraph{Installation with \marktool{\opensusename}-repository}

The \marktool{\openmpiname}-repository is part of the openSUSE distribution. Therefore, it can be directly installed from the system repositories.

To use \marktool{\zyppername} open a terminal as root. Use the following commands to install  \marktool{\openmpiname} from the repositories. Answer the questions if installation shall continue with yes.

\begin{code}
zypper install openmpi
\end{code}

\paragraph{Installation from source}

The Gzipped tarball source files can be obtained from

\href{https://www.open-mpi.org/software/ompi/}{https://www.open-mpi.org/software/ompi/}

In the subfolder choose the version of your liking. Change directory to the designated download folder, e.g. \verb+/usr/local/src/+ and perform the following steps:

\begingroup
\lstset{breaklines=true}
\begin{code}
cd $DOWNLOAD_DIR
wget https://www.open-mpi.org/software/ompi/v1.10/downloads/openmpi-1.10.2.tar.gz	
tar xvfz openmpi-1.10.2.tar.gz	%\lstcomment{\# unzip}%
cd openmpi-1.10.2 %\lstcomment{\# go into directory}%
./configure --prefix=/usr/local/lib/openmpi-1.10.2 > configure_openmpi.log 2>&1
make > make_openmpi.log 2>&1 %\lstcomment{\# build}%
make altinstall > make_install_openmpi.log 2>&1
\end{code}
% make altinstall > make_install_openmpi.log 2>&1 %\lstcomment{\# install}%
\endgroup

If no previous version of \marktool{\openmpiname} exists use \verb+make install+ instead of \verb+make altinstall+.

\paragraph{Set the \texttt{PATH} variables}

Unfortunately the \marktool{\openmpiname} installation does not work out of the box. You need to set the \verb+PATH+ and \verb+LD_LIBRARY_PATH+ variables and edit a configuration file first.

The \verb+LD_LIBRARY_PATH+ must be set so that \marktool{mpi4py} can find the \marktool{\openmpiname} libraries.% To set the variable we must first find out where the \marktool{\openmpiname} libs can be found, to do so execute:
% For 32 bit machines:
%
% \begin{code}
% mpicxx -showme:link
% -pthread -L/usr/lib/mpi/gcc/openmpi/lib -lmpi_cxx -lmpi
% -lopen-rte -lopen-pal -ldl -Wl,--export-dynamic -lnsl -lutil -lm -ldl
% \end{code}
%
% For 64 bit machines:
%
% \begin{code}
% mpicxx -showme:link
% -pthread -L/usr/lib64/mpi/gcc/openmpi/lib64 -lmpi_cxx -lmpi
% -lopen-rte -lopen-pal -ldl -Wl,--export-dynamic -lnsl -lutil -lm -ldl
% \end{code}
% We need to set \verb+LD_LIBRARY_PATH+ variable to the path after the \verb+-L+ in the output (so in this example case \verb+/usr/lib/mpi/gcc/openmpi/lib+ or \verb+/usr/lib64/mpi/gcc/openmpi/lib64+).

In bash do for 32-bit

\begingroup
\lstset{breaklines=true}
\begin{code}
export PATH=$PATH:/usr/local/lib/openmpi-1.10.2/bin
export LD_LIBRARY_PATH=$LD_LIBRARY_PATH:/usr/local/lib/openmpi-1.10.2/lib
\end{code}
\endgroup

or 64-bit

\begingroup
\lstset{breaklines = true}
\lstinputlisting[
  style=scriptstyle,
  linerange={40-41},
  nolol,
  ]{Scripts/modified_bashrc.txt}
\endgroup

We recommend you add this line to your \verb+.bashrc+ file in case you use \marktool{Bash} or call \verb+setenv+ and edit the \verb+.cshrc+ file if you use a C shell so that the variable is set correctly for all sessions. For the modification of user \verb+.bashrc+ file for all libraries, please consult section \ref{sec:Build-script_Bashrc}. 