%%%%%%%%%%%%%%%%%%%%%%%%%%%%%%%%%%%%
% Header                           %
%%%%%%%%%%%%%%%%%%%%%%%%%%%%%%%%%%%%
% 
% Revisions: 2017-04-10 Martin R�del <martin.raedel@dlr.de>
%                       Initial draft
%               
% Contact:   Martin R�del,  martin.raedel@dlr.de
%            DLR Composite Structures and Adaptive Systems
%          
%                                 __/|__
%                                /_/_/_/  
%            www.dlr.de/fa/en      |/ DLR
% 
%%%%%%%%%%%%%%%%%%%%%%%%%%%%%%%%%%%%
% Content                          %
%%%%%%%%%%%%%%%%%%%%%%%%%%%%%%%%%%%%

\levelup{Elastic Plastic Hardening Correspondence}
% \levelup{\idxPDKwElasticPlasticHardeningCorrespondence}
\label{sec:Peridigm:QRG:Materials:ElasticPlasticHardeningCorrespondence}
\myindex[\idxPDKeywordName]{\idxPDKwElasticPlasticHardeningCorrespondence}
\myindex[\idxPDKeywordName]{\idxPDKwMaterials!\idxPDKwElasticPlasticHardeningCorrespondence|see{\idxPDKwElasticPlasticHardeningCorrespondence}}

\leveldown{Description}

A linear-elastic linear-hardening correspondence material model.

\levelstay{Stiffness model sketch}

\begin{figure}[htbp]
  \begin{subfigure}{0.49\linewidth}
    \centering
    \begin{tikzpicture}
  % Variable
  \def\modulus{70000}
  \def\hardeningmodulus{4000}
  \def\yieldstress{350}
  \def\failstrain{0.025}
  \def\unloadingfac{0.8}
  \def\xLabel{$\glssymbol{symb:scalar:mech:strain:normal:engineering}$}
  \def\yLabel{$\glssymbol{symb:scalar:mech:stress:normal:engineering}$}
  \def\pinELabel{$\glssymbol{symb:scalar:mat:modulus:bulk}$}
  \def\pinEHLabel{$\glssymbol{symb:scalar:mat:modulus:bulk}_{\glssymbol{symb:index:hardening}}$}
  \def\pinEPos{0.5}
  \def\pinEHPos{0.3}
  \def\yieldlabel{$\glssymbol{symb:scalar:mech:stress:normal:engineering}_{\glssymbol{symb:index:yield}}$}
  \newtoggle{tclabel}
  \toggletrue{tclabel}
    % Math
  \pgfkeys{/pgf/fpu}
  \pgfmathsetmacro{\yieldstrain}{\yieldstress/\modulus}
  \pgfmathsetmacro{\failstress}{\yieldstress+\hardeningmodulus*(\failstrain-\yieldstrain)}
  \pgfmathsetmacro{\unloadingstress}{\yieldstress+\unloadingfac*\hardeningmodulus*(\failstrain-\yieldstrain)}
  \pgfmathsetmacro{\unloadingstrain}{\yieldstrain+\unloadingfac*(\failstrain-\yieldstrain)}
  \pgfkeys{/pgf/fpu=false}
  % Shapes
  \tikzset{%
    myarrowdecoration1/.style={postaction={decorate,decoration={
      markings,
      mark=between positions .4 and .6 step .1pt with {\draw [thin] circle (.1pt);},
      mark=at position .6 with {\arrow[thin,xshift=1pt]{latex}},
      raise=-0.7ex,
    }}},
    myarrowdecoration2/.style={postaction={decorate,decoration={
      markings,
      mark=between positions .4 and .6 step .1pt with {\draw [thin] circle (.1pt);},
      mark=at position .4 with {\arrow[thin,xshift=1pt]{latex reversed}},
      raise=0.7ex,
    }}},
  }
  % Axis
  \begin{axis}[
%     scale only axis,
    axis lines=middle,
    ticks=none,
    %restrict y to domain=-\ultimatestrength:\ultimatestrength,
    xmin=-1.1*\failstrain,
    xmax= 1.1*\failstrain,
    ymin=-1.5*\yieldstress,
    ymax= 1.5*\yieldstress,
    width=0.99\textwidth,
    height=0.99\textwidth,
    xlabel=\xLabel,
    ylabel=\yLabel,
    every axis x label/.style={
      at={(ticklabel* cs:1.005)},
      anchor=west,
    },
    every axis y label/.style={
      at={(ticklabel* cs:1.005)},
      anchor=south,
    },
  ]
    % Coordinates
    \coordinate (origin)     at (0,0);
    \coordinate (yieldt)     at ( \yieldstrain, \yieldstress);
    \coordinate (yieldc)     at (-\yieldstrain,-\yieldstress);
    \coordinate (failuret)   at ( \failstrain, \failstress);
    \coordinate (failurec)   at (-\failstrain,-\failstress);
    % Lines
    \draw[thick,draw=\plotcolor] (origin) -- node[pos=\pinEPos, pin={[pin distance=1ex]-60:{\pinELabel}}](ELabel){} (yieldt);
    \draw[thick,draw=\plotcolor] (yieldt) -- node[pos=\pinEHPos, pin={[pin distance=1ex]120:{\pinEHLabel}}](EHLabel){} (failuret);
    \draw[thick,draw=\plotcolor] (origin) -- (yieldc);
    \draw[thick,draw=\plotcolor] (yieldc) -- (failurec);
    
    \draw[dashed] (origin|-yieldt) node[anchor=east]{\yieldlabel} -- (yieldt);
    % Loading/Unloading
    \draw[dashed,myarrowdecoration1,myarrowdecoration2] (\unloadingstrain,\unloadingstress) -- (\unloadingstrain-\yieldstrain,0);
    % Label
    \iftoggle{tclabel}{%
      \node[anchor=north east] (tensionlabel) at (rel axis cs:1,1) {\footnotesize tension};
      \node[anchor=south west] (compressionlabel) at (rel axis cs:0,0) {\footnotesize compression};
    }{}
  \end{axis}
\end{tikzpicture}
    \caption{Axial}
    \label{fig:Material_Models_LinearElasticLinearHardeningCorrespondence-K}
  \end{subfigure}%
  \begin{subfigure}{0.49\linewidth}
    \centering
    \input{Figures/Theory/Material_Model_ElasticPlastic-Linear-Linear-G}
    \caption{Shear}
    \label{fig:Material_Models_LinearElasticLinearHardeningCorrespondence-G}
  \end{subfigure}%
  \caption{Linear-elastic linear-hardening correspondence material model}
  \label{fig:Material_Models_LinearElasticLinearHardeningCorrespondence}
\end{figure}

\levelstay{Code}

\leveldown{Release version}

Available from \href{\toolrepoversiononetwo}{version 1.2}.

\levelstay{Required compiler options}

-

\levelstay{Routines}

\begin{itemize}[noitemsep]
  \item IO:
  \begin{itemize}[noitemsep]
  \item \verb+/src/materials/Peridigm_IsotropicHardeningPlasticCorrespondenceMaterial.cpp+
  \item \verb+/src/materials/Peridigm_IsotropicHardeningPlasticCorrespondenceMaterial.hpp+
  \end{itemize}
  \item Computation:
  \begin{itemize}[noitemsep]
    \item \verb+/src/materials/isotropic_hardening_correspondence.cxx+
    \item \verb+/src/materials/isotropic_hardening_correspondence.h+
  \end{itemize}
\end{itemize}

\levelup{Input parameters}

\leveldown{List}

% \begin{tabularx}{\linewidth}{lcccX}
% \toprule
% Name            & Type          & Required      & Default       & Description           \\
% \midrule
\begin{filecontents}{\tabledir\jobname-parammatelasticplastichardeningcorrespondenceltxtable.tex}
\begin{longtable}{@{}lcccX@{}}
% ---------------------------
% Header & Footer
% ---------------------------
%
% Header
% -----------------
% 1st head
\toprule
Name          & Type          & Required      & Default       & Description           \\
\midrule
\endfirsthead
% Last head
\multicolumn{5}{@{}l}{\ldots continued}\\
\toprule
Name          & Type          & Required      & Default       & Description           \\
\midrule
\endhead
%
% Footer
% -----------------
% n-th foot
\bottomrule
\multicolumn{5}{r@{}}{continued \ldots}\\
\endfoot
% last foot
\bottomrule
\endlastfoot
% ---------------------------
% Content
% ---------------------------
Material Model & string & -          & - & Material type ``\idxPDKwElasticPlasticHardeningCorrespondence''  \\
Density        & double & \checkmark & - & Material density \\
Bulk modulus   & double & \checkmark\textsuperscript{\ref{enm:Peridigm:QRG:Materials:ElasticPlasticHardeningCorrespondence:Remark:Modulus:One},\ref{enm:Peridigm:QRG:Materials:ElasticPlasticHardeningCorrespondence:Remark:Modulus:Two}} & - & Volumetric elasticity\\
Shear Modulus  & double & \checkmark\textsuperscript{\ref{enm:Peridigm:QRG:Materials:ElasticPlasticHardeningCorrespondence:Remark:Modulus:One},\ref{enm:Peridigm:QRG:Materials:ElasticPlasticHardeningCorrespondence:Remark:Modulus:Two}}       & - & Shear elasticity or engineering constant for shear stiffness\\
Young\verb+'+s modulus  & double & \checkmark\textsuperscript{\ref{enm:Peridigm:QRG:Materials:ElasticPlasticHardeningCorrespondence:Remark:Modulus:One},\ref{enm:Peridigm:QRG:Materials:ElasticPlasticHardeningCorrespondence:Remark:Modulus:Two}} & - & Axial stiffness  \\
Poisson\verb+'+s ratio  & double & \checkmark\textsuperscript{\ref{enm:Peridigm:QRG:Materials:ElasticPlasticHardeningCorrespondence:Remark:Modulus:One},\ref{enm:Peridigm:QRG:Materials:ElasticPlasticHardeningCorrespondence:Remark:Modulus:Two}} & - & Transverse contraction\\
Yield Stress  & double & \checkmark & -  & Yield stress  \\
Hardening modulus & double & \checkmark & -  & Linear hardening modulus \\
Apply Shear Correction Factor & bool  & -  & true  &    \\
% Apply Automatic Differentiation Jacobian& bool & - & ? &   \\
Hourglass Coefficient\textsuperscript{\ref{enm:Peridigm:QRG:Materials:ElasticPlasticHardeningCorrespondence:Remark:HourglassCoefficient}} & double & \checkmark &  \\
Disable Plasticity & bool  & -  & false  & \\
\end{longtable}
\end{filecontents}
% \bottomrule
% \end{tabularx}

\begingroup
\LTXtable{\linewidth}{\tabledir\jobname-parammatelasticplastichardeningcorrespondenceltxtable.tex}
\endgroup

\levelstay{Remarks}

\begin{enumerate}[noitemsep]
  \item \label{enm:Peridigm:QRG:Materials:ElasticPlasticHardeningCorrespondence:Remark:Modulus:One} The stiffness can be defined by either elastic modulus combination: Volumetric and shear elasticity ($\glssymbol{symb:scalar:mat:modulus:bulk}$,$\glssymbol{symb:scalar:mat:modulus:shear}$) or the engineering constants ($\glssymbol{symb:scalar:mat:modulus:young}$,$\glssymbol{symb:scalar:mat:poissonratio}$,$\glssymbol{symb:scalar:mat:modulus:shear}$)
  \item \label{enm:Peridigm:QRG:Materials:ElasticPlasticHardeningCorrespondence:Remark:Modulus:Two} In case engineering constants are used, only two of the three values \textit{Young's Modulus}, \textit{Poisson's Ratio} and \textit{Shear Modulus} have to be specified. The missing value is calculated from
  \begin{align*}
  \glssymbol{symb:scalar:mat:modulus:shear}&=\dfrac{\glssymbol{symb:scalar:mat:modulus:young}}{2\cdot\left(1+\glssymbol{symb:scalar:mat:poissonratio}\right)} &
  \glssymbol{symb:scalar:mat:modulus:shear}_{\glssymbol{symb:index:hardening}}&=\dfrac{\glssymbol{symb:scalar:mat:modulus:young}_{\glssymbol{symb:index:hardening}}}{2\cdot\left(1+\glssymbol{symb:scalar:mat:poissonratio}\right)}
  \end{align*}
  Internally, the engineering constants are converted to ($\glssymbol{symb:scalar:mat:modulus:bulk}$,$\glssymbol{symb:scalar:mat:modulus:shear}$).
  \item Yield stress is an equivalent uniaxial stress and therefore applicable to axial, shear as well as mixed stress states.
  \item \label {enm:Peridigm:QRG:Materials:ElasticPlasticHardeningCorrespondence:Remark:HourglassCoefficient} Hourglass coefficient is usually between $\num{0.0}$ and $\num{0.05}$
  \item Automatic Differentiation is not supported for the correspondence material model
  \item Shear Correction Factor is not supported for the correspondence material model
  \item Thermal expansion is not currently supported for the correspondence material model
  \item Consider the general remarks on correspondence materials in section \ref{sec:Peridigm:QRG:Materials:Preliminaries:Correspondence}
\end{enumerate}

\levelup{Exemplary input section}

\leveldown{XML-format}

\begingroup
\lstset{breaklines=true}
\begin{code}
<ParameterList name="Materials">
  <ParameterList name="My Elastic Plastic Hardening Correspondence Material">
    <Parameter name="Material Model" type="string" value="Isotropic Hardening Correspondence"/>
    <Parameter name="Density" type="double" value="7800.0"/>
    <Parameter name="Young's Modulus" type="double" value="211.0e9"/>
    <Parameter name="Poisson's Ratio" type="double" value="0.0"/>   <!-- One-dimensional simulation -->
    <Parameter name="Yield Stress" type="double" value="460.0e6"/>
    <Parameter name="Hardening Modulus" type="double" value="500.0e7"/>
    <Parameter name="Hourglass Coefficient" type="double" value="0.0"/>
  </ParameterList>
</ParameterList>
\end{code}
\endgroup
%     <Parameter name="Enable Flaw" type="bool" value="false"/>
%     <Parameter name="Flaw Location X" type="double" value="0.0"/>
%     <Parameter name="Flaw Location Y" type="double" value="0.0"/>
%     <Parameter name="Flaw Location Z" type="double" value="0.0"/>
%     <Parameter name="Flaw Size" type="double" value="0.751"/>
%     <Parameter name="Flaw Magnitude" type="double" value="0.2"/>

\levelstay{Free format}

-

\levelstay{YAML format}

-
  
\levelup{Possible output variables for the material model}

\begin{multicols}{2}
\begin{itemize}[noitemsep]
  \item Bond\_Damage
  \item Coordinates
  \item Damage
  \item Deviatoric\_Plastic\_Extension
  \item Dilatation
  \item Force\_Density
  \item Lambda
  \item Model\_Coordinates
  \item Surface\_Correction\_Factor
  \item Volume
  \item Weighted\_Volume
\end{itemize}
\end{multicols}

Additional correspondence material output variables:

\begin{multicols}{2}
\begin{itemize}[noitemsep]
  \item Equivalent\_Plastic\_Strain
  \item Unrotated\_Cauchy\_Stress
  \item Unrotated\_Rate\_Of\_Deformation
  \item Von\_Mises\_Stress
\end{itemize}
\end{multicols}

\levelstay{List of examples}

\begin{itemize}[noitemsep]
%   \item From \texttt{examples/}:
%   \begin{itemize}[noitemsep]
%     \item \texttt{examples/tensile\_test/tensile\_test.peridigm}
%   \end{itemize}
%   \item From \texttt{test/regression/}:
%   \begin{itemize}[noitemsep]
%     \item \texttt{Contact\_Cubes/Contact\_Cubes.xml}
%   \end{itemize}
  \item From \texttt{test/verification/}:
  \begin{itemize}[noitemsep]
    \item \texttt{IsotropicHardeningPlasticFullyPrescribedTension\_NoFlaw/IsotropicHardeningPlasticFullyPrescribedTension\_NoFlaw.xml}
    \item \texttt{IsotropicHardeningPlasticFullyPrescribedTension\_WithFlaw/IsotropicHardeningPlasticFullyPrescribedTension\_WithFlaw.xml}
  \end{itemize}
\end{itemize}