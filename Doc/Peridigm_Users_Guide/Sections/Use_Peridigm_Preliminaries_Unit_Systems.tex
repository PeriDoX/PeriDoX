%%%%%%%%%%%%%%%%%%%%%%%%%%%%%%%%%%%%
% Header                           %
%%%%%%%%%%%%%%%%%%%%%%%%%%%%%%%%%%%%
% 
% Revisions: 2017-04-10 Martin R�del <martin.raedel@dlr.de>
%                       Initial draft
%               
% Contact:   Martin R�del,  martin.raedel@dlr.de
%            DLR Composite Structures and Adaptive Systems
%          
%                                 __/|__
%                                /_/_/_/  
%            www.dlr.de/fa/en      |/ DLR
% 
%%%%%%%%%%%%%%%%%%%%%%%%%%%%%%%%%%%%
% Content                          %
%%%%%%%%%%%%%%%%%%%%%%%%%%%%%%%%%%%%

\levelstay{Unit systems}

For the application of all methods, tools and solutions input values are required. It has to be made sure by the user to make sure that all input values are defined in a single consistent unit system, like the SI- or imperial unit system. The following \autoref{tab:Unit_systems} shows some well-established unit systems.

%%%%%%%%%%%%%%%%%%%%%%%%%%%%%%%%%%%%
% Header                           %
%%%%%%%%%%%%%%%%%%%%%%%%%%%%%%%%%%%%
% 
% Revisions: 2017-04-10 Martin Raedel <martin.raedel@dlr.de>
%                       Initial draft
%               
% Contact:   Martin Raedel,  martin.raedel@dlr.de
%            DLR Composite Structures and Adaptive Systems
%          
%                                 __/|__
%                                /_/_/_/  
%            www.dlr.de/fa/en      |/ DLR
% 
%%%%%%%%%%%%%%%%%%%%%%%%%%%%%%%%%%%%
% Content                          %
%%%%%%%%%%%%%%%%%%%%%%%%%%%%%%%%%%%%

\begin{filecontents}{\tabledir\jobname-unitsystemsltxtable.tex}
\begin{longtable}{@{}Xccccc@{}}
% Caption
\caption{Consistent unit systems according to \protect\marktool{\ansysname} /UNITS command}\\
\label{tab:Unit_systems}\\
% ---------------------------
% Header & Footer
% ---------------------------
%
% Header
% -----------------
% 1st head
\toprule
\multirow{2}{*}{Value}                 & \multicolumn{5}{c}{Unit system}        \\
%         & \multicolumn{5}{c}{Unit system}        \\
& SI        & CGS        & MPA        & BFT        & BIN        \\
\midrule
\endfirsthead
% n-th head
\multicolumn{6}{@{}l}{\ldots continued}\\
\toprule
%\multirow{2}{*}{Value}                 & \multicolumn{5}{c}{Unit system}        \\
& SI        & CGS        & MPA        & BFT        & BIN        \\
\midrule
\endhead
%
% Footer
% -----------------
% n-th foot
\bottomrule
\multicolumn{6}{r@{}}{continued \ldots}\\
\endfoot
% last foot
\bottomrule
\endlastfoot
% ---------------------------
% Content
% ---------------------------
Mass                                    &
$\left[\si{\kilo\gram}\right]$          &
$\left[\si{\gram}\right]$               &
$\left[\si{\tonne}\right]$              &
$\left[\si{slug}\right]$                &
$\frac{\left[\si{lbf}\right]\left[\si{\second}\right]^2}{\left[\si{\inch}\right]}$                  \\[1em]
Length                                  &
$\left[\si{\meter}\right]$              &
$\left[\si{\centi\meter}\right]$        &
$\left[\si{\milli\meter}\right]$        &
$\left[\si{\foot}\right]$               &
$\left[\si{\inch}\right]$               \\[1em]
Time                                    &
$\left[\si{\second}\right]$             &
$\left[\si{\second}\right]$             &
$\left[\si{\second}\right]$             &
$\left[\si{\second}\right]$             &
$\left[\si{\second}\right]$             \\[1em]
Temperature                             &
$\left[\si{\kelvin}\right]$             &
$\left[\si{\kelvin}\right]$             &
$\left[\si{\kelvin}\right]$             &
$\left[\si{\degreeRankine}\right]$      &
$\left[\si{\degreeRankine}\right]$      \\[1em]
Velocity                                                               &
$\frac{\left[\si{\meter}\right]}{\left[\si{\second}\right]}$           &
$\frac{\left[\si{\centi\meter}\right]}{\left[\si{\second}\right]}$     &
$\frac{\left[\si{\milli\meter}\right]}{\left[\si{\second}\right]}$     &
$\frac{\left[\si{\foot}\right]}{\left[\si{\second}\right]}$            &
$\frac{\left[\si{\inch}\right]}{\left[\si{\second}\right]}$            \\[1em]
Acceleration                                                           &
$\frac{\left[\si{\meter}\right]}{\left[\si{\second}\right]^2}$         &
$\frac{\left[\si{\centi\meter}\right]}{\left[\si{\second}\right]^2}$   &
$\frac{\left[\si{\milli\meter}\right]}{\left[\si{\second}\right]^2}$   &
$\frac{\left[\si{\foot}\right]}{\left[\si{\second}\right]^2}$          &
$\frac{\left[\si{\inch}\right]}{\left[\si{\second}\right]^2}$          \\[1em]
Force                                                                  &
$\left[\si{\newton}\right]$                                            &
$\left[\si{\dyn}\right]$                                               &
$\left[\si{\newton}\right]$                                            &
$\left[\si{\poundforce}\right]$                                        &
$\left[\si{\poundforce}\right]$                                        \\[1em]
Moment                                                                 &
$\left[\si{\newton}\right]\left[\si{\meter}\right]$                    &
$\left[\si{\dyn}\right]\left[\si{\centi\meter}\right]$                 &
$\left[\si{\newton}\right]\left[\si{\milli\meter}\right]$              &
$\left[\si{\poundforce}\right]\left[\si{\foot}\right]$                 &
$\left[\si{\poundforce}\right]\left[\si{\inch}\right]$                 \\[1em]
Pressure                                                               &
$\left[\si{\pascal}\right]$                                            & $\left[\si{\barye}\right]$                                             &
$\left[\si{\mega\pascal}\right]$                                       &
$\left[\si{\poundforce}\right]/\left[\si{\foot}\right]^2$              &
$\left[\si{\psi}\right]$                                               \\[1em]
Density                                                                &
$\frac{\left[\si{\kilo\gram}\right]}{\left[\si{\meter}\right]^3}$      &
$\frac{\left[\si{\gram}\right]}{\left[\si{\centi\meter}\right]^3}$     &
$\frac{\left[\si{\tonne}\right]}{\left[\si{\milli\meter}\right]^3}$    &
$\frac{\left[\si{\slug}\right]}{\left[\si{\foot}\right]^3}$            &
$\frac{\left[\si{\poundforce}\right]\left[\si{\second}\right]^2/\left[\si{\inch}\right]}{\left[\si{\inch}\right]^3}$                                           \\[1em]
Energy                                                                 &
$\left[\si{\joule}\right]$                                             &
$\left[\si{\erg}\right]$                                               &
$\left[\si{\milli\joule}\right]$                                       &
$\left[\si{\foot}\right]\left[\si{\poundforce}\right]$                 &
$\left[\si{\inch}\right]\left[\si{\poundforce}\right]$                 \\[1em]
Energy release rate                                                    &
$\frac{\left[\si{\joule}\right]}{\left[\si{\meter}\right]^2}=%
\frac{\left[\si{\newton}\right]}{\left[\si{\meter}\right]}$            &
$\frac{\left[\si{\erg}\right]}{\left[\si{\centi\meter}\right]^2}$      &
$\frac{\left[\si{\milli\joule}\right]}{\left[\si{\milli\meter}\right]^2}=%
\frac{\left[\si{\newton}\right]}{\left[\si{\milli\meter}\right]}$      &
\\
\end{longtable}
\end{filecontents}

\begingroup
\LTXtable{\linewidth}{\tabledir\jobname-unitsystemsltxtable.tex}
\endgroup