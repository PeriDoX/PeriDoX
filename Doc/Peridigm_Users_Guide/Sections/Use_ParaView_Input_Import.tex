%%%%%%%%%%%%%%%%%%%%%%%%%%%%%%%%%%%%
% Header                           %
%%%%%%%%%%%%%%%%%%%%%%%%%%%%%%%%%%%%
% 
% Revisions: 2017-04-10 Martin R�del <martin.raedel@dlr.de>
%                       Initial draft
%               
% Contact:   Martin R�del,  martin.raedel@dlr.de
%            DLR Composite Structures and Adaptive Systems
%          
%                                 __/|__
%                                /_/_/_/  
%            www.dlr.de/fa/en      |/ DLR
% 
%%%%%%%%%%%%%%%%%%%%%%%%%%%%%%%%%%%%
% Content                          %
%%%%%%%%%%%%%%%%%%%%%%%%%%%%%%%%%%%%

\leveldown{Input \& Import}

\marktool{\toolname} creates result files in the \marktool{Exodus} format. This format can read by \marktool{\paraviewname}. It is basically the same format as the \marktool{\exodusname} finite element mesh discretization. Therefore, also the underlying \marktool{\exodusname} mesh can be visualized in \marktool{\paraviewname}.

If multiple processors are used for the execution of \marktool{\toolname} each MPI core creates an individual \marktool{Exodus} output file. The individual result files must be merged before the output can be used in \marktool{\paraviewname}. For the proper procedure please consult section \ref{sec:Peridigm:Run:Execution:Local:MergeOutput}.

To import a file to \marktool{\paraviewname} perform the following steps

\begin{enumerate}[noitemsep]
\item From the menu bar:
  \begin{itemize}[noitemsep]
  \item Click File
  \item Click Open
  \item Select the .g/.e-\marktool{\toolname} input or output file
  \end{itemize}
  or click \includegraphics[width=\iconsize]{Figures/Icons/pqOpen32} in the Main Controls toolbar
\item In the newly opened \textit{Properties} tab:
  \begin{itemize}[noitemsep]
  \item Choose the variables you want to use for post-processing
  \item Choose the blocks, assemblies and material regions you want to include
  \item Click \textit{Apply}
  \end{itemize}
\end{enumerate}