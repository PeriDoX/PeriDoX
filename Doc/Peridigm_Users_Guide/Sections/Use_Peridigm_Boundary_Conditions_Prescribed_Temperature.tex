%%%%%%%%%%%%%%%%%%%%%%%%%%%%%%%%%%%%
% Header                           %
%%%%%%%%%%%%%%%%%%%%%%%%%%%%%%%%%%%%
% 
% Revisions: 2017-04-10 Martin R�del <martin.raedel@dlr.de>
%                       Initial draft
%               
% Contact:   Martin R�del,  martin.raedel@dlr.de
%            DLR Composite Structures and Adaptive Systems
%          
%                                 __/|__
%                                /_/_/_/  
%            www.dlr.de/fa/en      |/ DLR
% 
%%%%%%%%%%%%%%%%%%%%%%%%%%%%%%%%%%%%
% Content                          %
%%%%%%%%%%%%%%%%%%%%%%%%%%%%%%%%%%%%

\levelup{\idxPDKwPrescribedTemperature}
\label{sec:QRG:BoundaryConditions:PrescribedTemperature}
\myindex[\idxPDKeywordName]{\idxPDKwPrescribedTemperature}
\myindex[\idxPDKeywordName]{\idxPDKwBoundaryConditions!\idxPDKwPrescribedTemperature|see{\idxPDKwPrescribedTemperature}}

\leveldown{Description}

Definition of a prescribed temperature on a nodeset. The prescribed temperature might be dependent of the scalar coordinates of the nodeset members and/or the time.

\levelstay{Code}

\begin{itemize}[noitemsep]
  \item from \texttt{src/core/}:
  \begin{itemize}[noitemsep]
    \item \texttt{Peridigm\_BoundaryAndInitialConditionManager.cpp}
    \item \texttt{Peridigm\_BoundaryAndInitialConditionManager.hpp}
    \item \texttt{Peridigm\_BoundaryCondition.cpp}
    \item \texttt{Peridigm\_BoundaryCondition.hpp}
    \item \texttt{Peridigm\_Enums.cpp}
    \item \texttt{Peridigm\_Enums.hpp}
  \end{itemize}
\end{itemize}

\levelstay{Input parameters}

\paragraph{List}

\begin{tabularx}{\linewidth}{lcccX}
\toprule
Name		& Type		& Required	& Default	& Description		\\
\midrule
Type		& string	& \checkmark	& -		& ``Prescribed Temperature''	\\
Node Set        & string        & \checkmark\textsuperscript{\ref{enm:Peridigm:QRG:BoundaryConditions:PrescribedTemperature:Remark:NodeSet}}    & -             & Application region name | ``Full Domain'' | ``All Sets''      \\
% Coordinate	& string	& \checkmark	& -		& ``x'' | ``y'' | ``z'' 	\\
Value		& string	& \checkmark	& -		& String with function for function parser	\\
\bottomrule
\end{tabularx}

\paragraph{Remarks}

\begin{enumerate}[noitemsep]
  \item \label{enm:Peridigm:QRG:BoundaryConditions:PrescribedTemperature:Remark:NodeSet} Any string that is not ``Full Domain'' or ``All Sets'' will be considered a custom node set name in the model or mesh file, dependent of the discretization type.
  \item \label{enm:Peridigm:QRG:BoundaryConditions:PrescribedTemperature:Remark:Value} The string in the variable \textit{Value} should start with \verb+value = +. If it does not, \toolname{} will automatically add it for the function parser to work.
  \item This boundary conditions might not work with some of the material models in the current implementation.
\end{enumerate}

\levelstay{Exemplary input section}

\paragraph{XML format}

\begingroup
\lstset{breaklines=true}
\begin{code}
<ParameterList name="Boundary Conditions">
  <ParameterList name="Prescribed Thermal Loading">
    <Parameter name="Type" type="string" value="Prescribed Temperature"/>
    <Parameter name="Node Set" type="string" value="FULL_domain"/>
    <Parameter name="Value" type="string" value="10000.0 - 10000.0*0.5*(cos(3.14159265359 + t*3.14159265359/0.1) + 1.0)"/>
  </ParameterList>
</ParameterList>
\end{code}
\endgroup

\paragraph{Free format}

-

\paragraph{YAML format}

-

\levelstay{List of examples}

\begin{itemize}[noitemsep]
%   \item From \texttt{Models/}:
%   \begin{itemize}[noitemsep]
%     \item \texttt{Models/Dogbone}
%   \end{itemize}
%   \item From \texttt{examples/}:
%   \begin{itemize}[noitemsep]
%     \item \texttt{twist\_and\_pull/twist\_and\_pull.peridigm}
%   \end{itemize}
%   \item From \texttt{test/regression/}:
%   \begin{itemize}[noitemsep]
%     \item \texttt{Body\_Force/Body\_Force\_Implicit.xml}
%   \end{itemize}
  \item From \texttt{test/verification/}:
  \begin{itemize}[noitemsep]
    \item \texttt{ThermalExpansionBondFailure/ThermalExpansionBondFailure.xml}
    \item \texttt{ThermalExpansionBondFailure/ThermalExpansionCube.xml}
  \end{itemize}
\end{itemize}