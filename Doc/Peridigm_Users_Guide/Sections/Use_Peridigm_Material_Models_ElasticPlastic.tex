%%%%%%%%%%%%%%%%%%%%%%%%%%%%%%%%%%%%
% Header                           %
%%%%%%%%%%%%%%%%%%%%%%%%%%%%%%%%%%%%
% 
% Revisions: 2017-04-10 Martin R�del <martin.raedel@dlr.de>
%                       Initial draft
%               
% Contact:   Martin R�del,  martin.raedel@dlr.de
%            DLR Composite Structures and Adaptive Systems
%          
%                                 __/|__
%                                /_/_/_/  
%            www.dlr.de/fa/en      |/ DLR
% 
%%%%%%%%%%%%%%%%%%%%%%%%%%%%%%%%%%%%
% Content                          %
%%%%%%%%%%%%%%%%%%%%%%%%%%%%%%%%%%%%

\levelup{\idxPDKwElasticPlastic}
\label{sec:Peridigm:QRG:Materials:ElasticPlastic}
\myindex[\idxPDKeywordName]{\idxPDKwElasticPlastic}
\myindex[\idxPDKeywordName]{\idxPDKwMaterials!\idxPDKwElasticPlastic|see{\idxPDKwElasticPlastic}}

\leveldown{Description}

An isotropic, linear elastic perfectly plastic material model.

\levelstay{Literature}

\begin{itemize}[noitemsep]
  \item \cite{MitchellJA2011}
\end{itemize}

\levelstay{Stiffness model sketch}

\begin{figure}[htbp]
  \begin{subfigure}{0.49\linewidth}
    \centering
    \tikzexternalenable
    \tikzsetnextfilename{Material_Model_ElasticPlastic-Linear-Perfect-K}
    \begin{tikzpicture}
  % Variable
  \def\modulus{70000}
  \def\hardeningmodulus{0}
  \def\yieldstress{350}
  \def\failstrain{0.025}
  \def\unloadingfac{0.8}
  \def\xLabel{$\glssymbol{symb:scalar:mech:strain:normal:engineering}$}
  \def\yLabel{$\glssymbol{symb:scalar:mech:stress:normal:engineering}$}
  %\def\pinELabel{$K$}
  %\def\pinEHLabel{$K_H=0$}
  \def\pinELabel{$\glssymbol{symb:scalar:mat:modulus:bulk}$}
  \def\pinEHLabel{$\glssymbol{symb:scalar:mat:modulus:bulk}_{\glssymbol{symb:index:hardening}}=0$}
  \def\pinEPos{0.5}
  \def\pinEHPos{0.6}
  %\def\yieldlabel{$\sigma_y$}
  \def\yieldlabel{$\glssymbol{symb:scalar:mech:stress:normal:engineering}_{\glssymbol{symb:index:yield}}$}
  \newtoggle{tclabel}
  \toggletrue{tclabel}
    % Math
  \pgfkeys{/pgf/fpu}
  \pgfmathsetmacro{\yieldstrain}{\yieldstress/\modulus}
  \pgfmathsetmacro{\failstress}{\yieldstress+\hardeningmodulus*(\failstrain-\yieldstrain)}
  \pgfmathsetmacro{\unloadingstress}{\yieldstress+\unloadingfac*\hardeningmodulus*(\failstrain-\yieldstrain)}
  \pgfmathsetmacro{\unloadingstrain}{\yieldstrain+\unloadingfac*(\failstrain-\yieldstrain)}
  \pgfkeys{/pgf/fpu=false}
  % Shapes
  \tikzset{%
    myarrowdecoration1/.style={postaction={decorate,decoration={
      markings,
      mark=between positions .4 and .6 step .1pt with {\draw [thin] circle (.1pt);},
      mark=at position .6 with {\arrow[thin,xshift=1pt]{latex}},
      raise=-0.7ex,
    }}},
    myarrowdecoration2/.style={postaction={decorate,decoration={
      markings,
      mark=between positions .4 and .6 step .1pt with {\draw [thin] circle (.1pt);},
      mark=at position .4 with {\arrow[thin,xshift=1pt]{latex reversed}},
      raise=0.7ex,
    }}},
  }
  % Axis
  \begin{axis}[
%     scale only axis,
    axis lines=middle,
    ticks=none,
    %restrict y to domain=-\ultimatestrength:\ultimatestrength,
    xmin=-1.1*\failstrain,
    xmax= 1.1*\failstrain,
    ymin=-1.5*\yieldstress,
    ymax= 1.5*\yieldstress,
    width=0.99\textwidth,
    height=0.99\textwidth,
    xlabel=\xLabel,
    ylabel=\yLabel,
    every axis x label/.style={
      at={(ticklabel* cs:1.005)},
      anchor=west,
    },
    every axis y label/.style={
      at={(ticklabel* cs:1.005)},
      anchor=south,
    },
  ]
    % Coordinates
    \coordinate (origin)     at (0,0);
    \coordinate (yieldt)     at ( \yieldstrain, \yieldstress);
    \coordinate (yieldc)     at (-\yieldstrain,-\yieldstress);
    \coordinate (failuret)   at ( \failstrain, \failstress);
    \coordinate (failurec)   at (-\failstrain,-\failstress);
    % Lines
    \draw[thick,draw=\plotcolor] (origin) -- node[pos=\pinEPos, pin={[pin distance=1ex]-60:{\pinELabel}}](ELabel){} (yieldt);
    \draw[thick,draw=\plotcolor] (yieldt) -- node[pos=\pinEHPos, pin={[pin distance=1ex]120:{\pinEHLabel}}](EHLabel){} (failuret);
    \draw[thick,draw=\plotcolor] (origin) -- (yieldc);
    \draw[thick,draw=\plotcolor] (yieldc) -- (failurec);
    
    \draw[dashed] (origin|-yieldt) node[anchor=east]{\yieldlabel} -- (yieldt);
    % Loading/Unloading
    \draw[dashed,myarrowdecoration1,myarrowdecoration2] (\unloadingstrain,\unloadingstress) -- (\unloadingstrain-\yieldstrain,0);
    % Label
    \iftoggle{tclabel}{%
      \node[anchor=north east] (tensionlabel) at (rel axis cs:1,1) {\footnotesize tension};
      \node[anchor=south west] (compressionlabel) at (rel axis cs:0,0) {\footnotesize compression};
    }{}
  \end{axis}
\end{tikzpicture}
    \tikzexternaldisable
    \caption{Axial}
    \label{fig:Material_Models_LinearElasticPerfectlyPlastic-K}
  \end{subfigure}%
  \begin{subfigure}{0.49\linewidth}
    \centering
    \tikzexternalenable
    \tikzsetnextfilename{Material_Model_ElasticPlastic-Linear-Perfect-G}
    \begin{tikzpicture}
  % Variable
  \def\modulus{70000}
  \def\hardeningmodulus{0}
  \def\yieldstress{350}
  \def\failstrain{0.025}
  \def\unloadingfac{0.8}
  \def\xLabel{$\glssymbol{symb:scalar:mech:strain:shear:engineering}$}
  \def\yLabel{$\glssymbol{symb:scalar:mech:stress:shear:engineering}$}
  \def\pinELabel{$\glssymbol{symb:scalar:mat:modulus:shear}$}
  \def\pinEHLabel{$\glssymbol{symb:scalar:mat:modulus:shear}_{\glssymbol{symb:index:hardening}}=0$}
  \def\pinEPos{0.5}
  \def\pinEHPos{0.6}
  \def\yieldlabel{$\glssymbol{symb:scalar:mech:stress:shear:engineering}\left(\glssymbol{symb:scalar:mech:stress:normal:engineering}_{\glssymbol{symb:index:yield}}\right)$}
  \newtoggle{tclabel}
  \togglefalse{tclabel}
    % Math
  \pgfkeys{/pgf/fpu}
  \pgfmathsetmacro{\yieldstrain}{\yieldstress/\modulus}
  \pgfmathsetmacro{\failstress}{\yieldstress+\hardeningmodulus*(\failstrain-\yieldstrain)}
  \pgfmathsetmacro{\unloadingstress}{\yieldstress+\unloadingfac*\hardeningmodulus*(\failstrain-\yieldstrain)}
  \pgfmathsetmacro{\unloadingstrain}{\yieldstrain+\unloadingfac*(\failstrain-\yieldstrain)}
  \pgfkeys{/pgf/fpu=false}
  % Shapes
  \tikzset{%
    myarrowdecoration1/.style={postaction={decorate,decoration={
      markings,
      mark=between positions .4 and .6 step .1pt with {\draw [thin] circle (.1pt);},
      mark=at position .6 with {\arrow[thin,xshift=1pt]{latex}},
      raise=-0.7ex,
    }}},
    myarrowdecoration2/.style={postaction={decorate,decoration={
      markings,
      mark=between positions .4 and .6 step .1pt with {\draw [thin] circle (.1pt);},
      mark=at position .4 with {\arrow[thin,xshift=1pt]{latex reversed}},
      raise=0.7ex,
    }}},
  }
  % Axis
  \begin{axis}[
%     scale only axis,
    axis lines=middle,
    ticks=none,
    %restrict y to domain=-\ultimatestrength:\ultimatestrength,
    xmin=-1.1*\failstrain,
    xmax= 1.1*\failstrain,
    ymin=-1.5*\yieldstress,
    ymax= 1.5*\yieldstress,
    width=0.99\textwidth,
    height=0.99\textwidth,
    xlabel=\xLabel,
    ylabel=\yLabel,
    every axis x label/.style={
      at={(ticklabel* cs:1.005)},
      anchor=west,
    },
    every axis y label/.style={
      at={(ticklabel* cs:1.005)},
      anchor=south,
    },
  ]
    % Coordinates
    \coordinate (origin)     at (0,0);
    \coordinate (yieldt)     at ( \yieldstrain, \yieldstress);
    \coordinate (yieldc)     at (-\yieldstrain,-\yieldstress);
    \coordinate (failuret)   at ( \failstrain, \failstress);
    \coordinate (failurec)   at (-\failstrain,-\failstress);
    % Lines
    \draw[thick,draw=\plotcolor] (origin) -- node[pos=\pinEPos, pin={[pin distance=1ex]-60:{\pinELabel}}](ELabel){} (yieldt);
    \draw[thick,draw=\plotcolor] (yieldt) -- node[pos=\pinEHPos, pin={[pin distance=1ex]120:{\pinEHLabel}}](EHLabel){} (failuret);
    \draw[thick,draw=\plotcolor] (origin) -- (yieldc);
    \draw[thick,draw=\plotcolor] (yieldc) -- (failurec);
    
    \draw[dashed] (origin|-yieldt) node[anchor=east]{\yieldlabel} -- (yieldt);
    % Loading/Unloading
    \draw[dashed,myarrowdecoration1,myarrowdecoration2] (\unloadingstrain,\unloadingstress) -- (\unloadingstrain-\yieldstrain,0);
    % Label
    \iftoggle{tclabel}{%
      \node[anchor=north east] (tensionlabel) at (rel axis cs:1,1) {\footnotesize tension};
      \node[anchor=south west] (compressionlabel) at (rel axis cs:0,0) {\footnotesize compression};
    }{}
  \end{axis}
\end{tikzpicture}
    \tikzexternaldisable
    \caption{Shear}
    \label{fig:Material_Models_LinearElasticPerfectlyPlastic-G}
  \end{subfigure}%
  \caption{Linear-elastic perfectly-plastic material model}
  \label{fig:Material_Models_LinearElasticPerfectlyPlastic}
\end{figure}

\levelstay{Code}

\leveldown{Release version}

Available from \href{\toolrepoversiononetwo}{version 1.2}.

\levelstay{Required compiler options}

-

\levelstay{Routines}

\begin{itemize}[noitemsep]
  \item IO:
  \begin{itemize}[noitemsep]
  \item \verb+/src/materials/Peridigm_ElasticPlasticMaterial.cpp+
  \item \verb+/src/materials/Peridigm_ElasticPlasticMaterial.hpp+
  \end{itemize}
  \item Computation:
  \begin{itemize}[noitemsep]
    \item \verb+/src/materials/elastic_plastic.cxx+
    \item \verb+/src/materials/elastic_plastic.h+
  \end{itemize}
\end{itemize}

\levelup{Input parameters}

\leveldown{List}

% \begin{tabularx}{\linewidth}{lcccX}
% \toprule
% Name            & Type          & Required      & Default       & Description           \\
% \midrule
\begin{filecontents}{\tabledir\jobname-parammatelasticplasticltxtable.tex}
% \begin{longtable}{@{}lX@{}}
\begin{longtable}{@{}lcccX@{}}
% ---------------------------
% Header & Footer
% ---------------------------
%
% Header
% -----------------
% 1st head
\toprule
Name          & Type          & Required      & Default       & Description           \\
\midrule
\endfirsthead
% Last head
\multicolumn{5}{@{}l}{\ldots continued}\\
\toprule
Name          & Type          & Required      & Default       & Description           \\
\midrule
\endhead
%
% Footer
% -----------------
% n-th foot
\bottomrule
\multicolumn{5}{r@{}}{continued \ldots}\\
\endfoot
% last foot
\bottomrule
\endlastfoot
% ---------------------------
% Content
% ---------------------------
Material Model & string & \checkmark & -  & Material type ``\idxPDKwElasticPlastic''  \\
Density        & double & \checkmark & -  & Material density \\
Bulk modulus   & double & \checkmark\textsuperscript{\ref{enm:Peridigm:QRG:Materials:ElasticPlastic:Remark:Modulus:One},\ref{enm:Peridigm:QRG:Materials:ElasticPlastic:Remark:Modulus:Two}} & -  & Volumetric elasticity\\
Shear Modulus  & double & \checkmark\textsuperscript{\ref{enm:Peridigm:QRG:Materials:ElasticPlastic:Remark:Modulus:One},\ref{enm:Peridigm:QRG:Materials:ElasticPlastic:Remark:Modulus:Two}} & -  & Shear elasticity or engineering constant for shear stiffness\\
Young\verb+'+s Modulus & double & \checkmark\textsuperscript{\ref{enm:Peridigm:QRG:Materials:ElasticPlastic:Remark:Modulus:One},\ref{enm:Peridigm:QRG:Materials:ElasticPlastic:Remark:Modulus:Two}} & -  & Engineering constant for axial stiffness  \\
Poisson\verb+'+s Ratio & double & \checkmark\textsuperscript{\ref{enm:Peridigm:QRG:Materials:ElasticPlastic:Remark:Modulus:One},\ref{enm:Peridigm:QRG:Materials:ElasticPlastic:Remark:Modulus:Two}} & -  & Engineering constant for transverse contraction\\
Yield Stress   & double & \checkmark & -  & Yield stress  \\
Apply Shear Correction Factor & bool  & -  & true  &    \\
% Apply Automatic Differentiation Jacobian& bool & - & ? &   \\
Disable Plasticity & bool  & -  & false  & \\
\end{longtable}
\end{filecontents}
% \bottomrule
% \end{tabularx}

\begingroup
\LTXtable{\linewidth}{\tabledir\jobname-parammatelasticplasticltxtable.tex}
\endgroup

\levelstay{Remarks}

\begin{enumerate}[noitemsep]
  \item \label{enm:Peridigm:QRG:Materials:ElasticPlastic:Remark:Modulus:One} The stiffness can be defined by either elastic modulus combination: Volumetric and shear elasticity ($\glssymbol{symb:scalar:mat:modulus:bulk}$,$\glssymbol{symb:scalar:mat:modulus:shear}$) or the engineering constants ($\glssymbol{symb:scalar:mat:modulus:young}$,$\glssymbol{symb:scalar:mat:poissonratio}$,$\glssymbol{symb:scalar:mat:modulus:shear}$)
  \item \label{enm:Peridigm:QRG:Materials:ElasticPlastic:Remark:Modulus:Two} In case engineering constants are used, only two of the three values \textit{Young's Modulus}, \textit{Poisson's Ratio} and \textit{Shear Modulus} have to be specified. The missing value is calculated from
  \begin{align*}
  \glssymbol{symb:scalar:mat:modulus:shear}&=\dfrac{\glssymbol{symb:scalar:mat:modulus:young}}{2\cdot\left(1+\glssymbol{symb:scalar:mat:poissonratio}\right)}
  \end{align*}
  Internally, the engineering constants are converted to ($\glssymbol{symb:scalar:mat:modulus:bulk}$,$\glssymbol{symb:scalar:mat:modulus:shear}$).
  \item Yield stress is an equivalent uniaxial stress and therefore applicable to axial, shear as well as mixed stress states.
  %\item Automatic Differentiation is not supported for the ElasticCorrespondence material model
  %\item Shear Correction Factor is not supported for the ElasticCorrespondence material model
  \item Thermal expansion is not currently supported for the Elastic Plastic material model
  \item Consider the general remarks on non-correspondence materials in \autoref{sec:Peridigm:QRG:Materials:Preliminaries:NonCorrespondence}
\end{enumerate}

\levelup{Exemplary input section}

\leveldown{XML-format}$\mbox{ }$\\

From \texttt{ep\_cube.xml}:

\begingroup
\lstset{breaklines=true}
\begin{code}
<ParameterList name="Materials">
  <ParameterList name="Elastic Plastic">
    <Parameter name="Density" type="double" value="2.7e-3"/>
    <Parameter name="Bulk Modulus" type="double" value="67549.0"/>
    <Parameter name="Shear Modulus" type="double" value="25902.2"/>
    <Parameter name="Yield Stress" type="double" value="351.79"/>
    <Parameter name="Apply Shear Correction Factor" type="bool" value="false"/>
  </ParameterList>
</ParameterList>
\end{code}
\endgroup

\begingroup
\lstset{breaklines=true}
\begin{code}
<ParameterList name="RightMaterial">
  <Parameter name="Material Model" type="string" value="Elastic Plastic"/>
  <Parameter name="Density" type="double" value="8.0e-9"/>
  <Parameter name="Bulk Modulus" type="double" value="1.515e5"/>
  <Parameter name="Shear Modulus" type="double" value="7.813e4"/>
  <Parameter name="Yield Stress" type="double" value="1.0e5"/>
  <Parameter name="Disable Plasticity" type="bool" value="true"/>
  <Parameter name="Apply Shear Correction Factor" type="bool" value="false"/>
</ParameterList>
\end{code}
\endgroup

\levelstay{Free format}

-

\levelstay{YAML format}

-
  
\levelup{Possible output variables for the material model}

\begin{multicols}{2}
\begin{itemize}[noitemsep]
  \item Bond\_Damage
  \item Coordinates
  \item Damage
  \item Deviatoric\_Plastic\_Extension
  \item Dilatation
  \item Force\_Density
  \item Lambda
  \item Model\_Coordinates
  \item Volume
  \item Weighted\_Volume
\end{itemize}
\end{multicols}

\levelstay{List of examples}

\begin{itemize}[noitemsep]
%   \item From \texttt{examples/}:
%   \begin{itemize}[noitemsep]
%     \item \texttt{examples/tensile\_test/tensile\_test.peridigm}
%   \end{itemize}
  \item From \texttt{test/regression/}:
  \begin{itemize}[noitemsep]
    \item \texttt{Bar\_TwoBlocks\_TwoDifferentMaterial\_QS/Bar.xml}
  \end{itemize}
  \item From \texttt{test/verification/}:
  \begin{itemize}[noitemsep]
    \item \texttt{ep\_cube/ep\_cube.xml}
  \end{itemize}
\end{itemize}