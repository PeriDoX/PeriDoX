%%%%%%%%%%%%%%%%%%%%%%%%%%%%%%%%%%%%
% Header                           %
%%%%%%%%%%%%%%%%%%%%%%%%%%%%%%%%%%%%
%
% Revisions: 2017-04-10 Martin R�del <martin.raedel@dlr.de>
%                       Initial draft
%
% Contact:   Martin R�del,  martin.raedel@dlr.de
%            DLR Composite Structures and Adaptive Systems
%
%                                 __/|__
%                                /_/_/_/
%            www.dlr.de/fa/en      |/ DLR
%
%%%%%%%%%%%%%%%%%%%%%%%%%%%%%%%%%%%%
% Content                          %
%%%%%%%%%%%%%%%%%%%%%%%%%%%%%%%%%%%%

\levelup{\idxPDKwInfluenceFunction}
\label{sec:Peridigm:QRG:Discretization:Tools:InfluenceFunction}
\myindex[\idxPDKeywordName]{\idxPDKwInfluenceFunction}

\leveldown{Description}

The influence function in the peridynamic theory is used to weight the contribution of all the bonds participating in the computation of volume-dependent properties dependent of the distance to the family origin.

\levelstay{Literature}

\begin{itemize}[noitemsep]
  \item \cite{SelesonP2011}
  \item \cite{TupekMR2013}
\end{itemize}

\levelstay{Code}

\leveldown{Release version}

Available from \href{\toolrepoversiononetwo}{version 1.2}.

\levelstay{Required compiler options}

-

\levelstay{Routines}

\begin{itemize}[noitemsep]
  \item from \texttt{src/core/}:
  \begin{itemize}[noitemsep]
    \item \texttt{Peridigm\_Peridigm\_InfluenceFunction.cpp}
    \item \texttt{Peridigm\_Peridigm\_InfluenceFunction.hpp}
  \end{itemize}
\end{itemize}

% Each class uses the method as:
% \begin{itemize}[noitemsep]
% \item \verb+PeridigmNS::InfluenceFunction::functionPointer m_OMEGA;+
% 
% The function value can be obtained by
% \item \verb+omega = m_OMEGA(initialDistance, m_horizon);+
% \end{itemize}

\levelup{Input parameters}

\leveldown{List}

The influence function is not an individual object. Therefore, no parameter list is required here.

\levelstay{Remarks}

\begin{enumerate}[noitemsep]
  \item The influence function is defined for the whole model.
  \item Function values are dependent of the bond length $\lVert\glssymbol{symb:vector:pd:bond:undeformed}\rVert$ and the local horizon $\glssymbol{symb:scalar:pd:horizon}$
  \item The influence functions in \autoref{fig:Peridigm:QRG:Discretization:Common:InfluenceFunctions} are pre-implemented. Each unrecognized string is assumed to be an user-defined influence function. The default value is set to \textit{\idxPDKwOne} in case the keyword is not specified, see \texttt{Peridigm.cpp}.
    
  \begin{figure}[htbp]
    \def\figxlabel{$\frac{\lVert\glssymbol{symb:vector:pd:bond:undeformed}\rvert|}{\glssymbol{symb:scalar:pd:horizon}}$}
    \def\figylabel{$\glssymbol{symb:scalar:pd:function:influence:radial}\left(\glssymbol{symb:vector:pd:bond:undeformed}\right)$}
    %\def\figwidth{\linewidth}
    \setlength{\figwidth}{0.33\linewidth}
    \setlength{\figheight}{\figwidth}
    
    \begin{subfigure}{\figwidth}
      \centering
      \tikzexternalenable
      \tikzsetnextfilename{Discretization_InfluenceFunction_One}
      %%%%%%%%%%%%%%%%%%%%%%%%%%%%%%%%%%%%
% Header                           %
%%%%%%%%%%%%%%%%%%%%%%%%%%%%%%%%%%%%
% 
% Revisions: 2017-04-10 Martin R�del <martin.raedel@dlr.de>
%                       Initial draft
%               
% Contact:   Martin R�del,  martin.raedel@dlr.de
%            DLR Composite Structures and Adaptive Systems
%          
%                                 __/|__
%                                /_/_/_/  
%            www.dlr.de/fa/en      |/ DLR
% 
%%%%%%%%%%%%%%%%%%%%%%%%%%%%%%%%%%%%
% Content                          %
%%%%%%%%%%%%%%%%%%%%%%%%%%%%%%%%%%%%

\begin{tikzpicture}
  \begin{axis}[influencefunctionaxis style]
    \addplot[\plotcolor]{1.0};
  \end{axis}
\end{tikzpicture}
      \tikzexternaldisable
      \caption{\idxPDKwOne}
      \label{fig:Peridigm:QRG:Discretization:Common:InfluenceFunction:One}
    \end{subfigure}%
    \hfill
    \begin{subfigure}{\figwidth}
      \centering
      \tikzexternalenable
      \tikzsetnextfilename{Discretization_InfluenceFunction_ParabolicDecay}
      %%%%%%%%%%%%%%%%%%%%%%%%%%%%%%%%%%%%
% Header                           %
%%%%%%%%%%%%%%%%%%%%%%%%%%%%%%%%%%%%
% 
% Revisions: 2017-04-10 Martin R�del <martin.raedel@dlr.de>
%                       Initial draft
%               
% Contact:   Martin R�del,  martin.raedel@dlr.de
%            DLR Composite Structures and Adaptive Systems
%          
%                                 __/|__
%                                /_/_/_/  
%            www.dlr.de/fa/en      |/ DLR
% 
%%%%%%%%%%%%%%%%%%%%%%%%%%%%%%%%%%%%
% Content                          %
%%%%%%%%%%%%%%%%%%%%%%%%%%%%%%%%%%%%

\begin{tikzpicture}[
  declare function={
    parabolicdecay(\x) = (\x<=0.5) * (1.0)   +
                          (\x> 0.5) * (-4.0*\x*\x+4.0*\x);
  }
]
  \begin{axis}[influencefunctionaxis style]
    \addplot[\plotcolor]{parabolicdecay(x)};
  \end{axis}
\end{tikzpicture}
      \tikzexternaldisable
      \caption{\idxPDKwParabolicDecay}
      \label{fig:Peridigm:QRG:Discretization:Common:InfluenceFunction:ParabolicDecay}
    \end{subfigure}%
    \hfill
    \begin{subfigure}{\figwidth}
  %       \def\legendentrywithciteexpansion{\cite{SelesonP2014b}}
      \centering
  %       \tikzexternalenable
  %       \tikzsetnextfilename{Discretization_InfluenceFunction_Gaussian}
      %%%%%%%%%%%%%%%%%%%%%%%%%%%%%%%%%%%%
% Header                           %
%%%%%%%%%%%%%%%%%%%%%%%%%%%%%%%%%%%%
% 
% Revisions: 2017-04-10 Martin R�del <martin.raedel@dlr.de>
%                       Initial draft
%               
% Contact:   Martin R�del,  martin.raedel@dlr.de
%            DLR Composite Structures and Adaptive Systems
%          
%                                 __/|__
%                                /_/_/_/  
%            www.dlr.de/fa/en      |/ DLR
% 
%%%%%%%%%%%%%%%%%%%%%%%%%%%%%%%%%%%%
% Content                          %
%%%%%%%%%%%%%%%%%%%%%%%%%%%%%%%%%%%%

\begin{tikzpicture}
  \begin{axis}[
    influencefunctionaxis style,
    legend pos=south west,
  ]
    \addplot[\plotcolor]{e^(-x*x/(0.4*0.4))};
    \addlegendentry{\cite{SelesonP2014b}}
    \addplot[black,dashed]{e^(-x*x)};
    \addlegendentry{\cite{TupekMR2013}}
  \end{axis}
\end{tikzpicture}
  %       \tikzexternaldisable
      \caption{\idxPDKwGaussian}
      \label{fig:Peridigm:QRG:Discretization:Common:InfluenceFunction:Gaussian}
    \end{subfigure}%
    \caption{Influence functions}
    \label{fig:Peridigm:QRG:Discretization:Common:InfluenceFunctions}
  \end{figure}
  \item For further information on the effect of the influence function see: \cite{SelesonP2011} \& \cite{LittlewoodD2015}.
  \item Other Influence functions and their usage can be found in \cite{SelesonP2014b}.  The ``\idxPDKwParabolicDecay'' is used in \cite{LittlewoodD2013b}. Several different functions can be found in the literature for the ``\idxPDKwGaussian'' influence function:
  \begin{itemize}%[noitemsep]
    \item \cite{TupekMR2013,KilicB2009c}: \tab$\glssymbol{symb:scalar:pd:function:influence:radial}\left(\glssymbol{symb:vector:pd:bond:undeformed}\right)=e^{-\left(\frac{\lVert\glssymbol{symb:vector:pd:bond:undeformed}\rVert}{\glssymbol{symb:scalar:pd:horizon}}\right)^2}$
    \item \cite{SelesonP2014b}: \tab$\glssymbol{symb:scalar:pd:function:influence:radial}\left(\glssymbol{symb:vector:pd:bond:undeformed}\right)=e^{-\left(\frac{\lVert\glssymbol{symb:vector:pd:bond:undeformed}\rVert}{\num{0.4}\cdot\glssymbol{symb:scalar:pd:horizon}}\right)^2}$
  \end{itemize}
  Here, the version from \cite{SelesonP2014b} is used\footnote{see \href{https://github.com/peridigm/peridigm/issues/42}{https://github.com/peridigm/peridigm/issues/42}}.
\end{enumerate}

\levelup{Exemplary input section}

\leveldown{XML-format}

\begingroup
\lstset{breaklines=true}
\begin{code}
<ParameterList name="Discretization">
  <Parameter name="Type" type="string" value="Exodus" />
  <Parameter name="Input Mesh File" type="string" value="Plate.g"/>
  <Parameter name="Influence Function" type="string" value="Gaussian"/>
</ParameterList>
\end{code}
\endgroup

\levelstay{Free format}

-

\levelstay{YAML format}

-

\levelup{List of examples}

-
