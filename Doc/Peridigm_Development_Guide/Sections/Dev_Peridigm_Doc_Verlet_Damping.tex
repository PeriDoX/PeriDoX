%%%%%%%%%%%%%%%%%%%%%%%%%%%%%%%%%%%%
% Header                           %
%%%%%%%%%%%%%%%%%%%%%%%%%%%%%%%%%%%%
% 
% Revisions: 2017-04-10 Martin Rädel <martin.raedel@dlr.de>
%                       Initial draft
%               
% Contact:   Christian Willberg,  christian.willberg@dlr.de
%            DLR Composite Structures and Adaptive Systems
%          
%                                 __/|__
%                                /_/_/_/  
%            www.dlr.de/fa/en      |/ DLR
% 
%%%%%%%%%%%%%%%%%%%%%%%%%%%%%%%%%%%%
% Content                          %
%%%%%%%%%%%%%%%%%%%%%%%%%%%%%%%%%%%%
\section{Implemented time integration methods}
\subsection{Numerical damping added to Velocity-Verlet}
\subsubsection{Theory}

  The theory for the Velocity-Verlet solver with damping is derived in the following. The equations to compute position and velocity at 
  the next time step without damping are given by
  \begin{equation}
  \begin{aligned}
   x_{n+1} \, &= \, x_n \, + \, \Delta t \, v_n \, + \, \dfrac{1}{2} \, \Delta t^2 \, a_n \\
   v_{n+1} \, &= \, v_n \, + \, \dfrac{1}{2} \, \Delta t \, (\, a_{n+1} \, + \, a_n \, ) ~.
  \end{aligned}
  \end{equation}
  $x$ denotes the position, $\Delta t$ the time increment, $v$ the velocity, $a$ the acceleration and subscript $n$ the time step number.
  The Velocity-Verlet algorithm with damping introduces an additional parameter $\gamma$ connected to the damping term and the velocity can be described by
  \begin{equation}
   v_{n} \, = \, \dfrac{x_{n+1} \, - \, x_{n-1}}{2 \, \Delta t} ~.
  \label{velointerpolation}
  \end{equation}
  Starting now with the Taylor expansion series
  \begin{equation}
  \begin{aligned}
   x_{n+1} \, &= \, x_n + \Delta t \, v_n \, + \, \dfrac{1}{2} \Delta t^2 \, a_n \, + \, \dfrac{1}{6} \Delta t^3 \dot{a}_n + O(\Delta t^4 ) \\
   x_{n-1} \, &= \, x_n - \Delta t \, v_n \, + \, \dfrac{1}{2} \Delta t^2 \, a_n \, - \, \dfrac{1}{6} \Delta t^3 \dot{a}_n + O(\Delta t^4 ) 
  \end{aligned}
  \end{equation}
  and adding both equations gives the Verlet algorithm
  \begin{equation}
   x_{n+1} \, = \, 2 \, x_n \, - \, x_{n-1} \, + \, \Delta t^2 a_n + O(\Delta t^4 ) ~.
  \end{equation}
  Expressing the accerlation in terms of force $F$ and mass $m$ allows to introduce a damping force $G$.
  \begin{equation}
   x_{n+1} \, = \, 2 \, x_n \, - \, x_{n-1} \, + \, \dfrac{\Delta t^2}{m} (\, F_n \, - \, G_n \, ) \, + \, O(\Delta t^4 )
  \label{verlet1}
  \end{equation}
  The damping force $G$ is defined by a damping coefficient $\gamma$ and the velocity
  \begin{equation}
   G(v_n) \, = \, \gamma \, v_n \, = \, \gamma \, \dfrac{x_{n+1} \, - \, x_{n-1}}{2 \, \Delta t} ~.
  \end{equation}
  Setting eq. \eqref{verlet1} back by one time step leads to
  \begin{equation}
   x_{n} \, = \, 2 \, x_{n-1} \, - \, x_{n-2} \, + \, \dfrac{\Delta t^2}{m} (\, F_n \, - \, \gamma \, v_n \, ) \, + \, O(\Delta t^4 ) ~.
  \label{verlet2}
  \end{equation}
  By adding eq. \eqref{verlet1} and eq. \eqref{verlet2} the resulting equation reads
  \begin{equation}
   x_{n+1} \, + \, x_n \, = \, 2 \, x_n \, + \, x_{n-1} \, - \, x_{n-2} \, + \, \dfrac{\Delta t^2}{m} (\, F_n \, - \, v_n \, + \, F_{n-1} \, + \, v_{n-1} \, ) \, + \, O(\Delta t^4 )~.
  \end{equation}
  Reformulating the equation, using the velocity interpolation for $v_n$ and $v_{n-1}$, shown in eq. \eqref{velointerpolation} and setting the equation one time step forward leads 
  to the velocity integration for the Velocity-Verlet algorithm with damping
  \begin{equation}
   v_{n+1} \, = \, \dfrac{1}{1 \, + \, \gamma \dfrac{\Delta t}{2 \, m}} \, [ \, v_n \, ( \, 1 \, - \, \gamma \, \dfrac{\Delta t}{2 \, m} \, ) \, + \, \dfrac{\Delta t}{2} \, 
   ( \, a_{n+1} \, + \, a_n \, ) \, ] ~.
  \end{equation}
  The time integration for the position stays the same as for Verlet-Velocity without damping. The parameter $\gamma$ is defined in Peridigm in the Verlet solver list \\ 
  \texttt{
  	<ParameterList name=``Verlet''> \\
	<Parameter name=``Numerical Damping'' type=``double'' value=``0.01''/> \\
  }
  The value should be chosen with respect to the material parameters in the input file.

\subsubsection{Changed files}
\begin{code}
./src/core/Peridigm.cpp
\end{code}
\subsubsection{Documentation of changes}
\paragraph{Peridigm.cpp}
The following changes have been made.
\begin{code}
  double numericalDamping = 0.0;
  if(verletParams->isParameter("Numerical Damping")){
    numericalDamping = verletParams->get<double>("Numerical Damping");
  }
\end{code}
If the parameter for numerical damping is defined in the input file, it will be read.
\begin{code}
    if(verletParams->isParameter("Numerical Damping")){
    	for (int i = 0; i < a->MyLength(); ++i) {
    		(*a)[i] = (*a)[i] / (1 + numericalDamping * dt2 
    		/ (*density)[i/3]);
    		(*v)[i] = (*v)[i] * (1 - numericalDamping * dt2
    		/ (*density)[i/3]) / (1 + numericalDamping * dt2 
    		/ (*density)[i/3]);
    	}
    }
\end{code}
The so-called mothership vectors are updated with the respective coeffcients for the step $v_{n+{\frac{1}{2}}}$.
\begin{code}
    if(verletParams->isParameter("Numerical Damping")){
	for (int i = 0; i < a->MyLength(); ++i) {
		(*a)[i] = (*a)[i] / (1 + numericalDamping * dt2
		/ (*density)[i/3]);
	}
    }
\end{code}
The mothership vector is updated with the respective coeffcients for the step $v_{n+1}$.