%%%%%%%%%%%%%%%%%%%%%%%%%%%%%%%%%%%%
% Header                           %
%%%%%%%%%%%%%%%%%%%%%%%%%%%%%%%%%%%%
% 
% Revisions: 2017-04-10 Martin R�del <martin.raedel@dlr.de>
%                       Initial draft
%               
% Contact:   Martin R�del,  martin.raedel@dlr.de
%            DLR Composite Structures and Adaptive Systems
%          
%                                 __/|__
%                                /_/_/_/  
%            www.dlr.de/fa/en      |/ DLR
% 
%%%%%%%%%%%%%%%%%%%%%%%%%%%%%%%%%%%%
% Content                          %
%%%%%%%%%%%%%%%%%%%%%%%%%%%%%%%%%%%%

\levelup{User defined block interaction}
% \levelup{\idxPDKwGeneralContact}
\label{sec:Peridigm:QRG:Contact:Interaction:Block}
\myindex[\idxPDKeywordName]{\idxPDKwBlockContact}

\leveldown{Description}

Define interactions between individual blocks. This type of interaction is activated by setting the parameter list name to any different name than \idxPDKwGeneralContact{} and \idxPDKwSelfContact{}.

\levelstay{Literature}

-
% \begin{itemize}[noitemsep]
%   \item \cite{SillingSA2005}
%   \item \cite{PeridigmUserGuide100}
% \end{itemize}

\levelstay{Code}

\leveldown{Release version}

Available from \href{\toolrepoversiononefour}{version 1.4}.

\levelstay{Required compiler options}

-

\levelstay{Routines}

-
% \begin{itemize}[noitemsep]
%   \item IO:
%   \begin{itemize}[noitemsep]
%   \item \verb+/src/contact/Peridigm_ShortRangeForceContactModel.cpp+
%   \item \verb+/src/contact/Peridigm_ShortRangeForceContactModel.hpp+
%   \end{itemize}
%   \item Computation:
%   \begin{itemize}[noitemsep]
%     \item \verb+/src/materials/elastic.cxx+
%     \item \verb+/src/materials/elastic.h+
%   \end{itemize}
% \end{itemize}

\levelup{Input parameters}

\leveldown{List}%\mbox{}\\

\begin{tabularx}{\linewidth}{lcccX}
\toprule
Name          & Type   & Required      & Default & Description           \\
\midrule
Contact Model\myindex[\idxPDKeywordName]{\idxPDKwContactModel} & string & \checkmark    & -       & Contact model type from section \ref{sec:Peridigm:QRG:Contact:Models} \\
First Block   & string & \checkmark    & -       & Block name/id of first interaction partner \\
Second Block  & string & \checkmark    & -       & Block name/id of second interaction partner \\
\bottomrule
\end{tabularx}

\levelstay{Remarks}%\mbox{}\\

-
% \begin{enumerate}[noitemsep]
%   \item \label{enm:Peridigm:QRG:Contact:Model:ShortRangeForce:Remark:One} In case \texttt{Friction Coefficient} is not defined, no friction is applied by using the default value
% \end{enumerate}

\levelup{Exemplary input section}

\leveldown{XML-format}%\mbox{}\\

\begingroup
\lstset{breaklines=true}
\begin{code}
<ParameterList name="Interactions">
  <ParameterList name="My Contact Interaction">
    <Parameter name="Contact Model" type="string" value="My Contact Model"/>
    <Parameter name="First Block" type="string" value="block_2"/>
    <Parameter name="Second Block" type="string" value="block_3"/>
  </ParameterList>
</ParameterList>
\end{code}
\endgroup

\levelstay{Free format}%\mbox{}\\

-

\levelstay{YAML format}%\mbox{}\\

-

\levelup{List of examples}

\begin{itemize}[noitemsep]
%   \item From \texttt{examples/}:
%   \begin{itemize}[noitemsep]
%     \item \texttt{examples/tensile\_test/tensile\_test.peridigm}
%   \end{itemize}
  \item From \texttt{test/regression/}:
  \begin{itemize}[noitemsep]
    \item \texttt{Contact\_Cubes\_Interaction\_Blocks/Contact\_Cubes\_Interaction\_Blocks.xml}
    \item \texttt{Contact\_Perforation/Contact\_Perforation.xml}
  \end{itemize}
  \item From \texttt{test/verification/}:
  \begin{itemize}[noitemsep]
    \item \texttt{Contact\_2x1x1/Contact\_2x1x1.xml}
    \item \texttt{Contact\_Friction/Contact\_Friction.xml}
    \item \texttt{Contact\_Friction\_Time\_Dependent\_Coefficient/Contact\_Friction\_Time\_Dependent\_Coefficient.xml}
  \end{itemize}
\end{itemize}