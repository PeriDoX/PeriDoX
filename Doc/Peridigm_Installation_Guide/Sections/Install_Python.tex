%%%%%%%%%%%%%%%%%%%%%%%%%%%%%%%%%%%%
% Header                           %
%%%%%%%%%%%%%%%%%%%%%%%%%%%%%%%%%%%%
% 
% Revisions: 2017-04-10 Martin Rädel <martin.raedel@dlr.de>
%                       Initial draft
%               
% Contact:   Martin Rädel,  martin.raedel@dlr.de
%            DLR Composite Structures and Adaptive Systems
%          
%                                 __/|__
%                                /_/_/_/  
%            www.dlr.de/fa/en      |/ DLR
% 
%%%%%%%%%%%%%%%%%%%%%%%%%%%%%%%%%%%%
% Content                          %
%%%%%%%%%%%%%%%%%%%%%%%%%%%%%%%%%%%%

\levelup{\texorpdfstring{\protect\marktool{\pythonname}}{\pythonname{}}}

\paragraph{Use the operating system distribution}

\marktool{\pythonname} is already part of an openSUSE standard installation since also system components require python. The installed version can be shown in the terminal by the command

\begin{code}
python -V
\end{code}

Packages are available for both \marktool{\pythonname} 2.7 as well as \marktool{\pythonname} 3.x. A parallel installation if \marktool{\pythonname} 2 and \marktool{\pythonname} 3 possible without problems or package conflicts.

To update the python distribution to the newest available state in the OS repositories, open a terminal, login as \verb+root\verb+ and use the following command

\begin{code}
zypper update python
\end{code}

Additionally, \verb+python-devel+ is required, so

\begin{code}
zypper install python-devel
\end{code}

\paragraph{Perform a new installation}

If no initial version of \marktool{\pythonname} is present in the operating system it is necessary to download the source and install the source. For the latest or required version of \marktool{\pythonname} visit

\href{http://www.python.org/download/}{http://www.python.org/download/}

For the installation, open a terminal and change directory to \verb+/home/USERNAME/bin+ for a single-user installation or \verb+/usr/local/bin+ for an installation for all users

\begin{code}
cd $DOWNLOAD_DIR
wget https://www.python.org/ftp/python/2.7.11/Python-2.7.11.tgz	%\lstcomment{\# Download}%
tar xvfz Python-2.7.11.tgz 	%\lstcomment{\# unzip}%
cd Python-2.7.11		%\lstcomment{\# go into directory}%
./configure
make				%\lstcomment{\# build}%
make altinstall			%\lstcomment{\# install}%
\end{code}

Afterwards, you are free to delete the downloaded Gzipped source tarball, here

\begin{code}
cd $DOWNLOAD_DIR
rm Python-2.7.11.tgz 		%\lstcomment{\# delete}%
\end{code}

% \paragraph{Update symbolic links}
% 
% In case a newer \marktool{\pythonname} version is installed locally, the \verb+alias+ or symbolic links to the binary files have to be updated. The default binaries are located in \verb+/usr/bin/+. To list all current python binaries and links open a terminal, change directory to \verb+/usr/bin/+ and type
% 
% \begin{code}
% ls -l | grep python
% \end{code}
% 
% The result should look something like the following:
% 
% \begingroup
% \lstset{keepspaces=true}
% \begin{code}
% lrwxrwxrwx 1 root root           9  2. Dez 12:58 python -> python2.7
% lrwxrwxrwx 1 root root           9  2. Dez 12:58 python2 -> python2.7
% -rwxr-xr-x 1 root root        6296 25. Okt 09:59 python2.7
% lrwxrwxrwx 1 root root           9  2. Dez 12:58 python3 -> python3.4
% -rwxr-xr-x 2 root root       10448 25. Okt 09:59 python3.4
% -rwxr-xr-x 2 root root       10448 25. Okt 09:59 python3.4m
% \end{code}
% \endgroup
% 
% Depending on the version of the new installation the path to the binaries can be updated using symbolic links to the user installation directory. Assuming the current directory is \verb+/usr/bin/+ and the new \marktool{\pythonname} installation directory is \verb+/usr/local/bin+ the links can be created with the help of \verb+ln -sf+:
% 
% \begin{code}
% ln -sf python /usr/local/bin/python2.7
% ln -sf python2 /usr/local/bin/python2.7
% ln -sf python2.7 /usr/local/bin/python2.7
% \end{code}