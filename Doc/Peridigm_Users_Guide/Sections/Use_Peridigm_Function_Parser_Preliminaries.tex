%%%%%%%%%%%%%%%%%%%%%%%%%%%%%%%%%%%%
% Header                           %
%%%%%%%%%%%%%%%%%%%%%%%%%%%%%%%%%%%%
% 
% Revisions: 2017-04-10 Martin R�del <martin.raedel@dlr.de>
%                       Initial draft
%               
% Contact:   Martin R�del,  martin.raedel@dlr.de
%            DLR Composite Structures and Adaptive Systems
%          
%                                 __/|__
%                                /_/_/_/  
%            www.dlr.de/fa/en      |/ DLR
% 
%%%%%%%%%%%%%%%%%%%%%%%%%%%%%%%%%%%%
% Content                          %
%%%%%%%%%%%%%%%%%%%%%%%%%%%%%%%%%%%%

\leveldown{Preliminaries}

\leveldown{Description}

You can use functions inside a \marktool{\toolname} input deck. This means, it is possible to use algebraic expressions, e.g. for the definition of boundary conditions. \toolname{} uses the \href{https://github.com/trilinos/Trilinos/tree/master/packages/pamgen/rtcompiler}{RTCompiler} function parser to process C-style expressions for the specification of input parameters, including initial and boundary conditions. The RTCompiler (RTC) is used via the \trilinosname{} PG\_RuntimeCompiler class.

\levelstay{History}

The internal mechanism for processing functions used to be \href{https://sourceforge.net/projects/muparser/}{muParser}, and was recently changed to rtcompiler, see \href{https://github.com/peridigm/peridigm/commit/02dc1bb21b17e6ed406bf084442541f3723ab368}{this commit}.