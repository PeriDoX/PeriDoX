%%%%%%%%%%%%%%%%%%%%%%%%%%%%%%%%%%%%
% Header                           %
%%%%%%%%%%%%%%%%%%%%%%%%%%%%%%%%%%%%
% 
% Revisions: 2017-04-10 Martin R�del <martin.raedel@dlr.de>
%                       Initial draft
%               
% Contact:   Martin R�del,  martin.raedel@dlr.de
%            DLR Composite Structures and Adaptive Systems
%          
%                                 __/|__
%                                /_/_/_/  
%            www.dlr.de/fa/en      |/ DLR
% 
%%%%%%%%%%%%%%%%%%%%%%%%%%%%%%%%%%%%
% Content                          %
%%%%%%%%%%%%%%%%%%%%%%%%%%%%%%%%%%%%

\levelup{\idxPDKwElasticBondBased}
\label{sec:Peridigm:QRG:Materials:ElasticBondBased}
\myindex[\idxPDKeywordName]{\idxPDKwElasticBondBased}
\myindex[\idxPDKeywordName]{\idxPDKwMaterials!\idxPDKwElasticBondBased|see{\idxPDKwElasticBondBased}}

\leveldown{Description}

An isotropic, linear elastic bond-based material model.

\levelstay{Literature}

\begin{itemize}[noitemsep]
  \item \cite{SillingSA2000}
\end{itemize}

\levelstay{Stiffness model sketch}

\begin{figure}[htbp]
  \begin{subfigure}{0.49\linewidth}
    \centering
    \tikzexternalenable
    \tikzsetnextfilename{Material_Model_Elastic-Linear-K}
    \begin{tikzpicture}
  % Variable
  \def\modulus{70000}
  \def\yieldstresst{350}
  \def\yieldstressc{350}
  \def\xlabel{$\glssymbol{symb:scalar:mech:strain:normal:engineering}$}
  \def\ylabel{$\glssymbol{symb:scalar:mech:stress:normal:engineering}$}
  \def\pinlabel{$\glssymbol{symb:scalar:mat:modulus:bulk}$}
  \newtoggle{tclabel}
  \toggletrue{tclabel}
  %%%%%%%%%%%%%%%%%%%%%%%%%%%%%%%%%%%%
% Header                           %
%%%%%%%%%%%%%%%%%%%%%%%%%%%%%%%%%%%%
% 
% Revisions: 2017-04-10 Martin R�del <martin.raedel@dlr.de>
%                       Initial draft
%               
% Contact:   Martin R�del,  martin.raedel@dlr.de
%            DLR Composite Structures and Adaptive Systems
%          
%                                 __/|__
%                                /_/_/_/  
%            www.dlr.de/fa/en      |/ DLR
%
%%%%%%%%%%%%%%%%%%%%%%%%%%%%%%%%%%%%
% Content                          %
%%%%%%%%%%%%%%%%%%%%%%%%%%%%%%%%%%%%

% Math
\pgfkeys{/pgf/fpu}
\pgfmathsetmacro{\yieldstraint}{\yieldstresst/\modulus}
\pgfmathsetmacro{\yieldstrainc}{\yieldstressc/\modulus}
\pgfkeys{/pgf/fpu=false}
% Shapes
\tikzset{%
  myarrowdecoration1/.style={%
    postaction={%
      decorate,%
      decoration={%
        markings,%
        mark=between positions .4 and .6 step .1pt with {\draw [thin] circle (.1pt);},%
        mark=at position .6 with {\arrow[thin,xshift=1pt]{latex}},%
        raise=-0.7ex,%
      }%
    }%
  },
  myarrowdecoration2/.style={
    postaction={%
      decorate,%
      decoration={%
        markings,%
        mark=between positions .4 and .6 step .1pt with {\draw [thin] circle (.1pt);},%
        mark=at position .4 with {\arrow[thin,xshift=1pt]{latex reversed}},%
        raise=0.7ex,%
      }%
    }%
  },
}
% Axis
\begin{axis}[
%     scale only axis,
  axis lines=middle,
  ticks=none,
  %restrict y to domain=-\ultimatestrength:\ultimatestrength,
  xmin=-1.2*\yieldstrainc,
  xmax= 1.2*\yieldstraint,
  ymin=-1.5*\yieldstressc,
  ymax= 1.5*\yieldstresst,
  width=0.99\textwidth,
  height=0.99\textwidth,
  xlabel=\xlabel,
  ylabel=\ylabel,
  every axis x label/.style={
    at={(ticklabel* cs:1.005)},
    anchor=west,
  },
  every axis y label/.style={
    at={(ticklabel* cs:1.005)},
    anchor=south,
  },
]
  % Coordinates
  \coordinate (origin)     at (0,0);
  \coordinate (yieldt)     at ( \yieldstraint, \yieldstresst);
  \coordinate (yieldc)     at (-\yieldstrainc,-\yieldstressc);
  % Lines
  \draw[thick,draw=\plotcolor,myarrowdecoration1,myarrowdecoration2] (origin) -- node[pos=0.8, pin=-60:{\pinlabel}](ELabel){} (yieldt);
  \draw[thick,draw=\plotcolor,myarrowdecoration1,myarrowdecoration2] (origin) -- (yieldc);
  % Label
  \iftoggle{tclabel}{%
    \node[anchor=north east] (tensionlabel) at (rel axis cs:1,1) {\footnotesize tension};
    \node[anchor=south west] (compressionlabel) at (rel axis cs:0,0) {\footnotesize compression};
  }{}
\end{axis}
\end{tikzpicture}
    \tikzexternaldisable
    \caption{Axial}
    \label{fig:Material_Models_ElasticBondBased-E}
  \end{subfigure}%
  \begin{subfigure}{0.49\linewidth}
    \centering
    \tikzexternalenable
    \tikzsetnextfilename{Material_Model_Elastic-Linear-G}
    \begin{tikzpicture}
  % Variable
  \def\modulus{70000}
  \def\yieldstresst{350}
  \def\yieldstressc{350}
  \def\xlabel{$\glssymbol{symb:scalar:mech:strain:shear:engineering}$}
  \def\ylabel{$\glssymbol{symb:scalar:mech:stress:shear:engineering}$}
  \def\pinlabel{$\glssymbol{symb:scalar:mat:modulus:shear}$}
  \newtoggle{tclabel}
  \togglefalse{tclabel}
  %%%%%%%%%%%%%%%%%%%%%%%%%%%%%%%%%%%%
% Header                           %
%%%%%%%%%%%%%%%%%%%%%%%%%%%%%%%%%%%%
% 
% Revisions: 2017-04-10 Martin R�del <martin.raedel@dlr.de>
%                       Initial draft
%               
% Contact:   Martin R�del,  martin.raedel@dlr.de
%            DLR Composite Structures and Adaptive Systems
%          
%                                 __/|__
%                                /_/_/_/  
%            www.dlr.de/fa/en      |/ DLR
%
%%%%%%%%%%%%%%%%%%%%%%%%%%%%%%%%%%%%
% Content                          %
%%%%%%%%%%%%%%%%%%%%%%%%%%%%%%%%%%%%

% Math
\pgfkeys{/pgf/fpu}
\pgfmathsetmacro{\yieldstraint}{\yieldstresst/\modulus}
\pgfmathsetmacro{\yieldstrainc}{\yieldstressc/\modulus}
\pgfkeys{/pgf/fpu=false}
% Shapes
\tikzset{%
  myarrowdecoration1/.style={%
    postaction={%
      decorate,%
      decoration={%
        markings,%
        mark=between positions .4 and .6 step .1pt with {\draw [thin] circle (.1pt);},%
        mark=at position .6 with {\arrow[thin,xshift=1pt]{latex}},%
        raise=-0.7ex,%
      }%
    }%
  },
  myarrowdecoration2/.style={
    postaction={%
      decorate,%
      decoration={%
        markings,%
        mark=between positions .4 and .6 step .1pt with {\draw [thin] circle (.1pt);},%
        mark=at position .4 with {\arrow[thin,xshift=1pt]{latex reversed}},%
        raise=0.7ex,%
      }%
    }%
  },
}
% Axis
\begin{axis}[
%     scale only axis,
  axis lines=middle,
  ticks=none,
  %restrict y to domain=-\ultimatestrength:\ultimatestrength,
  xmin=-1.2*\yieldstrainc,
  xmax= 1.2*\yieldstraint,
  ymin=-1.5*\yieldstressc,
  ymax= 1.5*\yieldstresst,
  width=0.99\textwidth,
  height=0.99\textwidth,
  xlabel=\xlabel,
  ylabel=\ylabel,
  every axis x label/.style={
    at={(ticklabel* cs:1.005)},
    anchor=west,
  },
  every axis y label/.style={
    at={(ticklabel* cs:1.005)},
    anchor=south,
  },
]
  % Coordinates
  \coordinate (origin)     at (0,0);
  \coordinate (yieldt)     at ( \yieldstraint, \yieldstresst);
  \coordinate (yieldc)     at (-\yieldstrainc,-\yieldstressc);
  % Lines
  \draw[thick,draw=\plotcolor,myarrowdecoration1,myarrowdecoration2] (origin) -- node[pos=0.8, pin=-60:{\pinlabel}](ELabel){} (yieldt);
  \draw[thick,draw=\plotcolor,myarrowdecoration1,myarrowdecoration2] (origin) -- (yieldc);
  % Label
  \iftoggle{tclabel}{%
    \node[anchor=north east] (tensionlabel) at (rel axis cs:1,1) {\footnotesize tension};
    \node[anchor=south west] (compressionlabel) at (rel axis cs:0,0) {\footnotesize compression};
  }{}
\end{axis}
\end{tikzpicture}
    \tikzexternaldisable
    \caption{Shear}
    \label{fig:Material_Models_ElasticBondBased-G}
  \end{subfigure}%
  \caption{Linear-elastic material model}
  \label{fig:Material_Models_ElasticBondBased}
\end{figure}

\levelstay{Code}

\leveldown{Release version}

Available from version 1.5.

\levelstay{Required compiler options}

-

\levelstay{Routines}

\begin{itemize}[noitemsep]
  \item IO:
  \begin{itemize}[noitemsep]
    \item \texttt{/src/materials/Peridigm\_ElasticBondBasedMaterial.cpp}
    \item \texttt{/src/materials/Peridigm\_ElasticBondBasedMaterial.hpp}
  \end{itemize}
  \item Computation:
  \begin{itemize}[noitemsep]
    \item \texttt{/src/materials/elastic\_bond\_based.cxx}
    \item \texttt{/src/materials/elastic\_bond\_based.h}
  \end{itemize}
\end{itemize}

\levelup{Input parameters}

\leveldown{List}

\begin{tabularx}{\linewidth}{lcccX}
\toprule
Name           & Type   & Required   & Default & Description \\
\midrule
Material Model & string & \checkmark & -       & Material type ``\idxPDKwElasticBondBased'' \\
Density        & double & \checkmark & -       & Material density	\\
Bulk modulus   & double & \checkmark\textsuperscript{\ref{enm:Peridigm:QRG:Materials:ElasticBondBased:Remark:Modulus:One},\ref{enm:Peridigm:QRG:Materials:ElasticBondBased:Remark:Modulus:Two},\ref{enm:Peridigm:QRG:Materials:ElasticBondBased:Remark:Modulus:Three}} & - & Volumetric elasticity \\
\bottomrule
\end{tabularx}

\levelstay{Remarks}

\begin{enumerate}[noitemsep]
  \item \label{enm:Peridigm:QRG:Materials:ElasticBondBased:Remark:Modulus:One} Only the bulk modulus, $\glssymbol{symb:scalar:mat:modulus:bulk}$, can be used as elastic constant.
  \item \label{enm:Peridigm:QRG:Materials:ElasticBondBased:Remark:Modulus:Two} The Poisson's Ratio is set to a fixed value of $\frac14$. The validity range of this value is discussed in \autoref{tab:Peridigm:QRG:Preliminaries:PDCMConversion}
  \item \label{enm:Peridigm:QRG:Materials:ElasticBondBased:Remark:Modulus:Three} The \textit{Shear Modulus} is calculated from
  \begin{align*}
  \glssymbol{symb:scalar:mat:modulus:shear}&=\dfrac{3\glssymbol{symb:scalar:mat:modulus:bulk}\left(1-2\glssymbol{symb:scalar:mat:poissonratio}\right)}{2\cdot\left(1+\glssymbol{symb:scalar:mat:poissonratio}\right)}
  \end{align*}
  \item Shear Correction Factor is not supported for the material model
  \item Thermal expansion is not currently supported for the material model
%   \item Consider the general remarks on bond-based materials in section \ref{sec:Peridigm:QRG:Materials:Preliminaries:Correspondence}
\end{enumerate}

\levelup{Exemplary input section}

\leveldown{XML-format}

-

\levelstay{Free format}

-

\levelstay{YAML format}

-
  
\levelup{Possible output variables for the material model}

% \begin{multicols}{2}
% \begin{itemize}[noitemsep]
%   \item Bond\_Damage
%   \item Coordinates
%   \item Damage
%   \item Deviatoric\_Plastic\_Extension
%   \item Dilatation
%   \item Force\_Density
%   \item Lambda
%   \item Model\_Coordinates
%   \item Partial\_Stress
%   \item Surface\_Correction\_Factor
%   \item Temperature\_Change
%   \item Volume
%   \item Weighted\_Volume
% \end{itemize}
% \end{multicols}

\levelstay{List of examples}

-
% \begin{itemize}[noitemsep]
%   \item From \texttt{examples/}:
%   \begin{itemize}[noitemsep]
%     \item \texttt{examples/tensile\_test/tensile\_test.peridigm}
%   \end{itemize}
%   \item From \texttt{test/regression/}:
%   \begin{itemize}[noitemsep]
%     \item \texttt{Bar_OneBlock_OneMaterial_QS/Bar.xml}
%   \end{itemize}
%   \item From \texttt{test/verification/}:
%   \begin{itemize}[noitemsep]
%     \item \texttt{Contact\_Friction\_Time\_Dependent\_Coefficient/Contact\_Friction\_Time\_Dependent\_Coefficient.xml}
%   \end{itemize}
% \end{itemize}