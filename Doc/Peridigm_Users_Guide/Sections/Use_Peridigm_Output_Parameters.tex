%%%%%%%%%%%%%%%%%%%%%%%%%%%%%%%%%%%%
% Header                           %
%%%%%%%%%%%%%%%%%%%%%%%%%%%%%%%%%%%%
% 
% Revisions: 2017-04-10 Martin R�del <martin.raedel@dlr.de>
%                       Initial draft
%               
% Contact:   Martin R�del,  martin.raedel@dlr.de
%            DLR Composite Structures and Adaptive Systems
%          
%                                 __/|__
%                                /_/_/_/  
%            www.dlr.de/fa/en      |/ DLR
% 
%%%%%%%%%%%%%%%%%%%%%%%%%%%%%%%%%%%%
% Content                          %
%%%%%%%%%%%%%%%%%%%%%%%%%%%%%%%%%%%%

\levelup{Output}
\label{sec:Peridigm:QRG:Output:Output}

\leveldown{Code}

\begin{itemize}[noitemsep]
  \item \texttt{src/core/Peridigm.cpp}
  \item \texttt{src/io/Peridigm\_OutputManager\_ExodusII.cpp}
\end{itemize}

\levelstay{Input parameters}
\label{sec:Peridigm:QRG:Output:Output:Parameters}

\leveldown{List}

\begin{tabularx}{\linewidth}{lcccX}
\toprule
Name		& Type		& Required	& Default	& Description		\\
\midrule
\multicolumn{5}{l}{Base parameters}	\\
Output File Type	& string	& -		& ``ExodusII''		& ``ExodusII''	\\
Output Format		& string	& -		& ``BINARY''		& ``BINARY'' | ``ASCII''	\\
Output Filename\textsuperscript{\ref{enm:Peridigm:QRG:Output:Output:Remark:One}}		& string	& -		& ``dump''		& Output file name	\\
Output Frequency	& int		& \checkmark	& -			& \\
Initial Output Step	& int		& -		& 1			& First time step this output shall apply to\\
Final Output Step	& int		& -		& last in model		& Last time step this output shall apply to\\
Parallel Write		& bool		& -		& true			& Flag whether to write in parallel	\\
Output Variables\textsuperscript{\ref{enm:Peridigm:QRG:Output:Output:Remark:Two}}	& list		& \checkmark	& -			& Opens list of output variables	\\
\bottomrule
\end{tabularx}

\levelstay{Remarks}

\begin{enumerate}[noitemsep]
  \item \label{enm:Peridigm:QRG:Output:Output:Remark:One} The \textit{Output Filename} should be unique for each \textit{Output}. Otherwise, output files are probably overwritten.
  \item \label{enm:Peridigm:QRG:Output:Output:Remark:Two} See section \ref{sec:Peridigm:QRG:Output:Variables}
\end{enumerate}

\levelup{Exemplary input section}

\leveldown{XML-format}

from \texttt{Body\_Force\_QS.xml}:

\begingroup
\lstset{breaklines=true}
\begin{code}
<ParameterList name="Output">
  <Parameter name="Output File Type" type="string" value="ExodusII"/>
  <Parameter name="Output Format" type="string" value="BINARY"/>
  <Parameter name="Output Filename" type="string" value="Body_Force_QS"/>
  <Parameter name="Output Frequency" type="int" value="1"/>
  <Parameter name="Parallel Write" type="bool" value="true"/>
  <ParameterList name="Output Variables">
    <Parameter name="Displacement" type="bool" value="true"/>
    <Parameter name="Velocity" type="bool" value="true"/>
    <Parameter name="Element_Id" type="bool" value="true"/>
    <Parameter name="Proc_Num" type="bool" value="true"/>
    <Parameter name="Dilatation" type="bool" value="true"/>
    <Parameter name="Force_Density" type="bool" value="true"/>
    <Parameter name="Weighted_Volume" type="bool" value="true"/>
  </ParameterList>
</ParameterList>
\end{code}
\endgroup

\levelstay{Free format}

from \texttt{tensile\_test.peridigm}:
from \texttt{NOX\_QS\_MatrixFree\_NoPrec\_YAML.yaml}:

\begingroup
\lstset{breaklines=true}
\begin{code}
Output
  Output File Type "ExodusII"
  Output Filename "tensile_test"
  Output Frequency 1
  Output Variables
    Displacement "true"
    Velocity "true"
    Element_Id "true"
    Proc_Num "true"
    Force_Density "true"
    Hourglass_Force_Density "true"
    Force "true"
    Volume "true"
\end{code}
\endgroup

\levelstay{YAML format}

from \texttt{NOX\_QS\_MatrixFree\_NoPrec\_YAML.yaml}:

\begingroup
\lstset{breaklines=true}
\begin{code}
Output:
  Output File Type: "ExodusII"
  Output Filename: "NOX_QS_MatrixFree_NoPrec_YAML"
  Output Frequency: 1
  Output Variables:
    Displacement: "true"
    Velocity: "true"
    Force: "true"
    Volume: "true"
    Radius: "true"
    Nonlinear_Solver_Iterations: "true"
\end{code}
\endgroup

\levelup{List of examples}

Basically all models in \texttt{/test/} and \texttt{/examples/}.