%%%%%%%%%%%%%%%%%%%%%%%%%%%%%%%%%%%%
% Header                           %
%%%%%%%%%%%%%%%%%%%%%%%%%%%%%%%%%%%%
% 
% Revisions: 2017-04-10 Martin R�del <martin.raedel@dlr.de>
%                       Initial draft
%               
% Contact:   Martin R�del,  martin.raedel@dlr.de
%            DLR Composite Structures and Adaptive Systems
%          
%                                 __/|__
%                                /_/_/_/  
%            www.dlr.de/fa/en      |/ DLR
%
%%%%%%%%%%%%%%%%%%%%%%%%%%%%%%%%%%%%
% Content                          %
%%%%%%%%%%%%%%%%%%%%%%%%%%%%%%%%%%%%

\levelup{\idxPDKwElasticPartialVolume}
\label{sec:Peridigm:QRG:Materials:ElasticPartialVolume}
\myindex[\idxPDKeywordName]{\idxPDKwElasticPartialVolume}
\myindex[\idxPDKeywordName]{\idxPDKwMaterials!\idxPDKwElasticPartialVolume|see{\idxPDKwElasticPartialVolume}}


\leveldown{Description}

An isotropic, partial volume linear elastic material model.

\levelstay{Literature}

-
% \begin{itemize}[noitemsep]
%   \item \cite{MitchellJA2015}
% \end{itemize}

\levelstay{Stiffness model sketch}

\begin{figure}[htbp]
  \begin{subfigure}{0.49\linewidth}
    \centering
    \tikzexternalenable
    \tikzsetnextfilename{Material_Model_Elastic-Linear-K}
    \begin{tikzpicture}
  % Variable
  \def\modulus{70000}
  \def\yieldstresst{350}
  \def\yieldstressc{350}
  \def\xlabel{$\glssymbol{symb:scalar:mech:strain:normal:engineering}$}
  \def\ylabel{$\glssymbol{symb:scalar:mech:stress:normal:engineering}$}
  \def\pinlabel{$\glssymbol{symb:scalar:mat:modulus:bulk}$}
  \newtoggle{tclabel}
  \toggletrue{tclabel}
  %%%%%%%%%%%%%%%%%%%%%%%%%%%%%%%%%%%%
% Header                           %
%%%%%%%%%%%%%%%%%%%%%%%%%%%%%%%%%%%%
% 
% Revisions: 2017-04-10 Martin R�del <martin.raedel@dlr.de>
%                       Initial draft
%               
% Contact:   Martin R�del,  martin.raedel@dlr.de
%            DLR Composite Structures and Adaptive Systems
%          
%                                 __/|__
%                                /_/_/_/  
%            www.dlr.de/fa/en      |/ DLR
%
%%%%%%%%%%%%%%%%%%%%%%%%%%%%%%%%%%%%
% Content                          %
%%%%%%%%%%%%%%%%%%%%%%%%%%%%%%%%%%%%

% Math
\pgfkeys{/pgf/fpu}
\pgfmathsetmacro{\yieldstraint}{\yieldstresst/\modulus}
\pgfmathsetmacro{\yieldstrainc}{\yieldstressc/\modulus}
\pgfkeys{/pgf/fpu=false}
% Shapes
\tikzset{%
  myarrowdecoration1/.style={%
    postaction={%
      decorate,%
      decoration={%
        markings,%
        mark=between positions .4 and .6 step .1pt with {\draw [thin] circle (.1pt);},%
        mark=at position .6 with {\arrow[thin,xshift=1pt]{latex}},%
        raise=-0.7ex,%
      }%
    }%
  },
  myarrowdecoration2/.style={
    postaction={%
      decorate,%
      decoration={%
        markings,%
        mark=between positions .4 and .6 step .1pt with {\draw [thin] circle (.1pt);},%
        mark=at position .4 with {\arrow[thin,xshift=1pt]{latex reversed}},%
        raise=0.7ex,%
      }%
    }%
  },
}
% Axis
\begin{axis}[
%     scale only axis,
  axis lines=middle,
  ticks=none,
  %restrict y to domain=-\ultimatestrength:\ultimatestrength,
  xmin=-1.2*\yieldstrainc,
  xmax= 1.2*\yieldstraint,
  ymin=-1.5*\yieldstressc,
  ymax= 1.5*\yieldstresst,
  width=0.99\textwidth,
  height=0.99\textwidth,
  xlabel=\xlabel,
  ylabel=\ylabel,
  every axis x label/.style={
    at={(ticklabel* cs:1.005)},
    anchor=west,
  },
  every axis y label/.style={
    at={(ticklabel* cs:1.005)},
    anchor=south,
  },
]
  % Coordinates
  \coordinate (origin)     at (0,0);
  \coordinate (yieldt)     at ( \yieldstraint, \yieldstresst);
  \coordinate (yieldc)     at (-\yieldstrainc,-\yieldstressc);
  % Lines
  \draw[thick,draw=\plotcolor,myarrowdecoration1,myarrowdecoration2] (origin) -- node[pos=0.8, pin=-60:{\pinlabel}](ELabel){} (yieldt);
  \draw[thick,draw=\plotcolor,myarrowdecoration1,myarrowdecoration2] (origin) -- (yieldc);
  % Label
  \iftoggle{tclabel}{%
    \node[anchor=north east] (tensionlabel) at (rel axis cs:1,1) {\footnotesize tension};
    \node[anchor=south west] (compressionlabel) at (rel axis cs:0,0) {\footnotesize compression};
  }{}
\end{axis}
\end{tikzpicture}
    \tikzexternaldisable
    \caption{Axial}
    \label{fig:Material_Models_PVLinearElastic-E}
  \end{subfigure}%
  \begin{subfigure}{0.49\linewidth}
    \centering
    \tikzexternalenable
    \tikzsetnextfilename{Material_Model_Elastic-Linear-G}
    \begin{tikzpicture}
  % Variable
  \def\modulus{70000}
  \def\yieldstresst{350}
  \def\yieldstressc{350}
  \def\xlabel{$\glssymbol{symb:scalar:mech:strain:shear:engineering}$}
  \def\ylabel{$\glssymbol{symb:scalar:mech:stress:shear:engineering}$}
  \def\pinlabel{$\glssymbol{symb:scalar:mat:modulus:shear}$}
  \newtoggle{tclabel}
  \togglefalse{tclabel}
  %%%%%%%%%%%%%%%%%%%%%%%%%%%%%%%%%%%%
% Header                           %
%%%%%%%%%%%%%%%%%%%%%%%%%%%%%%%%%%%%
% 
% Revisions: 2017-04-10 Martin R�del <martin.raedel@dlr.de>
%                       Initial draft
%               
% Contact:   Martin R�del,  martin.raedel@dlr.de
%            DLR Composite Structures and Adaptive Systems
%          
%                                 __/|__
%                                /_/_/_/  
%            www.dlr.de/fa/en      |/ DLR
%
%%%%%%%%%%%%%%%%%%%%%%%%%%%%%%%%%%%%
% Content                          %
%%%%%%%%%%%%%%%%%%%%%%%%%%%%%%%%%%%%

% Math
\pgfkeys{/pgf/fpu}
\pgfmathsetmacro{\yieldstraint}{\yieldstresst/\modulus}
\pgfmathsetmacro{\yieldstrainc}{\yieldstressc/\modulus}
\pgfkeys{/pgf/fpu=false}
% Shapes
\tikzset{%
  myarrowdecoration1/.style={%
    postaction={%
      decorate,%
      decoration={%
        markings,%
        mark=between positions .4 and .6 step .1pt with {\draw [thin] circle (.1pt);},%
        mark=at position .6 with {\arrow[thin,xshift=1pt]{latex}},%
        raise=-0.7ex,%
      }%
    }%
  },
  myarrowdecoration2/.style={
    postaction={%
      decorate,%
      decoration={%
        markings,%
        mark=between positions .4 and .6 step .1pt with {\draw [thin] circle (.1pt);},%
        mark=at position .4 with {\arrow[thin,xshift=1pt]{latex reversed}},%
        raise=0.7ex,%
      }%
    }%
  },
}
% Axis
\begin{axis}[
%     scale only axis,
  axis lines=middle,
  ticks=none,
  %restrict y to domain=-\ultimatestrength:\ultimatestrength,
  xmin=-1.2*\yieldstrainc,
  xmax= 1.2*\yieldstraint,
  ymin=-1.5*\yieldstressc,
  ymax= 1.5*\yieldstresst,
  width=0.99\textwidth,
  height=0.99\textwidth,
  xlabel=\xlabel,
  ylabel=\ylabel,
  every axis x label/.style={
    at={(ticklabel* cs:1.005)},
    anchor=west,
  },
  every axis y label/.style={
    at={(ticklabel* cs:1.005)},
    anchor=south,
  },
]
  % Coordinates
  \coordinate (origin)     at (0,0);
  \coordinate (yieldt)     at ( \yieldstraint, \yieldstresst);
  \coordinate (yieldc)     at (-\yieldstrainc,-\yieldstressc);
  % Lines
  \draw[thick,draw=\plotcolor,myarrowdecoration1,myarrowdecoration2] (origin) -- node[pos=0.8, pin=-60:{\pinlabel}](ELabel){} (yieldt);
  \draw[thick,draw=\plotcolor,myarrowdecoration1,myarrowdecoration2] (origin) -- (yieldc);
  % Label
  \iftoggle{tclabel}{%
    \node[anchor=north east] (tensionlabel) at (rel axis cs:1,1) {\footnotesize tension};
    \node[anchor=south west] (compressionlabel) at (rel axis cs:0,0) {\footnotesize compression};
  }{}
\end{axis}
\end{tikzpicture}
    \tikzexternaldisable
    \caption{Shear}
    \label{fig:Material_Models_PVLinearElastic-G}
  \end{subfigure}%
  \caption{Partial-volume linear-elastic material model}
  \label{fig:Material_Models_PVLinearElastic}
\end{figure}

\levelstay{Code}

\leveldown{Release version}

?% Available from version 1.5.

\levelstay{Required compiler options}

use \texttt{-D USE\_PV:BOOL=ON} %and specify/adapt the following path:

% \begin{code}
% -D USE_LCM:BOOL=ON
% -D LCM_INCLUDE_DIR:PATH=/Users/djlittl/Albany/src/LCM
% -D LCM_LIBRARY_DIR:PATH=/Users/djlittl/Albany/GCC_4.7.2_OPT/src
% \end{code}

\levelstay{Routines}

% \begin{itemize}[noitemsep]
%   \item IO:
%   \begin{itemize}[noitemsep]
%     \item \verb+/src/materials/Peridigm_Pals_Model.cpp+
%     \item \verb+/src/materials/Peridigm_Pals_Model.hpp+
%   \end{itemize}
%   \item Computation:
%   \begin{itemize}[noitemsep]
%     \item \verb+/src/materials/pals.cxx+
%     \item \verb+/src/materials/pals.h+
%   \end{itemize}
% \end{itemize}

\levelup{Input parameters}

\leveldown{List}

% \begin{tabularx}{\linewidth}{lcccX}
% \toprule
% Name            & Type          & Required      & Default       & Description           \\
% \midrule
\begin{filecontents}{\tabledir\jobname-parammatelasticpvltxtable.tex}
\begin{longtable}{@{}lcccX@{}}
% ---------------------------
% Header & Footer
% ---------------------------
%
% Header
% -----------------
% 1st head
\toprule
Name          & Type          & Required      & Default       & Description           \\
\midrule
\endfirsthead
% Last head
\multicolumn{5}{@{}l}{\ldots continued}\\
\toprule
Name          & Type          & Required      & Default       & Description           \\
\midrule
\endhead
%
% Footer
% -----------------
% n-th foot
\bottomrule
\multicolumn{5}{r@{}}{continued \ldots}\\
\endfoot
% last foot
\bottomrule
\endlastfoot
% ---------------------------
% Content
% ---------------------------
Material Model  & string & \checkmark & - & Material type ``\idxPDKwElasticPartialVolume''        \\
Density         & double & \checkmark & - & Material density      \\
Bulk modulus    & double & \checkmark\textsuperscript{\ref{enm:Peridigm:QRG:Materials:ElasticPartialVolume:Remark:Modulus:One},\ref{enm:Peridigm:QRG:Materials:ElasticPartialVolume:Remark:Modulus:Two}}       & -             & Volumetric elasticity\\
Shear Modulus   & double & \checkmark\textsuperscript{\ref{enm:Peridigm:QRG:Materials:ElasticPartialVolume:Remark:Modulus:One},\ref{enm:Peridigm:QRG:Materials:ElasticPartialVolume:Remark:Modulus:Two}}       & -             & Shear elasticity or engineering constant for shear stiffness\\
Young\verb+'+s Modulus & double & \checkmark\textsuperscript{\ref{enm:Peridigm:QRG:Materials:ElasticPartialVolume:Remark:Modulus:One},\ref{enm:Peridigm:QRG:Materials:ElasticPartialVolume:Remark:Modulus:Two}}       & -             & Engineering constant for axial stiffness\\
Poisson\verb+'+s Ratio & double & \checkmark\textsuperscript{\ref{enm:Peridigm:QRG:Materials:ElasticPartialVolume:Remark:Modulus:One},\ref{enm:Peridigm:QRG:Materials:ElasticPartialVolume:Remark:Modulus:Two}}       & -             & Engineering constant for transverse contraction\\
\end{longtable}
\end{filecontents}
% \bottomrule
% \end{tabularx}

\begingroup
\LTXtable{\linewidth}{\tabledir\jobname-parammatelasticpvltxtable.tex}
\endgroup

\levelstay{Remarks}

\begin{enumerate}[noitemsep]
  \item \label{enm:Peridigm:QRG:Materials:ElasticPartialVolume:Remark:Modulus:One} The stiffness can be defined by either elastic modulus combination: Volumetric and shear elasticity ($\glssymbol{symb:scalar:mat:modulus:bulk}$,$\glssymbol{symb:scalar:mat:modulus:shear}$) or the engineering constants ($\glssymbol{symb:scalar:mat:modulus:young}$,$\glssymbol{symb:scalar:mat:poissonratio}$,$\glssymbol{symb:scalar:mat:modulus:shear}$)
  \item \label{enm:Peridigm:QRG:Materials:ElasticPartialVolume:Remark:Modulus:Two} In case engineering constants are used, only two of the three values \textit{Young's Modulus}, \textit{Poisson's Ratio} and \textit{Shear Modulus} have to be specified. The missing value is calculated from
  \begin{align*}
  \glssymbol{symb:scalar:mat:modulus:shear}&=\dfrac{\glssymbol{symb:scalar:mat:modulus:young}}{2\cdot\left(1+\glssymbol{symb:scalar:mat:poissonratio}\right)}
  \end{align*}
  Internally, the engineering constants are converted to ($\glssymbol{symb:scalar:mat:modulus:bulk}$,$\glssymbol{symb:scalar:mat:modulus:shear}$).
  \item Consider the general remarks on non-correspondence materials in section \ref{sec:Peridigm:QRG:Materials:Preliminaries:NonCorrespondence}
\end{enumerate}

\levelup{Exemplary input section}

\leveldown{XML-format}

-

\levelstay{Free format}

\begingroup
\lstset{breaklines=true}
\begin{code}
Materials
  LPS Material
    Material Model "Elastic Partial Volume"
    Density 8.05
    Bulk Modulus 160.0e10
    Shear Modulus 79.3e10
\end{code}
\endgroup

\levelstay{YAML format}

-
  
\levelup{Additional output variables for the material model}

% \begin{multicols}{2}
% \begin{itemize}[noitemsep]
%   \item Deviatoric\_Normalization
%   \item Dilatation
%   \item Dilatation\_Normalization
%   \item Lagrange\_Multiplier\_Dilatation\_{1-6}
% \end{itemize}
% \end{multicols}

\levelstay{List of examples}

\begin{itemize}[noitemsep]
%   \item From \texttt{examples/}:
%   \begin{itemize}[noitemsep]
%     \item \texttt{fragmenting\_cylinder/fragmenting\_cylinder.peridigm}
%   \end{itemize}
%   \item From \texttt{test/regression/}:
%   \begin{itemize}[noitemsep]
%     \item \texttt{WaveInBar/WaveInBar.xml}
%   \end{itemize}
  \item From \texttt{test/verification/}:
  \begin{itemize}[noitemsep]
    \item \texttt{LinearLPSBar/LinearLPSBar.peridigm}
  \end{itemize}
\end{itemize}