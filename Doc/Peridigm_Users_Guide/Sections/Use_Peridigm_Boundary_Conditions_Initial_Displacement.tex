%%%%%%%%%%%%%%%%%%%%%%%%%%%%%%%%%%%%
% Header                           %
%%%%%%%%%%%%%%%%%%%%%%%%%%%%%%%%%%%%
% 
% Revisions: 2017-04-10 Martin R�del <martin.raedel@dlr.de>
%                       Initial draft
%               
% Contact:   Martin R�del,  martin.raedel@dlr.de
%            DLR Composite Structures and Adaptive Systems
%          
%                                 __/|__
%                                /_/_/_/  
%            www.dlr.de/fa/en      |/ DLR
%
%%%%%%%%%%%%%%%%%%%%%%%%%%%%%%%%%%%%
% Content                          %
%%%%%%%%%%%%%%%%%%%%%%%%%%%%%%%%%%%%

\levelup{\idxPDKwInitialDisplacement}
\label{sec:Peridigm:QRG:BoundaryConditions:InitialDisplacement}
\myindex[\idxPDKeywordName]{\idxPDKwInitialDisplacement}
\myindex[\idxPDKeywordName]{\idxPDKwBoundaryConditions!\idxPDKwInitialDisplacement|see{\idxPDKwInitialDisplacement}}

\leveldown{Description}

Application of a displacement at time $\glssymbol{symb:scalar:time}=0$ to the application region. For $\glssymbol{symb:scalar:time}>0$ this boundary condition is inactive.

\levelstay{Code}

\begin{itemize}[noitemsep]
  \item \verb+/src/core/Peridigm_BoundaryAndInitialConditionManager.cpp+
  \item \verb+/src/core/Peridigm_BoundaryAndInitialConditionManager.hpp+
  \item \verb+/src/core/Peridigm_BoundaryCondition.cpp+
  \item \verb+/src/core/Peridigm_BoundaryCondition.hpp+
  \item \verb+/src/core/Peridigm_Enums.cpp+
  \item \verb+/src/core/Peridigm_Enums.hpp+
\end{itemize}

\levelstay{Input parameters}

\paragraph{List}

\begin{tabularx}{\linewidth}{lcccX}
\toprule
Name		& Type		& Required	& Default	& Description		\\
\midrule
Type		& string	& \checkmark	& -		& ``\idxPDKwInitialDisplacement''	\\
Node Set	& string	& \checkmark\textsuperscript{\ref{enm:Peridigm:QRG:BoundaryConditions:InitialDisplacement:Remark:NodeSet}}	& -		& Application region name | ``Full Domain'' | ``All Sets'' 	\\
Coordinate	& string	& \checkmark	& -		& ``x'' | ``y'' | ``z'' 	\\
Value		& string	& \checkmark\textsuperscript{\ref{enm:Peridigm:QRG:BoundaryConditions:InitialDisplacement:Remark:Value}}	& -		& String with function for function parser	\\
\bottomrule
\end{tabularx}

\paragraph{Remarks}

\begin{enumerate}[noitemsep]
  \item It is not totally clear what happens when a prescribed displacement and an initial displacement/velocity are applied to the same application region. Who wins?. See \verb+BoundaryCondition.cpp+ lines 136-138.
  \item \label{enm:Peridigm:QRG:BoundaryConditions:InitialDisplacement:Remark:NodeSet} Any string that is not ``Full Domain'' or ``All Sets'' will be considered a custom node set name in the model or mesh file, dependent of the discretization type.
  \item \label{enm:Peridigm:QRG:BoundaryConditions:InitialDisplacement:Remark:Value} The string in the variable \textit{Value} should start with \verb+value = +. If it does not, \toolname{} will automatically add it for the function parser to work.
\end{enumerate}

\levelstay{Exemplary input section}

\leveldown{XML format}

\begin{code}
<ParameterList name="Initial Displacement Min X Face">
  <Parameter name="Type" type="string" value="Initial Displacement"/>
  <Parameter name="Node Set" type="string" value="Min X Node Set"/>
  <Parameter name="Coordinate" type="string" value="x"/>
  <Parameter name="Value" type="string" value="0.1"/>
</ParameterList>
\end{code}

\levelstay{Free format}

-

\levelstay{YAML format}

-

\levelup{List of examples}

\begin{itemize}[noitemsep]
%   \item From \verb+examples/+:
%   \begin{itemize}[noitemsep]
%     \item \texttt{examples/tensile\_test/tensile\_test.peridigm}
%   \end{itemize}
%   \item From \texttt{test/regression/}:
%   \begin{itemize}[noitemsep]
%     \item \texttt{Contact\_Cubes/Contact\_Cubes.xml}
%   \end{itemize}
  \item From \texttt{test/verification/}:
  \begin{itemize}[noitemsep]
    \item \texttt{CompressionInitDisp\_2x1x1/CompressionInitDisp\_2x1x1.xml}
    \item \texttt{CompressionImplicitEssentialBC\_2x1x1/CompressionImplicitEssentialBC\_2x1x1.xml}
  \end{itemize}
\end{itemize}