%%%%%%%%%%%%%%%%%%%%%%%%%%%%%%%%%%%%
% Header                           %
%%%%%%%%%%%%%%%%%%%%%%%%%%%%%%%%%%%%
% 
% Revisions: 2017-04-10 Martin R�del <martin.raedel@dlr.de>
%                       Initial draft
%               
% Contact:   Martin R�del,  martin.raedel@dlr.de
%            DLR Composite Structures and Adaptive Systems
%          
%                                 __/|__
%                                /_/_/_/  
%            www.dlr.de/fa/en      |/ DLR
%
%%%%%%%%%%%%%%%%%%%%%%%%%%%%%%%%%%%%
% Content                          %
%%%%%%%%%%%%%%%%%%%%%%%%%%%%%%%%%%%%

\levelup{Supported operations}
\label{sec:Peridigm:Basics:FunctionParser:Operations}
\myindex[\idxPDKeywordName]{\idxPDKwFunctionParser!Operations}

The RTC language can be thought of as a small subset of the C language with a couple minor modifications. The following information is repeated from the RTCompiler \href{https://github.com/trilinos/Trilinos/blob/master/packages/pamgen/rtcompiler/rtclang.tex}{source code documentation}.

% \begin{table}[htbp]
% \begin{tabularx}{\linewidth}{lX}
% \toprule
% Group         & Supported operations  \\
% \midrule
% 
% \bottomrule
% \end{tabularx}
% \end{table}

\leveldown{Operators}

The RTC language has the following operators that work exactly as they do in C
and have the same precedence as they do in C:

\begin{multicols}{2}
\begin{itemize}[noitemsep]
  \item $+$     \tabto{1cm} Addition
  \item $-$     \tabto{1cm} Subtration
  \item $-$     \tabto{1cm} Negation
  \item $*$     \tabto{1cm} Multiplication
  \item $/$     \tabto{1cm} Division
  \item $\hat{}$\tabto{1cm} Exponentiation\footnote{This operator does not occur in the C language, but is added into the RTC language for convenience.}
  \item $==$    \tabto{1cm} Equality
  \item $>$     \tabto{1cm} Greater than
  \item $<$     \tabto{1cm} Less than
  \item $>=$    \tabto{1cm} Greater than or equal to
  \item $<=$    \tabto{1cm} Less than or equal to
  \item $=$     \tabto{1cm} Assignment
  \item $||$    \tabto{1cm} Logical or
  \item $\&\&$  \tabto{1cm} Logical and
  \item $!=$    \tabto{1cm} Inequality
  \item $\%$    \tabto{1cm} Modulo
  \item $!$     \tabto{1cm} Logical not
\end{itemize}
\end{multicols}

\levelstay{Control flow}

The RTC language has the following control flow statements:

\begin{itemize}[noitemsep]
  \item for( expr ; expr ; expr ) \{ ... \}
  \item while( expr )  \{ ... \}
  \item if (expr) \{...\}
  \item else if (expr) \{...\}
  \item else \{...\}
\end{itemize}

These control flow statements work exactly as they do in C except that the code blocks following a control flow statement \textbf{MUST} be enclosed within braces even if the block only consists of one line.

\levelstay{Line Structure}

The line structure in the RTC language is the same as that of C. Expressions
end with a semicolon unless they are inside a control flow statement.

According to \href{https://github.com/peridigm/peridigm/commit/1e625c0ca30a8be28a0fb970016baf5c820018ea}{this commit} the .peridigm input deck format (free format) does not support multi-line entries.

\levelstay{Math}

The following math.h functions are available in RTC. \toolname{} also seems to support the \href{https://cubit.sandia.gov/public/15.2/help_manual/WebHelp/cubithelp.htm#appendix/aprepro/aprepro.htm}{APREPRO} functionalities. The features are reported in \cite{SjaardemaGD2013}.

\begin{filecontents}{\tabledir\jobname-mathfunctionsltxtable.tex}
\begin{longtable}{@{}lX@{}}
\caption{Math commands for \protect\toolname{} function parser}\\
\toprule
Syntax          & Description   \\
\midrule
\endfirsthead
%
\multicolumn{2}{@{}l}{\ldots continued}\\
\toprule
Syntax          & Description   \\
\midrule
\endhead
%
\bottomrule
\multicolumn{2}{r@{}}{continued \ldots}\\
\endfoot
\bottomrule
\endlastfoot
%
abs(x)                  & Absolute value of x                                   \\
acos(x)                 & Inverse cosine of x, returns radians                  \\
acosd(x)                & Inverse cosine of x, returns degrees                  \\
acosh(x)                & Inverse hyperbolic cosine of x                        \\
asin(x)                 & Inverse sine of x, returns radians                    \\
asind(x)                & Inverse sine of x, returns degrees                    \\
asinh(x)                & Inverse hyperbolic sine of x                          \\
atan(x)                 & Inverse tangent of x, returns radians                 \\
atan2(x,y)              & Inverse tangent of y/x, returns radians               \\
atan2d(x,y)             & Inverse tangent of y/x, returns degrees               \\
atand(x)                & Inverse tangent of x, returns degrees                 \\
atanh(x)                & Inverse hyperbolic tangent of x                       \\
ceil(x)                 & Smallest integer not less than x                      \\
cos(x)                  & Cosine of x, with x in radians                        \\
cosd(x)                 & Cosine of x, with x in degrees                        \\
cosh(x)                 & Hyperbolic cosine of x                                \\
d2r(x)                  & Degrees to radians                                    \\
dim(x,y)                & $x-\min{x,y}$                                         \\
dist(x1,y1,x2,y2)       &$\sqrt{\left(x_1-x_2\right)^2+\left(y_1-y_2\right)^2}$ \\
erf(x)                  & Error function                                        \\
erfc(x)                 & Complementary error function  $(1.0 - \erf(x))$               \\
exp(x)                  & Exponential $e^x$                                     \\
fabs(x)                 & returns the absolute value of x                       \\
floor(x)                & Largest integer not greater than x                    \\
fmod(x,y)               & Floating-point remainder of x/y                       \\
gamma(x)                & returns $\Gamma(x)$                                   \\
hypot(x,y)              & $\sqrt{x^2+y^2}$                                      \\
int(x), [x]             & Integer part of x truncated toward 0                  \\
j0(x)                   & Bessel function of order zero                         \\
j1(x)                   & Bessel function of order one                          \\
i0(x)                   & Modified Bessel function of order zero                \\
i1(x)                   & Modified Bessel function of order one                 \\
julday(mm,dd,yy)        & Julian day corresponding to mm/dd/yy                  \\
juldayhms(mm,dd,yy,hh,mm,ss)&Julian day corresponding to mm/dd/yy at hh:mm:ss   \\
lgamma(x)               & $\log\left(\Gamma(x)\right)$                          \\
ln(x)                   & Natural (base e) logarithm of x                       \\
log(x)                  & Natural (base e) logarithm of x                       \\
log10(x)                & Base 10 logarithm of x                                \\
log1p(x)                & $\log(1+x)$ Accurate even for very small values of x  \\
max(x,y)                & Maximum of x and y                                    \\
min(x,y)                & Minimum of x and y                                    \\
nint(x)                 & Rounds x to nearest integer. <0.5 down; >= 0.5 up     \\
polarX(r,a)             & $r\cdot\cos(a)$, $a$ is in degrees                    \\
polarY(r,a)             & $r\cdot\sin(a)$, $a$ is in degrees                    \\
pow(b, e)               & returns b to the e power                              \\
                        & (Note: you can use the Exponentiation operator instead)\\
r2d(x)                  & Radians to degrees                                    \\
rand(xl,xh)             & Random value between xl and xh; uniformly distributed \\
rand\_lognormal(m,s)    & Random value with lognormal distribution with mean m and stddev s             \\
rand\_normal(m,s)       & Random value normally distributed with mean
m and stddev s          \\
rand\_weibull(a, b)     & Random value with weibull distribution with $\alpha=a$ and $\beta=b$          \\
sign(x,y)               & $x\cdot\sign(y)$                                      \\
sin(x)                  & Sine of x, with x in radians                          \\
sind(x)                 & Sine of x, with x in degrees                          \\
sinh(x)                 & Hyperbolic sine of x                                  \\
sqrt(x)                 & Square root of x                                      \\
srand(seed)             & Seed the random number generator with the given integer value. At  the  beginning  of Aprepro execution, srand() is  called  with the current time as the seed.               \\
strtod(svar)            & Returns a double-precision floating-point number equal to the value represented by the character string pointed to by svar.           \\
tan(x)                  & Tangent of x, with x in radians                       \\
tand(x)                 & Tangent of x, with x in radians                       \\
tanh(x)                 & Hyperbolic tangent of x                               \\
Vangle(x1,y1,x2,y2)     & Angle (radians) between vector $x_1\hat{i}+y_1\hat{j}$ and $x_2\hat{i}+y_2\hat{j}$            \\
Vangled(x1,y1,x2,y2)    & Angle (degrees) between vector $x_1\hat{i}+y_1\hat{j}$\\
word\_count(svar,del)   & Number of words in svar. Words are separated by one or more of the characters in the string variable del.\\
\end{longtable}
\end{filecontents}

\begingroup
% \renewcommand{\arraystretch}{1.2}
\LTXtable{\linewidth}{\tabledir\jobname-mathfunctionsltxtable.tex}
\endgroup