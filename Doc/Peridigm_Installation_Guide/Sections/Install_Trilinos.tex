%%%%%%%%%%%%%%%%%%%%%%%%%%%%%%%%%%%%
% Header                           %
%%%%%%%%%%%%%%%%%%%%%%%%%%%%%%%%%%%%
% 
% Revisions: 2017-04-10 Martin Rädel <martin.raedel@dlr.de>
%                       Initial draft
%               
% Contact:   Martin Rädel,  martin.raedel@dlr.de
%            DLR Composite Structures and Adaptive Systems
%          
%                                 __/|__
%                                /_/_/_/  
%            www.dlr.de/fa/en      |/ DLR
% 
%%%%%%%%%%%%%%%%%%%%%%%%%%%%%%%%%%%%
% Content                          %
%%%%%%%%%%%%%%%%%%%%%%%%%%%%%%%%%%%%

\levelstay{\texorpdfstring{\protect\marktool{\trilinosname{}}}{\trilinosname{}}}

The Trilinos Project is an effort to develop algorithms and enabling technologies within an object-oriented software framework for the solution of large-scale, complex multi-physics engineering and scientific problems. A unique design feature of Trilinos is its focus on packages. For more information visit

\href{\trilinosaddress}{\trilinosaddress}

A number of \marktool{\trilinosname} packages are required by \marktool{\toolname}. The \marktool{\trilinosname} source code distribution includes the full set of \marktool{\trilinosname} packages, each of which may be activated or deactivated using \marktool{\cmakename} build options, as described below. It is recommended that Makefiles be created by running \verb+cmake+ from the command line, as opposed to using the \verb+ccmake+ GUI.

The current release of \marktool[\trilinosaddress]{\trilinosname} can be obtained from the download section of the \marktool[\trilinosaddress]{\trilinosname} homepage. The download needs a short registration with a valid email-address. The download link is likely to be not reachable without the registration.

\begingroup
\lstset{breaklines=true}
\begin{code}
cd $DOWNLOAD_DIR
wget http://trilinos.csbsju.edu/download/files/trilinos-12.4.2-Source.tar.gz
tar xvfz trilinos-12.4.2-Source.tar.gz
\end{code}
\endgroup

\marktool{\trilinosname} does not allow the use of the directory with the source-files for the further progress of the installation. Therefore, create a new folder

\begin{code}
mkdir trilinos-12.4.2
\end{code}

and copy the file from section \ref{sec:Build-script_Trilinos} to the new folder. Change the line for the \marktool{\openmpiname}- and the \marktool{\trilinosname}-source-directory (last line) if necessary.

In order to use the script make it executable as described in section \ref{sec:Build-script_Executable}. Open a terminal as root, change directory to the created path and execute it with

\begin{code}
./cmake_trilinos.cmake > cmakeopts.log 2>&1
\end{code}

Once \marktool{\trilinosname} has been successfully configured, it can be compiled and installed as follows:

\begin{code}
make -j 4
\end{code}

If there occur any errors during the compilation of \marktool{\trilinosname} visit section \ref{sec:FAQ}. For comiling with \verb+make -j 4+ more than 8GB of RAM are necessary. If you do not know if there were any compilation errors have occured due to the long duration of the compilation process, repeat

\begin{code}
make
\end{code}

after the original compilation with \verb+make -j 4+. Only failed compilations are repeated. Afterwards perform

\begin{code}
make install
\end{code}

The final installation can be found in the folder specified in \verb+CMAKE_INSTALL_PREFIX:PATH+ in the script from section \ref{sec:Build-script_Trilinos}.

Afterwards, \marktool{\trilinosname} has to be added to the \verb+PATH+ and \verb+LD_LIBRARY_PATH+ environment variables to later use the \marktool{\trilinosname} decomposition features for the model decomposition for calculation on multiple processors.

\begingroup
\lstset{breaklines = true}
\lstinputlisting[
  style=scriptstyle,
  linerange={55-56},
  nolol,
  ]{Scripts/modified_bashrc.txt}
\endgroup 