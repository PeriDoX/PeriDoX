%%%%%%%%%%%%%%%%%%%%%%%%%%%%%%%%%%%%
% Header                           %
%%%%%%%%%%%%%%%%%%%%%%%%%%%%%%%%%%%%
% 
% Revisions: 2017-04-10 Martin Rädel <martin.raedel@dlr.de>
%                       Initial draft
%               
% Contact:   Christian Willberg,  christian.willberg@dlr.de
%            DLR Composite Structures and Adaptive Systems
%          
%                                 __/|__
%                                /_/_/_/  
%            www.dlr.de/fa/en      |/ DLR
% 
%%%%%%%%%%%%%%%%%%%%%%%%%%%%%%%%%%%%
% Content                          %
%%%%%%%%%%%%%%%%%%%%%%%%%%%%%%%%%%%%

\levelstay{Own damage models}

Peridigm is programmed in C++. All the essential vectors and matrices are stored as pointer. It must be noted that no Voigt notation is used. Therefore, the stress and strain matrices are stored in a vector of length 9.
%A sketch of dependencies is shown in \RefFig{fig:MaterialDependencies}.
The file Peridigm\_DamageFactory allows the definition of the material. Here, the name in the .xml datasheet and the corresponding C++ file are defined.
