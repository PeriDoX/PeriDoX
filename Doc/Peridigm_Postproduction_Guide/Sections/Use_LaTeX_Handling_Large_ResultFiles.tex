\levelstay{Handling large result files in \protect\LaTeX}

\leveldown{Increase \TeX{}-memory}      \label{sec:Increase_TeX_Memory}

For large figures or plots of results using \verb+pgfplots+ the initial \TeX{} memory might be insufficient. A first possibility to solve that problem is to increase the \TeX{} memory.

\leveldown{MiKTeX on Windows}

\begin{itemize}[noitemsep]
\item Open a command prompt (cmd)
\item Type \colorbox{verbgray}{\lstinline[style=inlinetexstyle]+initexmf --edit-config-file pdflatex+} and press Enter
\item A file pops up in the editor
\item Add \colorbox{verbgray}{\lstinline[style=inlinetexstyle]+main_memory=8000000+} to the file
\item Save the file
\item Go back to the command prompt
\item Type \colorbox{verbgray}{\lstinline[style=inlinetexstyle]+initexmf --dump=pdflatex+} and press Enter
\item After completion, close the command prompt
\end{itemize}

\levelstay{TeXLive on Linux}

Untested:

\begin{itemize}[noitemsep]
\item Open a terminal
\item Type \colorbox{verbgray}{\lstinline[style=inlinetexstyle]+kpsewhich texmf.cnf+} and press Enter
\item This gives the path to the \verb+texmf.cnf+ file
\item Open the file
\item Change or add \colorbox{verbgray}{\lstinline[style=inlinetexstyle]+main_memory=8000000+} to the file
\item Save the file
\item Go back to the command prompt
\item Type \colorbox{verbgray}{\lstinline[style=inlinetexstyle]+sudo fmtutil-sys --all+} and press Enter
\item After completion, close the command prompt
\end{itemize}

Be aware that using this method you modify the official config file which will be overwritten in an update. Thus, you have to repeat this step after an update. A better and update-independent solution is to create a local \verb+texmf.cnf+-file and register it in the \TeX{} path variables

\levelup{Externalize TikZ- \& pgfplots-figure creation} \label{sec:Externalize_Figure_Creation}

Creation of \lstinline[style=inlinetexstyle]+TikZ+ or \lstinline[style=inlinetexstyle]+pgfplotsset+-figures can be quite memory and time consuming. The creation of these figures can be externalized to save time. This means the figure is compiled once and written to a graphics file (pdf, png, ...). The next \lstinline[style=inlinetexstyle]+pdflatex+ run re-uses the created figure. It is not re-created for each compilation.

To achieve the externalization you need to add the following code to your preamble in addition to \lstinline[style=inlinetexstyle]+\usepackage{pgfplots}+:

\begingroup
\lstset{breaklines = true}
\lstinputlisting[
  style=texstyle,
  caption=TikZ externalization code,
  label=lst:TikZ_externalize,
  ]{\peridoccommonpath/Preamble/Pref_Packages_TikZ_Externalization}
\endgroup

\ifpdf
You can \textattachfile[author=raed_ma, color=0 0 1]{\peridoccommonpath/Preamble/Pref_Packages_TikZ_Externalization}{download} the file from within this document.
\fi

Using one of the last two lines of the \verb+tikzset+ converts the pdf file of the figure created by \verb+pdflatex+ to a png file. It requires the installation of \href{http://www.imagemagick.org/script/index.php}{ImageMagick} which provides the \verb+convert+ command for ImageMagick version prior to 7 or \verb+magick+ for version 7 or newer. For this to work, the path to the ImageMagick binaries \verb+convert.exe+ or \verb+magick.exe+ must be added to the system \texttt{PATH} environment variable.

In case you want to externalize the creation of a single figure add the following:

\begin{texcode}
\tikzexternalenable                     �\lstcomment{\% Enable externalization}�
\tikzsetnextfilename{Chart_Example}     �\lstcomment{\% Figure file name}�

\begin{tikzpicture}[%
  spy using outlines={%
    rectangle,                                % Spy form
    rounded corners,                          % Spy edge shape
    magnification=4,                          % Magnification factor
    height=0.6cm,                             % Height in original chart
    width=1.25cm,                             % Width in original chart
    dashed,                                   % Dashed line
    draw=gray,                                % Line color
    every spy on node/.append style={thick},  % Line style for spy
    connect spies,                            % Connect orig. & detail
    spy connection path={                     % Line style for spy connection
      \draw[thick] (tikzspyonnode) -- (tikzspyinnode);
    },
  }
]
  \begin{axis}[
    chart style,
    xlabel={Time [ms]},
    ylabel={Displacement [mm]},
  ]
    % Graph 1
    \addplot[
      gray,                                   % Plot color
      xaxis style,                            % Predefined style
    ] table [
      x expr=\thisrowno{0}*1000,              % Scale to ms
      y expr=\thisrowno{1}*1000,              % Scale to mm
    ]{Results/coord_disp_pd_nt100.txt};       % Input file
    \addlegendentry{Peridynamics}             % Legend entry
    % Graph 2
    \addplot[
      black,                                  % Plot color
      xaxis style,                            % Predefined style
    ] table [
      x expr=\thisrowno{0}*1000,              % Scale to ms
      y expr=\thisrowno{2}*1000,              % Scale to mm
    ]{Results/coord_disp_pd_nt100.txt};       % Input file
    \addlegendentry{Analytical}               % Legend entry
    % Spy 1
    \coordinate (spypoint1) at (axis cs:1.59,0.5);
    \coordinate (spyviewer1) at (axis cs:0.75,0.675);
    \spy[
      width=2.0cm,
      height=1.0cm,
    ] on (spypoint1) in node [
      fill=white,
      draw=gray,
      dashed
    ] at (spyviewer1);
    % Spy 2
    \coordinate (spypoint2) at (axis cs:4.76,0.5);
    \coordinate (spyviewer2) at (axis cs:3.92,0.675);
    \spy[
      width=2.0cm,
      height=1.0cm,
    ] on (spypoint2) in node [
      fill=white,
      draw=gray,
      dashed
    ] at (spyviewer2);
  \end{axis}
\end{tikzpicture}
     �\lstcomment{\% Path to figure}�
\tikzexternaldisable                    �\lstcomment{\% Disable externalization}�
\end{texcode}

Beware, if you make changes to a figure you have to delete the present figure file before compiling your document. Otherwise, the update will not be effective because the old figure is re-used.