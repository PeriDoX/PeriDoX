%%%%%%%%%%%%%%%%%%%%%%%%%%%%%%%%%%%%
% Header                           %
%%%%%%%%%%%%%%%%%%%%%%%%%%%%%%%%%%%%
% 
% Revisions: 2017-04-10 Martin R�del <martin.raedel@dlr.de>
%                       Initial draft
%               
% Contact:   Martin R�del,  martin.raedel@dlr.de
%            DLR Composite Structures and Adaptive Systems
%          
%                                 __/|__
%                                /_/_/_/  
%            www.dlr.de/fa/en      |/ DLR
% 
%%%%%%%%%%%%%%%%%%%%%%%%%%%%%%%%%%%%
% Content                          %
%%%%%%%%%%%%%%%%%%%%%%%%%%%%%%%%%%%%

\levelup{Increase timestep in explicit solver}

The critical time step calculation is performed in the  \texttt{PeridigmNS::ComputeCriticalTimeStep} method in the class \texttt{/src/core/Peridigm\_CriticalTimeStep.cpp}. The calculation is carried out for each block as requested in \texttt{PeridigmNS::Peridigm::executeExplicit} of \texttt{src/core/Peridigm.cpp}. The minimum of all block values is taken as the final critical time step.

The equation used for the determination of the critical time step is

\begin{align}
\glssymbol{symb:scalar:timestep}_{\mbox{Block}} &= \sqrt{\dfrac{2\glssymbol{symb:scalar:mat:density}}{\glssymbol{symb:scalar:timestep}_{\mbox{Denominator}}}}
\end{align}

The time step denominator itself is a function of the collocation point volume,  neighbor distance and ``bond'' stiffness. Despite implementing the state-based peridynamics, \marktool{\toolname} uses the bond constant for 3D bond-based peridynamics from \autoref{tab:Peridigm:QRG:Preliminaries:PDCMConversion} as the the ``bond'' stiffness. Thus

\begin{align}
\glssymbol{symb:scalar:timestep}_{\mbox{Denominator}} &= f\left(
\glssymbol{symb:operator:Delta}x,
\glssymbol{symb:scalar:mat:modulus:bulk},
%\dfrac{1}{\glssymbol{symb:scalar:pd:horizon}^{4}}
\glssymbol{symb:scalar:pd:horizon}^{-4}
\right)
\end{align}

The following actions can be taken to influence the time step:

\begin{itemize}[noitemsep]
  \item Increase material density
    \begin{itemize}[noitemsep]
      \item For quasi-static simulations a certain amount is possible
      \item Reduces oscillation effects due to increased mass intertia
      \item Limit is aperiodic boundary case
    \end{itemize}
  \item Increase element size:
  \begin{itemize}[noitemsep]
    \item A larger element size increases the volume associated with each peridynamic collocation point
    \item On the other hand the initial distance between points is increased
    \item Thus, do not expect a linear response of the time step for changing mesh sizes
  \end{itemize}
  \item Increase horizon
\end{itemize}