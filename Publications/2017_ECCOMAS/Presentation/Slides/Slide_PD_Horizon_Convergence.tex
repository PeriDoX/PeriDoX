\begin{frame}[t]{\secname}{\subsecname}
  
\begin{columns}[t]
  \begin{column}{0.55\textwidth}
    \begin{itemize}
      \item<1-> Ideally continuous \& homogeneous structure 
      \item<2-> Standard local theory applies until failure
      %\item CM PDEs not defined at discontinuities
      \item<3-> PD as way to model damage and fracture
      %\item Without expectation of long-range effects
      \only<4->{
        \item Questions:
          \begin{itemize}%[noitemsep]
            \item Choice of discretization?
            \item Choice of horizon?
            \item Convergence?
            \item Identical for different base discretizations?
          \end{itemize}
      }

%         \begin{itemize}
%           \item Exploitation of FRP lightweight potential limited
%           \item Missing reliability of failure predictions
%         \end{itemize}
%       \item Goals:
%         \begin{itemize}
%           \item Increase understanding of failure mechanisms
%           \item Derive improved failure criteria for preliminary design
%         \end{itemize}
%         
    \end{itemize}
  \end{column}
  \begin{column}{0.45\textwidth}
    \only<4->{
      \begin{figure}[htbp]
        \centering
        % Variables
        \def\x{2.5}
        \def\y{3}
        \newlength{\labeldistance}
        \setlength{\labeldistance}{0.5em}
        % Figure
        %\tikzexternalenable
        %\tikzsetnextfilename{Fig_Thry_PD_Convergence}
        \begingroup
          \figurefontsize
          %%%%%%%%%%%%%%%%%%%%%%%%%%%%%%%%%%%%
% Header                           %
%%%%%%%%%%%%%%%%%%%%%%%%%%%%%%%%%%%%
% 
% Revisions: 2017-04-10 Martin Raedel <martin.raedel@dlr.de>
%                       Initial draft
%               
% Contact:   Martin Raedel,  martin.raedel@dlr.de
%            DLR Composite Structures and Adaptive Systems
%          
%                                 __/|__
%                                /_/_/_/  
%            www.dlr.de/fa/en      |/ DLR
% 
%%%%%%%%%%%%%%%%%%%%%%%%%%%%%%%%%%%%
% Content                          %
%%%%%%%%%%%%%%%%%%%%%%%%%%%%%%%%%%%%

\begin{tikzpicture}
  % nodes
  \node (dnl) at ( 0,  0) {$u_{\delta}^h$};
  \node (cnl) at ( 0,-\y) {$u_{\delta}^0$};
  \node (dl)  at (\x,  0) {$u_0^h$};
  \node (cl)  at (\x,-\y) {$u_0^0$};
  % node labels
  \node [ left=\labeldistance of dnl,anchor=east,align=center] (dnllabel) {Discrete\\Nonlocal};
  \node [ left=\labeldistance of cnl,anchor=east,align=center] (cnllabel) {Continuum\\Nonlocal};
  \node [right=\labeldistance of  dl,anchor=west,align=center] (dllabel)  {Discrete\\Local};
  \node [right=\labeldistance of  cl,anchor=west,align=center] (cllabel)  {Continuum\\Local PDE};
  % lines
  \draw (dnl) -- (dl)  node [midway,above] {$\delta\longrightarrow0$};
  \draw (cnl) -- (cl)  node [midway,below] {$\delta\longrightarrow0$};
  \draw (dnl) -- (cnl) node [midway,sloped,below] {$h\longrightarrow0$};
  \draw (dl)  -- (cl)  node [midway,sloped,above] {$h\longrightarrow0$};
  % coordinates inside
  \coordinate (ul) at ($ (dnl) !.125! (cl) $);
  \coordinate (lr) at ($ (dnl) !.875! (cl) $);
%     \node[circle,fill=black,minimum size=2pt] at (ul){};
%     \node[circle,fill=black,minimum size=2pt] at (lr){};
  
  % Arrows inside
  \draw[dotted,shorten >=.2em,-latex] (ul) -- (lr) node [midway,sloped,below=1ex] {$\delta\longrightarrow0$} node [midway,sloped,above=1ex] {$h\longrightarrow0$};
  %\draw[dotted,shorten >=.2em,-latex] (ul) -- (lr) coordinate [midway] (hdc);
  
  \draw[dotted,shorten >=.2em,-latex] (ul) to [bend right=40] (lr);% coordinate [midway] (dc);
  \draw[dotted,shorten >=.2em,-latex] (ul) to [bend left =40] (lr);% coordinate [midway] (hc);
  
  %\node at ($ (hdc) !.5! (dc) $) {$\delta\rightarrow0$};
  %\node at ($ (hdc) !.5! (hc) $) {$h\rightarrow0$};
\end{tikzpicture}
        \endgroup
        %\tikzexternaldisable
        \caption{Types of convergence in PD \cite{BobaruF2017}}
        \label{fig:Peridynamic_convergence}
      \end{figure}
    }
  \end{column}
\end{columns}

\end{frame}