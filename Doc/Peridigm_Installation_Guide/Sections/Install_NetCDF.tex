%%%%%%%%%%%%%%%%%%%%%%%%%%%%%%%%%%%%
% Header                           %
%%%%%%%%%%%%%%%%%%%%%%%%%%%%%%%%%%%%
% 
% Revisions: 2017-04-10 Martin Rädel <martin.raedel@dlr.de>
%                       Initial draft
%               
% Contact:   Martin Rädel,  martin.raedel@dlr.de
%            DLR Composite Structures and Adaptive Systems
%          
%                                 __/|__
%                                /_/_/_/  
%            www.dlr.de/fa/en      |/ DLR
% 
%%%%%%%%%%%%%%%%%%%%%%%%%%%%%%%%%%%%
% Content                          %
%%%%%%%%%%%%%%%%%%%%%%%%%%%%%%%%%%%%

\levelstay{\texorpdfstring{\protect\marktool{\netcdfname{}}}{\netcdfname{}}}
\label{sec:Install_netcdf}

\marktool{\netcdfname} (Network Common Data Form) is a set of software libraries and machine-independent data formats that support the creation, access, and sharing of array-oriented scientific data. Distributions are provided for Java and C/C++/Fortran. For more information visit

\href{\netcdfaddress}{\netcdfaddress}

\marktool{\netcdfname} is required by the \marktool{\trilinosname} \marktool{SEACAS} package. \marktool{\netcdfname} should be configured with the \verb+--disable-netcdf-4+ and \verb+--disable-dap+ options.

\paragraph{Installation from source} The \marktool{\netcdfname} source code is available from the \marktool[\netcdfaddress]{\netcdfname} homepage. Download the source code for your platform

\begingroup
\lstset{breaklines=true}
\begin{code}
cd $DOWNLOAD_DIR
wget ftp://ftp.unidata.ucar.edu/pub/netcdf/netcdf-4.4.0.tar.gz
tar xvfz netcdf-4.4.0.tar.gz 	%\lstcomment{\# unzip}%
cd netcdf-4.4.0			%\lstcomment{\# go into directory}%
\end{code}
\endgroup

Prior to compiling \marktool{\netcdfname}, it is recommended that you modify the file \verb+netcdf.h+ in \\\verb+$DOWNLOAD_DIR/netcdf-4.4.0/include/+ to better support large-scale \marktool{\toolname} simulations. Modify the following \verb+#define+ statements in the \verb+netcdf.h+ file. Change the values to match what is given below.

\begingroup
\lstset{keepspaces=true}
\begin{code}
#define NC_MAX_DIMS      65536
#define NC_MAX_ATTRS      8192
#define NC_MAX_VARS     524288
#define NC_MAX_NAME        256
#define NC_MAX_VAR_DIMS      8
\end{code}
\endgroup

Afterwards, create the \marktool{\netcdfname} build script as described in section \ref{sec:Build-script_NetCDF}. In order to use the script make it executable as described in section \ref{sec:Build-script_Executable}. Open a terminal as root, change directory to the created install script and execute it with

\begin{code}
./install_netcdf.sh > install.log 2>&1
\end{code}

Due to an apparent glitch in the \marktool{\netcdfname} installer, in some cases it may be necessary to manually copy the file \verb+$DOWNLOAD_DIR/netcdf-4.4.0/include/netcdf_par.h+ from the source distribution into the installation \verb+include+ subdirectory.

\begingroup
\lstset{breaklines=true}
\begin{code}
cp $DOWNLOAD_DIR/netcdf-4.4.0/include/netcdf_par.h /usr/local/bin/netcdf-4.4.0/include/
\end{code}
\endgroup

Afterwards, the \marktool{\netcdfname}-directory has to be added to the \verb+PATH+ and \verb+LD_LIBRARY_PATH+ environment variable

\begingroup
\lstset{breaklines = true}
\lstinputlisting[
  style=scriptstyle,
  linerange={51-52},
  nolol,
  ]{Scripts/modified_bashrc.txt}
\endgroup 