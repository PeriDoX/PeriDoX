%%%%%%%%%%%%%%%%%%%%%%%%%%%%%%%%%%%%
% Header                           %
%%%%%%%%%%%%%%%%%%%%%%%%%%%%%%%%%%%%
% 
% Revisions: 2017-04-10 Martin R�del <martin.raedel@dlr.de>
%                       Initial draft
%               
% Contact:   Martin R�del,  martin.raedel@dlr.de
%            DLR Composite Structures and Adaptive Systems
%          
%                                 __/|__
%                                /_/_/_/  
%            www.dlr.de/fa/en      |/ DLR
% 
%%%%%%%%%%%%%%%%%%%%%%%%%%%%%%%%%%%%
% Content                          %
%%%%%%%%%%%%%%%%%%%%%%%%%%%%%%%%%%%%

\leveldown{\idxPDKwComputeClassParameters{} for node sets}
\label{sec:Peridigm:QRG:ComputeClassParameters:Nodeset}
\myindex[\idxPDKeywordName]{\idxPDKwComputeClassParameters!\idxPDKwNodeSet}

\leveldown{Description}

Define user defined calculation data for nodesets. These quantities can later be output, see section \ref{sec:Peridigm:QRG:Output:User:Nodeset}.

\levelstay{Code}

\begin{itemize}[noitemsep]
  \item \texttt{/src/compute/Peridigm\_Compute\_Node\_Set\_Data.cpp}
  \item \texttt{/src/compute/Peridigm\_Compute\_Node\_Set\_Data.cpp}
\end{itemize}

\levelstay{Input parameters}
\label{sec:Peridigm:QRG:ComputeClassParameters:Nodeset:Parameters}

\leveldown{List}%\mbox{ }\\

\begin{tabularx}{\linewidth}{lcccX}
\toprule
Name            & Type          & Required      & Default               & Description           \\
\midrule
Compute Class   & string        & \checkmark    &                       & ``Node\_Set\_Data''   \\
Calculation Type& string        & \checkmark    & -                     & ``Minimum'' | ``Maximum'' | ``Sum''   \\
Node Set        & string        & \checkmark    & -                     & Node set name         \\
Variable        & string        & \checkmark\textsuperscript{\ref{enm:Peridigm:QRG:ComputeClassParameters:Nodeset:Remark:One},\ref{enm:Peridigm:QRG:ComputeClassParameters:Nodeset:Remark:Three}}& -    & Output variable       \\
Output Label    & string        & \checkmark\textsuperscript{\ref{enm:Peridigm:QRG:ComputeClassParameters:Nodeset:Remark:Two}} & -     & Output label for output section       \\
\bottomrule
\end{tabularx}

\levelstay{Remarks}

\begin{enumerate}[noitemsep]
  \item \label{enm:Peridigm:QRG:ComputeClassParameters:Nodeset:Remark:One} The output value identifier are defined in \texttt{src/io/mesh\_output/Field.h}. Possible values are defined in \autoref{sec:Peridigm:QRG:Output:Variables}.
  \item \label{enm:Peridigm:QRG:ComputeClassParameters:Nodeset:Remark:Two} For the output to have any effect, the output label must be defined in the ``Output'' section of the input deck.
  \item \label{enm:Peridigm:QRG:ComputeClassParameters:Nodeset:Remark:Three} %%%%%%%%%%%%%%%%%%%%%%%%%%%%%%%%%%%%
% Header                           %
%%%%%%%%%%%%%%%%%%%%%%%%%%%%%%%%%%%%
% 
% Revisions: 2017-04-10 Martin R�del <martin.raedel@dlr.de>
%                       Initial draft
%               
% Contact:   Martin R�del,  martin.raedel@dlr.de
%            DLR Composite Structures and Adaptive Systems
%          
%                                 __/|__
%                                /_/_/_/  
%            www.dlr.de/fa/en      |/ DLR
%
%%%%%%%%%%%%%%%%%%%%%%%%%%%%%%%%%%%%
% Content                          %
%%%%%%%%%%%%%%%%%%%%%%%%%%%%%%%%%%%%

The item requested as \texttt{Variable} must be defined and requested as a global \texttt{Output Variable} in the \texttt{Output} block, e.g.:
In case you request a \texttt{Compute Class Parameter} \textit{Top Reaction Force} with \textit{Variable ``Force''}:

\begin{code}
Compute Class Parameters
  Top Reaction Force
    ...
    Variable "Force"
    Output Label "Top_Reaction_Force"
\end{code}

Than \texttt{Force ``true''} must be defined in the \texttt{Output} section:

\begin{code}
Output
  Output File Type "ExodusII"
  ..
  Output Variables
    Block_Id "false"
    ...
    Force "true"
    Top_Reaction_Force "true"
\end{code}

Without \texttt{Force ``true''} an error is thrown as the group variables cannot be extracted from the global solution.
  \item \label{enm:Peridigm:QRG:ComputeClassParameters:Nodeset:Remark:Four} There seems to be a bug in the text file reader in case nodesets\myindex[\idxPDKeywordName]{\idxPDKwNodeSet} are defined by additional text files for the text file discretization from \ref{sec:Peridigm:Basics:InputFileFormat:Discretization:Textfile}\myindex[\idxPDKeywordName]{\idxPDKwTextFile}. There are two possible workarounds:
  \begin{enumerate}
    \item Use the python script \textit{\href{https://github.com/peridigm/peridigm/blob/master/scripts/text_to_genesis.py}{text\_to\_genesis.py}} to convert your text file into a genesis file. This script is in the \href{https://github.com/peridigm/peridigm/tree/master/scripts}{scripts directory of the Peridigm repo}, you probably will have to edit the \textit{PATH} at the top of the python file to point to your Trilinos lib directory so it can load \textit{exodus.py}.
    \item If possible, create an individual block for the domain of the nodeset and use the compute class parameter option for blocks from \ref{sec:Peridigm:QRG:ComputeClassParameters:Block}\myindex[\idxPDKeywordName]{\idxPDKwComputeClassParameters!\idxPDKwBlocks}
  \end{enumerate}

\end{enumerate}

\levelup{Exemplary input section}

\leveldown{XML-format}

from \texttt{test/regression/Contact\_Ring/Contact\_Ring.xml}:

\begingroup
\lstset{breaklines=true}
\begin{code}
<ParameterList name="Compute Class Parameters">
  <ParameterList name="Node Set 10 Stored Elastic Energy">
    <Parameter name="Compute Class" type="string" value="Node_Set_Data"/>
    <Parameter name="Calculation Type" type="string" value="Sum"/>
    <Parameter name="Node Set" type="string" value="nodelist_10"/>
    <Parameter name="Variable" type="string" value="Stored_Elastic_Energy"/>
    <Parameter name="Output Label" type="string" value="NS_10_Stored_Elastic_Energy"/>      
  </ParameterList>
  <ParameterList name="Node Set 11 Stored Elastic Energy">
    <Parameter name="Compute Class" type="string" value="Node_Set_Data"/>
    <Parameter name="Calculation Type" type="string" value="Sum"/>
    <Parameter name="Node Set" type="string" value="nodelist_11"/>
    <Parameter name="Variable" type="string" value="Stored_Elastic_Energy"/>
    <Parameter name="Output Label" type="string" value="NS_11_Stored_Elastic_Energy"/>      
  </ParameterList>
</ParameterList>
\end{code}
\endgroup

\levelstay{Free format}

-

\levelstay{YAML format}

-

\levelup{List of examples}

\begin{itemize}[noitemsep]
%   \item From \texttt{examples/}:
%   \begin{itemize}[noitemsep]
%     \item \texttt{examples/tensile\_test/tensile\_test.peridigm}
%   \end{itemize}
  \item From \texttt{test/regression/}:
  \begin{itemize}[noitemsep]
    \item \texttt{Contact\_Ring/Contact\_Ring.xml}
    \item \texttt{WaveInBar\_MultiBlock/WaveInBar\_MultiBlock.xml}
  \end{itemize}
%   \item From \texttt{test/verification/}:
%   \begin{itemize}[noitemsep]
%     \item \texttt{MultipleHorizons/MultipleHorizons.xml}
%   \end{itemize}
\end{itemize}
