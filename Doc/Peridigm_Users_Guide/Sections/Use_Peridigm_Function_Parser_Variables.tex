%%%%%%%%%%%%%%%%%%%%%%%%%%%%%%%%%%%%
% Header                           %
%%%%%%%%%%%%%%%%%%%%%%%%%%%%%%%%%%%%
% 
% Revisions: 2017-04-10 Martin R�del <martin.raedel@dlr.de>
%                       Initial draft
%               
% Contact:   Martin R�del,  martin.raedel@dlr.de
%            DLR Composite Structures and Adaptive Systems
%          
%                                 __/|__
%                                /_/_/_/  
%            www.dlr.de/fa/en      |/ DLR
% 
%%%%%%%%%%%%%%%%%%%%%%%%%%%%%%%%%%%%
% Content                          %
%%%%%%%%%%%%%%%%%%%%%%%%%%%%%%%%%%%%

\levelup{Variables}
\label{sec:Peridigm:Basics:FunctionParser:Variable}
\myindex[\idxPDKeywordName]{\idxPDKwFunctionParser!Variables}

\leveldown{User variables}

\leveldown{Definition}

It is possible to define your own variables for use inside a \toolname{} input deck. The variables have to be defined at the beginning of the input deck. For YAML-input this is before the line \textit{Peridigm:}. The definition starts with a hash key, followed by the variable name, the equal sign and subsequently the variable value. It is possible to perform algebraic operations inside the variable definition.

\begin{code}
#{LENGTH=1.0}
#{WIDTH=0.25}
#{NUM_ELEM_ALONG_WIDTH=5}
#{ELEMENT_SIZE=WIDTH/NUM_ELEM_ALONG_WIDTH}
#{HORIZON=0.15}
\end{code}

\levelstay{Usage}

A variable value is used inside the model definition by calling the variable name in curly braces, e.g. for the variable \textit{HORIZON} is the following example excerpt of a block definition in an input file:

\begin{code}
Blocks:
  Block_1:
    Block Names: "Block_1"
    Material: "testmaterial"
    Horizon: {HORIZON}
\end{code}

\levelup{Pre-defined variables}

There are several pre-defined variables that can be used in the input deck. They are defined in \verb+/src/core/Peridigm_BoundaryCondition.cpp+

The current list contains:

\begin{tabularx}{\linewidth}{cX}
\toprule
Variable & Description   \\
\midrule
t        & Time          \\
x        & x-position of a peridynamic collocation point in the global cartesian coordinate system at time t     \\
y        & y-position of a peridynamic collocation point in the global cartesian coordinate system at time t     \\
z        & z-position of a peridynamic collocation point in the global cartesian coordinate system at time t     \\
value    & Prefix for string keyword entries\\
\bottomrule
\end{tabularx}

\levelup{List of examples}

\begin{itemize}[noitemsep]
\item \verb+/examples/twist_and_pull/twist_and_pull.peridigm+
\end{itemize} 