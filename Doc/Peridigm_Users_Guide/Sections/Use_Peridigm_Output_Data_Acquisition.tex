%%%%%%%%%%%%%%%%%%%%%%%%%%%%%%%%%%%%
% Header                           %
%%%%%%%%%%%%%%%%%%%%%%%%%%%%%%%%%%%%
% 
% Revisions: 2017-04-10 Martin R�del <martin.raedel@dlr.de>
%                       Initial draft
%               
% Contact:   Martin R�del,  martin.raedel@dlr.de
%            DLR Composite Structures and Adaptive Systems
%          
%                                 __/|__
%                                /_/_/_/  
%            www.dlr.de/fa/en      |/ DLR
%
%%%%%%%%%%%%%%%%%%%%%%%%%%%%%%%%%%%%
% Content                          %
%%%%%%%%%%%%%%%%%%%%%%%%%%%%%%%%%%%%

\levelstay{Output data acquisition}

\paragraph{Output data acquisition basics}

The output data acquisition block starts with information about how and where to store the entities. The following entries are not mandatory:

\begin{itemize}[noitemsep]
  \item Output Format	\tab: default: binary
  \item Parallel Write	\tab: default: true
\end{itemize}

\paragraph{Output data acquisition without global scalar entities}

In case no global scalar value is required, it is sufficient to create one output in the input deck. OUTPUT PARAMETER LIST can contain any of the aforementioned entities for the .e-results file. By default the output is set to false for all entities. The result file will be \verb+FILENAME.e+.

\subparagraph{Free-format}\mbox{}\\

\begin{code}
Output
  Output File Type ExodusII
  Output Format "BINARY"
  Output Filename "FILENAME"
  Output Frequency "1"
  Parallel Write "true"
  Output Variables
    OUTPUT PARAMETER LIST, e.g.
    Displacement "true"
    Velocity "true"
\end{code}

\subparagraph{XML-format}\mbox{}\\

\begin{code}
<ParameterList name="Output">
  <Parameter name="Output File Type" type="string" value="ExodusII"/>
  <Parameter name="Output Format" type="string" value="BINARY"/>
  <Parameter name="Output Filename" type="string" value="Filename"/>
  <Parameter name="Output Frequency" type="int" value="1"/>
  <Parameter name="Parallel Write" type="bool" value="true"/>
  <ParameterList name="Output Variables">
    OUTPUT PARAMETER LIST, e.g.
    <Parameter name="Displacement" type="bool" value="true"/>
    <Parameter name="Velocity" type="bool" value="true"/>
  </ParameterList>
</ParameterList>
\end{code}

\paragraph{Output data acquisition including global scalar entities}

Two different sets of outputs have to created, one for the .e-  and one for the .h-entities. The result files will be \verb+FILENAME.e+ for \verb+Output_1+ and \verb+FILENAME.h+ for \verb+Output_2+. The output frequencies can vary between the two outputs. The non-mandatory items are left out in this example.

\subparagraph{Free-format}\mbox{}\\

\begin{code}
Output1
  Output File Type ExodusII
  Output Filename "FILENAME"
  Output Frequency "10"
  Output Variables
    .e OUTPUT PARAMETER LIST, e.g.
    Displacement "true"
    Velocity "true"

Output2
  Output File Type ExodusII
  Output Filename "FILENAME"
  Output Frequency "5"
  Output Variables
    .h OUTPUT PARAMETER LIST, e.g.
    Global_Kinetic_Energy "true"
    Global_Linear_Momentum "true"
\end{code}

\subparagraph{XML-format}\mbox{}\\

equivalent