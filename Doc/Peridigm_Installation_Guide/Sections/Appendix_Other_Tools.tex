%%%%%%%%%%%%%%%%%%%%%%%%%%%%%%%%%%%%
% Header                           %
%%%%%%%%%%%%%%%%%%%%%%%%%%%%%%%%%%%%
% 
% Revisions: 2017-04-10 Martin Rädel <martin.raedel@dlr.de>
%                       Initial draft
%               
% Contact:   Martin Rädel,  martin.raedel@dlr.de
%            DLR Composite Structures and Adaptive Systems
%          
%                                 __/|__
%                                /_/_/_/  
%            www.dlr.de/fa/en      |/ DLR
% 
%%%%%%%%%%%%%%%%%%%%%%%%%%%%%%%%%%%%
% Content                          %
%%%%%%%%%%%%%%%%%%%%%%%%%%%%%%%%%%%%

The installation is described for \marktool{\opensusename} for version 42.1.

\leveldown{\texorpdfstring{\protect\marktool{NEdit}}{NEdit}}

\marktool{NEdit}, the Nirvana editor, is a text editor and source code editor for the X Window System. For the installation of the editor \marktool{NEdit} visit:

\href{http://software.opensuse.org/download.html?project=editors\&package=nedit}{http://software.opensuse.org/download.html?project=editors\&package=nedit}

You can either choose the 1-Click-installation or add the repository and install manually. For the latter login to a terminal as \verb+root+ and type

\begingroup
\lstset{breaklines=true}
\begin{code}
zypper addrepo http://download.opensuse.org/repositories/editors/openSUSE_Leap_42.1/editors.repo
zypper refresh
zypper install nedit
\end{code}
\endgroup

% \levelstay{Script to clean tex directory}
% 
% \leveldown{Windows}
% 
% \begingroup
% \lstset{breaklines = true}
% \lstinputlisting[style=scriptstyle, language=command.com, extendedchars=true,]{clean_tex_directory.cmd}
% \endgroup

% \ifpdf
% Alternatively, you can \textattachfile[author=raed_ma, color=0 0 1]{clean_tex_directory.cmd}{download} the script from within this document.
% \fi

\levelstay{RM-\LaTeX}	\label{sec:RM_LaTeX}

Download the package via the intranet (from within the DLR network or via a VPN connection):

\href{teamsites.dlr.de/rm/latex/SitePages/Homepage.aspx}{teamsites.dlr.de/rm/latex/SitePages/Homepage.aspx}

Follow the instructions given in \verb+/doc/RM-LaTeX-Guide/RM-LaTeX-Guide.pdf+.

\leveldown{Linux}

Before using \verb+mktexlsr+ set

\begin{code}
chmod +t texmf/
\end{code}

and

\begin{code}
chmod go+w texmf/
\end{code}

Due to some problems in the RM-\LaTeX{} package meaningful use is only possible under Windows. If using with Linux do not use everything related to the package \verb+dlrsecondpage+.

\levelstay{Windows}

There should be no additional steps necessary.