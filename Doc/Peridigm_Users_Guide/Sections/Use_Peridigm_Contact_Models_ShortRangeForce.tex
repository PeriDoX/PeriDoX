%%%%%%%%%%%%%%%%%%%%%%%%%%%%%%%%%%%%
% Header                           %
%%%%%%%%%%%%%%%%%%%%%%%%%%%%%%%%%%%%
% 
% Revisions: 2017-04-10 Martin R�del <martin.raedel@dlr.de>
%                       Initial draft
%               
% Contact:   Martin R�del,  martin.raedel@dlr.de
%            DLR Composite Structures and Adaptive Systems
%          
%                                 __/|__
%                                /_/_/_/  
%            www.dlr.de/fa/en      |/ DLR
% 
%%%%%%%%%%%%%%%%%%%%%%%%%%%%%%%%%%%%
% Content                          %
%%%%%%%%%%%%%%%%%%%%%%%%%%%%%%%%%%%%

\leveldown{\idxPDKwShortRangeForce}
\label{sec:Peridigm:QRG:Contact:Model:ShortRangeForce}
\myindex[\idxPDKeywordName]{\idxPDKwShortRangeForce}

\leveldown{Description}

A simple contact model for the interaction of non-bonded elements.

\begin{itemize}
  \item From David Littlewood:\\
  To the best of my knowledge, the short-range contact model in Peridigm is completely ad hoc. It isn't tied to anything physical as far as I know, and I'm not aware of any general usage rules. It's just a repulsive force that was implemented to give somewhat reasonable results for impact problems.  
\end{itemize}

\levelstay{Literature}

\begin{itemize}[noitemsep]
  \item \cite{SillingSA2005}
  \item \cite{PeridigmUserGuide100}
  \item \cite{LittlewoodD2011} \& \cite{LittlewoodD2015}
\end{itemize}

\levelstay{Code}

\leveldown{Release version}

Available from \href{\toolrepoversiononetwo}{version 1.2}.

\levelstay{Required compiler options}

-

\levelstay{Routines}

\begin{itemize}[noitemsep]
%   \item IO:
%   \begin{itemize}[noitemsep]
  \item \verb+/src/contact/Peridigm_ShortRangeForceContactModel.cpp+
  \item \verb+/src/contact/Peridigm_ShortRangeForceContactModel.hpp+
%   \end{itemize}
%   \item Computation:
%   \begin{itemize}[noitemsep]
%     \item \verb+/src/materials/elastic.cxx+
%     \item \verb+/src/materials/elastic.h+
%   \end{itemize}
\end{itemize}

\levelup{Input parameters}

\leveldown{List}%\mbox{}\\

\begin{tabularx}{\linewidth}{lcccX}
\toprule
Name                 & Type   & Required      & Default     & Description           \\
\midrule
Contact Model\myindex[\idxPDKeywordName]{\idxPDKwContactModel}        & string & \checkmark    & -           & Contact model type ``\idxPDKwShortRangeForce'' \\
Contact Radius\textsuperscript{\ref{enm:Peridigm:QRG:Contact:Model:ShortRangeForce:Remark:ContactRadius}}       & double & \checkmark    & -           &  \\
Spring Constant\textsuperscript{\ref{enm:Peridigm:QRG:Contact:Model:ShortRangeForce:Remark:SpringConstant}}      & double & \checkmark    & -           &  \\
Friction Coefficient\textsuperscript{\ref{enm:Peridigm:QRG:Contact:Model:ShortRangeForce:Remark:FrictionCoefficient}}\myindex[\idxPDKeywordName]{\idxPDKwFrictionCoefficient}  & double & -             & $\num{0.0}$        &  \\
Horizon\myindex[\idxPDKeywordName]{\idxPDKwHorizon}\textsuperscript{\ref{enm:Peridigm:QRG:Contact:Model:ShortRangeForce:Remark:Horizon}}              & double & -             & $\num{0.0}$ &  \\
\bottomrule
\end{tabularx}

\levelstay{Remarks}%\mbox{}\\

\begin{enumerate}[noitemsep]
  \item Contact algorithm operates on planar facets, these are created from peridynamic collocation points, see \cite{LittlewoodD2011} \& \cite{LittlewoodD2015}
  \item \label{enm:Peridigm:QRG:Contact:Model:ShortRangeForce:Remark:ContactRadius} Initial guess for contact radius: somewhat smaller than the initial node spacing, so that it doesn't add forces to material in the undeformed or slightly deformed configuration. I would also expect it to be large enough to prevent material interpenetration. 
  \item \label{enm:Peridigm:QRG:Contact:Model:ShortRangeForce:Remark:SpringConstant} From David Littlewood: The spring force parameter is determined by trial and error, although a good initial guess might be 10 times the force of the constitutive model.
  \item \label{enm:Peridigm:QRG:Contact:Model:ShortRangeForce:Remark:FrictionCoefficient} In case \texttt{Friction Coefficient} is not defined, no friction is applied by using the default value
  \item \label{enm:Peridigm:QRG:Contact:Model:ShortRangeForce:Remark:Horizon}
\end{enumerate}

\levelup{Exemplary input section}

\leveldown{XML-format}%\mbox{}\\

without friction:

\begingroup
\lstset{breaklines=true}
\begin{code}
<ParameterList name="Models">
  <ParameterList name="My Contact Model">
    <Parameter name="Contact Model" type="string" value="Short Range Force"/>
    <Parameter name="Contact Radius" type="double" value="0.2"/>
    <Parameter name="Spring Constant" type="double" value="1950.0e3"/>
  </ParameterList>
</ParameterList>
\end{code}
\endgroup

with friction:

\begingroup
\lstset{breaklines=true}
\begin{code}
<ParameterList name="Models">
  <ParameterList name="My Contact Model">
    <Parameter name="Contact Model" type="string" value="Short Range Force"/>
    <Parameter name="Contact Radius" type="double" value="0.2"/>
    <Parameter name="Spring Constant" type="double" value="1.0e12"/>
    <Parameter name="Friction Coefficient" type="double" value="0.3"/>
  </ParameterList>
</ParameterList>
\end{code}
\endgroup

\levelstay{Free format}%\mbox{}\\

Without friction:

\begingroup
\lstset{breaklines=true}
\begin{code}
Models
  My Contact Model
    Contact Model "Short Range Force"
    Contact Radius {0.8*MODEL_ESIZE_FIBRE}
    Spring Constant {10.0001*MAT_FIBRE_MODULUS_YOUNG}
\end{code}
\endgroup

\levelstay{YAML format}%\mbox{}\\

-

\levelup{List of examples}

\begin{itemize}[noitemsep]
%   \item From \texttt{examples/}:
%   \begin{itemize}[noitemsep]
%     \item \texttt{examples/tensile\_test/tensile\_test.peridigm}
%   \end{itemize}
%   \item From \texttt{test/regression/}:
%   \begin{itemize}[noitemsep]
%     \item \texttt{Contact\_Cubes/Contact\_Cubes.xml}
%   \end{itemize}
  \item From \texttt{test/verification/}:
  \begin{itemize}[noitemsep]
    \item \texttt{Contact\_Friction/Contact\_Friction.xml}
  \end{itemize}
\end{itemize}