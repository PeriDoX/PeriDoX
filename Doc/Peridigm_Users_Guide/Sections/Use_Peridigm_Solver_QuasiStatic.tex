%%%%%%%%%%%%%%%%%%%%%%%%%%%%%%%%%%%%
% Header                           %
%%%%%%%%%%%%%%%%%%%%%%%%%%%%%%%%%%%%
% 
% Revisions: 2017-04-10 Martin R�del <martin.raedel@dlr.de>
%                       Initial draft
%               
% Contact:   Martin R�del,  martin.raedel@dlr.de
%            DLR Composite Structures and Adaptive Systems
%          
%                                 __/|__
%                                /_/_/_/  
%            www.dlr.de/fa/en      |/ DLR
% 
%%%%%%%%%%%%%%%%%%%%%%%%%%%%%%%%%%%%
% Content                          %
%%%%%%%%%%%%%%%%%%%%%%%%%%%%%%%%%%%%

\levelup{\idxPDKwQuasiStatic}
\label{sec:Peridigm:QRG:Solver:QuasiStatic}
\myindex[\idxPDKeywordName]{\idxPDKwQuasiStatic}
\myindex[\idxPDKeywordName]{\idxPDKwSolver!\idxPDKwQuasiStatic|see{\idxPDKwQuasiStatic}}

\leveldown{Description}

Solver with implicit time integration.

Quasi-static, keyword \verb+QuasiStatic+, is a time integration scheme for quasi-static analysis using a nonlinear solver with modified Newton approach \cite{LittlewoodD2013_Congress}.

\levelstay{Code}

\begin{itemize}[noitemsep]
  \item \verb+/src/core/Peridigm.cpp+\\
  $\rightarrow$ \verb+PeridigmNS::Peridigm::execute+\\
  $\rightarrow$ \verb+PeridigmNS::Peridigm::executeQuasiStatic+
  \item \verb+/src/core/Peridigm.hpp+
\end{itemize}

\levelstay{Input parameters}

\leveldown{List}

\begin{tabularx}{\linewidth}{lcccX}
\toprule
Name		& Type		& Required	& Default	& Description		\\
\midrule
\multicolumn{5}{l}{Solver parameters}	\\
Initial Time	& double	& -		& 0.0		& Start time of simulation	\\
Final Time	& double	& -		& 1.0		& End time of simulation	\\
Verbose		& bool		& -		& false		& \\
Disable Heuristics & bool	& -		& false		& Disable solver heuristics	\\[1.0ex]
\multicolumn{5}{l}{Time integration}	\\[0.5ex]
Number of Load Steps	& int	& (\checkmark)	& -		& 	\\
Maximum Solver Iterations & int	& -		& 10		& 	\\
Time Steps	& int		& (\checkmark)		& -		& 	\\
Damped Newton Diagonal Scale Factor & double	& -	& 1.0001	& \\
Damped Newton Diagonal Shift Factor & double	& -	& 0.00001	& \\
Relative Tolerance & double	& -		& 1.0E-6	& 	\\
Absolute Tolerance & double	& -		& \checkmark\textsuperscript{3}	& Convergence criteria for the residual.	\\
\bottomrule
\end{tabularx}

\levelstay{Remarks}

\begin{enumerate}[noitemsep]
  \item Since the analysis is quasi-static, the values of \textit{Initial Time} and \textit{Final Time} are irrelevant%. Both values are not processed in \verb+PeridigmNS::Peridigm::executeQuasiStatic+
  \item Two types of load step input are mutually exclusive, but one of them is required in the definition
  \begin{itemize}
   \item \textit{Final Time} and \textit{Number of Load Steps}
   \item \textit{Time Steps}
  \end{itemize}
  \item A default value is calculated. In case of convergence problems, increase the value of \texttt{Absolute Tolerance}.
\end{enumerate}

\levelup{Exemplary input section}

\leveldown{XML-format}

\begin{code}
<ParameterList name="Solver">
  <Parameter name="Verbose" type="bool" value="false"/>
  <Parameter name="Initial Time" type="double" value="0.0"/>
  <Parameter name="Final Time" type="double" value="1.0"/> 
  <ParameterList name="QuasiStatic">
    <Parameter name="Number of Load Steps" type="int" value="1"/>
    <Parameter name="Absolute Tolerance" type="double" value="1.0"/>
    <Parameter name="Maximum Solver Iterations" type="int" value="10"/>
  </ParameterList>
</ParameterList>
\end{code}

\levelstay{Free format}

\begin{code}
  Solver
    Verbose "false"
    Initial Time 0.0
    Final Time 1.0
    QuasiStatic
      Number of Load Steps 1
      Absolute Tolerance 1.0
      Maximum Solver Iterations 10
\end{code}

\levelstay{YAML format}

-
  
% \levelup{Possible output variables for the material model}
% 
% \begin{multicols}{2}
% \begin{itemize}[noitemsep]
%   \item Bond\_Damage
%   \item Coordinates
%   \item Damage
%   \item Deviatoric\_Plastic\_Extension
%   \item Dilatation
%   \item Force\_Density
%   \item Lambda
%   \item Model\_Coordinates
%   \item Partial\_Stress
%   \item Surface\_Correction\_Factor
%   \item Temperature\_Change
%   \item Volume
%   \item Weighted\_Volume
% \end{itemize}
% \end{multicols}

\levelup{List of examples}

\begin{itemize}[noitemsep]
  \item From \texttt{examples/}:
  \begin{itemize}[noitemsep]
    \item \texttt{examples/tensile\_test/tensile\_test.peridigm}
  \end{itemize}
  \item From \texttt{test/regression/}:
  \begin{itemize}[noitemsep]
    \item \texttt{Interfaces/Interfaces.xml}
    \item \texttt{Pals\_Simple\_Shear/Pals\_Simple\_Shear.xml}
    \item \texttt{PrecrackedPlate/PrecrackedPlate.xml}
  \end{itemize}
  \item From \texttt{test/verification/}:
  \begin{itemize}[noitemsep]
    \item \texttt{MultipleHorizons/MultipleHorizons.xml}
  \end{itemize}
\end{itemize}