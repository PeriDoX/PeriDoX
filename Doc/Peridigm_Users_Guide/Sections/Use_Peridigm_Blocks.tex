\levelmultiup{Properties}{2}

% \leveldown{Block}
\leveldown{\idxPDKwBlocks}
\label{sec:Peridigm:QRG:Properties:Blocks}
\myindex[\idxPDKeywordName]{\idxPDKwBlocks}

\leveldown{Description}

Peridynamic property/section definition.

\levelstay{Code}

\begin{itemize}[noitemsep]
  \item \verb+/src/core/Peridigm_Block.cpp.cpp+
  \item \verb+/src/core/Peridigm_Block.cpp.hpp+
  \item \verb+/src/core/Peridigm_BlockBase.cpp+
  \item \verb+/src/core/Peridigm_BlockBase.cpp+
\end{itemize}

\levelstay{Input parameters}

\leveldown{List}

\begin{tabularx}{\linewidth}{lcccX}
\toprule
Name		& Type		& Required	& Default	& Description		\\
\midrule
Block Names	& string	& \checkmark	& -		& Block names seperated by spaces		\\
Material\myindex[\idxPDKeywordName]{\idxPDKwMaterials}	& string	& \checkmark	& -		& Name of block material	\\
Damage Model\myindex[\idxPDKeywordName]{\idxPDKwDamageModels}	& string	& -		& -		& Name of block damage model	\\
Horizon\myindex[\idxPDKeywordName]{\idxPDKwHorizon}		& double	& \checkmark\textsuperscript{\ref{enm:Peridigm:QRG:Properties:Elastic:Blocks:Remark:One}}	& -		& Block horizon	\\
\bottomrule
\end{tabularx}

\levelstay{Remarks}

\begin{enumerate}[noitemsep]
  \item \label{enm:Peridigm:QRG:Properties:Elastic:Blocks:Remark:One} For numerical reasons it is a good idea to choose units such that the horizon is roughly of order one (from the Peridigm Mailing List, by D.J. Littlewood, 07.07.17)
\end{enumerate}

\levelup{Exemplary input section}

\leveldown{XML-format}

Definition of one block:

\begingroup
\lstset{breaklines=true}
\begin{code}
<ParameterList name="Blocks">
  <ParameterList name="My Group of Blocks">
    <Parameter name="Block Names" type="string" value="block_1"/>
    <Parameter name="Material" type="string" value="My Elastic Material"/>
    <Parameter name="Horizon" type="double" value="1.75"/>
  </ParameterList>
</ParameterList>
\end{code}
\endgroup

Definition of multiple blocks with the same parameters:

\begingroup
\lstset{breaklines=true}
\begin{code}
<ParameterList name="Blocks">
  <ParameterList name="My Group">
    <Parameter name="Block Names" type="string" value="block_1 block_2"/>
    <Parameter name="Material" type="string" value="My Elastic Material"/>
    <Parameter name="Horizon" type="double" value="0.5025"/>
  </ParameterList>
</ParameterList>
\end{code}
\endgroup

Block definition with function parser, dependent on spatial position:

\begingroup
\lstset{breaklines=true}
\begin{code}
<ParameterList name="Group of Blocks E">
  <Parameter name="Block Names" type="string" value="block_5"/>
  <Parameter name="Material" type="string" value="My Material"/>
  <Parameter name="Horizon" type="string" value="0.5000001 + 0.2500001*(x-1.625) + 0.2500001*(y-1.625) + 0.2500001*(z+0.375)"/>
</ParameterList> 
\end{code}
\endgroup

\levelstay{Free format}

\begingroup
\lstset{breaklines=true}
\begin{code}
Blocks
  My Group of Blocks
    Block Names "block_1"
    Material "My Elastic Material"
    Horizon 2.0
\end{code}
\endgroup

\levelstay{YAML format}

\begingroup
\lstset{breaklines=true}
\begin{code}
Blocks:
  My Block:
    Block Names: "block_1 block_2 block_3"
    Material: "My Material"
    Horizon: 0.751
\end{code}
\endgroup
  
% \levelup{Possible output variables}

% \begin{multicols}{2}
% \begin{itemize}[noitemsep]
%   \item Bond\_Damage
%   \item Coordinates
%   \item Damage
%   \item Deviatoric\_Plastic\_Extension
%   \item Dilatation
%   \item Force\_Density
%   \item Lambda
%   \item Model\_Coordinates
%   \item Partial\_Stress
%   \item Surface\_Correction\_Factor
%   \item Temperature\_Change
%   \item Volume
%   \item Weighted\_Volume
% \end{itemize}
% \end{multicols}

\levelup{List of examples}

Basically all models in \texttt{/test/} and \texttt{/examples/}.